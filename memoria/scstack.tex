%%%%%%%%%%%%%%%%%%%%%%%%%%%%%%%%%%%%%%%%%%%%%%%%%%%%%%%%%%%%%%%%%%%%%%%%%%%%%%%%%%%%%%%%%%%%%%%%%%%%%%%%%%
% SCStack
%%%%%%%%%%%%%%%%%%%%%%%%%%%%%%%%%%%%%%%%%%%%%%%%%%%%%%%%%%%%%%%%%%%%%%%%%%%%%%%%%%%%%%%%%%%%%%%%%%%%%%%%%%

\chapter{SCStack}
\label{chap:scstack}

\par SCStack es el conjunto de herramientas que componen la Forja de desarrollo. La palabra \emph{stack} en Inglés significa \emph{pila}, con respecto a la forja se trata del conjunto de herramientas apiladas y conectadas que dan forma a la Forja.

\begin{figure}[H]
    \centering
    \includegraphics[width=0.7\textwidth]{stackstorage}
    \caption{Ejemplo de pila a partir de cajones}
    \label{fig:stackstorage}
\end{figure}

\section{Arquitectura}
\label{sec:arquitectura}

\par La arquitectura del Sistema de Gestión de la Forja se ve reflejada en el siguiente esquema:

% TODO Imagen UI -> REST Server -> API

\par El proyecto consta de tres componentes conectados entre sí para la comunicación del usuario con las herramientas para la gestión de los proyectos: La Consola de Administración, el servidor de servicios web REST y el API.

%%%%%%%%%%%%%%%%%%%%%%%%%%%%%%%%%%%%%%%%%%%%%%%%%%%%%%%%%%%%%%%%%%%%%%%%%%%%%%%%%%%%%%%%%%%%%%%%%%%%%%%%%%
% Consola de Administración
%%%%%%%%%%%%%%%%%%%%%%%%%%%%%%%%%%%%%%%%%%%%%%%%%%%%%%%%%%%%%%%%%%%%%%%%%%%%%%%%%%%%%%%%%%%%%%%%%%%%%%%%%%

\subsection{Consola de Administración}
\label{sub:consola-admin}

\par \textbf{Consola de Administración}: Se trata de la capa de la vista encargada de interactuar con el usuario. A través de la interfaz que se ejecuta sobre un navegador web, el usuario administrador gestiona los usuarios y los proyectos de la Forja: crear, editar, modificar, eliminar y buscar), el típico sistema CRUD-F\footnote{\url{http://en.wikipedia.org/wiki/Create,\_read,\_update\_and\_delete}} (\emph{CRUD: Create, Read, Update and Delete also Find}). Está diseñada con el framework Javascript \emph{jQuery} por lo que no requiere ninguna librería externa para su uso, únicamente el navegador del cliente. Las operaciones que se realizan en la capa UI se trasladan al servidor REST a través de peticiones \emph{http} asíncronas mediante las interfaces REST definidas.

% subsection consola-admin (end)

%%%%%%%%%%%%%%%%%%%%%%%%%%%%%%%%%%%%%%%%%%%%%%%%%%%%%%%%%%%%%%%%%%%%%%%%%%%%%%%%%%%%%%%%%%%%%%%%%%%%%%%%%%
% Servicio web REST
%%%%%%%%%%%%%%%%%%%%%%%%%%%%%%%%%%%%%%%%%%%%%%%%%%%%%%%%%%%%%%%%%%%%%%%%%%%%%%%%%%%%%%%%%%%%%%%%%%%%%%%%%%

\subsection{Servicio web REST}
\label{sub:rest-ws}

\par \emph{REST}: Representational State Transfer se trata de una arquitectura acuñada por \emph{Roy Fielding}\footnote{\url{http://roy.gbiv.com/}}, uno de los autores de la especificación del protocolo \emph{HTTP}. Esta arquitectura de comunicación se basa en cuatro principios:

\begin{itemize}
	\item Un protocolo cliente/servidor \emph{sin estado}: cada mensaje HTTP contiene toda la información necesaria para comprender la petición. Como resultado, ni el cliente ni el servidor necesitan recordar ningún estado de las comunicaciones entre mensajes. Sin embargo, en la práctica, muchas aplicaciones basadas en HTTP utilizan cookies y otros mecanismos para mantener el estado de la sesión.

	\item Un \emph{conjunto de operaciones} bien definidas que se aplican a todos los recursos de información: HTTP en sí define un conjunto pequeño de operaciones, las más importantes son \textbf{POST}, \textbf{GET}, \textbf{PUT} y \textbf{DELETE}.

	\item Una sintaxis \emph{universal} para identificar los recursos. En un sistema REST, cada recurso es direccionable únicamente a través de su URI.

	\item El uso de \emph{hipermedios}: tanto para la información de la aplicación como para las transiciones de estado de la aplicación. Como resultado de esto, es posible navegar de un recurso REST a muchos otros, simplemente siguiendo enlaces sin requerir el uso de registros u otra infraestructura adicional.
\end{itemize}

\par El uso de los servicios REST para la comunicación mediante el protocolo http con la Consola de Administración agiliza la adaptación de la funcionalidades al cliente, el tráfico y las posibles adaptaciones de nuevos clientes para el acceso a través de cualquier plataforma.

\begin{figure}[H]
    \centering
    \includegraphics[width=1\textwidth]{restlet}
    \caption{Diseño de un API RESTful}
    \label{fig:restful-api}
\end{figure}

\par \textbf{Servicio web REST}: Es el componente encargado de gestionar las peticiones http a los servicios REST de la Consola de Administración y el API de SidelabCode. Define e implementa la lógica a seguir para la construcción de un proyecto a través de la UI mediante las llamadas ordenadas al API de cada una de las herramientas involucradas en el servicio como componentes de la forja.

\par Los servicios REST ofrecen tres interfaces distintas accesibles a través de http: XML, JSON y HTML para la comunicación la API. Además, hay que desatacar que el servicio web es servido a través del framework \emph{Restlet} a través de Internet de forma continua y remota a cualquier usuario o proceso cliente.

\par Se encarga de la seguridad y validación para las operaciones permitidas o denegadas del usuario de la UI, proporcionando, según su rol, determinados servicios a través de la Consola de Administración, buscar proyectos, crear usuarios, editar proyectos, crear proyectos, etc.

% subsection rest-ws (end)

%%%%%%%%%%%%%%%%%%%%%%%%%%%%%%%%%%%%%%%%%%%%%%%%%%%%%%%%%%%%%%%%%%%%%%%%%%%%%%%%%%%%%%%%%%%%%%%%%%%%%%%%%%
% API
%%%%%%%%%%%%%%%%%%%%%%%%%%%%%%%%%%%%%%%%%%%%%%%%%%%%%%%%%%%%%%%%%%%%%%%%%%%%%%%%%%%%%%%%%%%%%%%%%%%%%%%%%%

\subsection{API}
\label{sub:api}

\par \textbf{API}: La API es el núcleo funcional de SCStack, coordina y realiza las tareas para cada componente de la forja con respecto a los usuarios y proyectos involucrados. Traslada las órdenes ejecutadas en la capa UI a las distintas herramientas para configurar su funcionamiento. Está conectada a todos los servicios que ofrece pivotando a través del directorio LDAP: Control de Versiones, Directorios, Integración Continua, ITS, Revisiones, Gestión de Dependencias, Seguridad y Autenticación que analizaremos extensamente más adelante componente a componente.

% subsection api (end)

% section arquitectura (end)

%%%%%%%%%%%%%%%%%%%%%%%%%%%%%%%%%%%%%%%%%%%%%%%%%%%%%%%%%%%%%%%%%%%%%%%%%%%%%%%%%%%%%%%%%%%%%%%%%%%%%%%%%%
% Aprovisionamiento (Puppet)
%%%%%%%%%%%%%%%%%%%%%%%%%%%%%%%%%%%%%%%%%%%%%%%%%%%%%%%%%%%%%%%%%%%%%%%%%%%%%%%%%%%%%%%%%%%%%%%%%%%%%%%%%%

\section{Aprovisionamiento (Puppet)}
\label{sec:puppet}

\par \emph{Aprovisionamiento}: 'Accción o efecto de aprovisionar', \emph{aprovisionar}: abastecer\footnote{Definición del diccionario de la RAE - \url{http://buscon.rae.es/drae/?type=3&val=aprovisionar&val_aux=&origen=REDRAE}}. Aplicado a la Ingeniería del Software el Aprovisionamiento nos provee de los componentes necesarias para construir una solución.

\par El aprovisionamiento trata de la automatización de tareas para construir el entorno deseado, en este caso la instalación de SidelabCode Stack. La evolución en el modelo de instalación para facilitar la ejecución de módulos, configuración y comunicación entre los distintos componentes.

\par La herramienta empleada para el aprovisionamiento es \textbf{Puppet} de la compañía \emph{PuppetLabs}\footnote{\url{https://puppetlabs.com/}}.

\begin{figure}[H]
    \centering
    \includegraphics[width=0.7\textwidth]{puppet-labs-logo}
    \caption{Puppet Labs Logo}
    \label{fig:puppet-labs}
\end{figure}

\par \emph{Puppet}: es una herramienta \emph{Software Libre}\footnote{\url{https://puppetlabs.com/puppet/puppet-open-source/}} de aprovisionamiento desarrollada en \emph{Ruby}\footnote{\url{https://github.com/puppetlabs/puppet}}. Gestiona la infraestructura a través de su ciclo de vida, desde el aprovisionamiento y la configuración de parches automatizando la ejecución de órdenes para la instalación y configuración del entorno.

\begin{itemize}
	\item Automatizar tareas repetitivas.
	\item Desplegar rápidamente aplicaciones críticas.
	\item Gestionar proactivamente el cambio.
	\item Escalar de 10 servidores para 1000.
	\item Instalaciones locales o en la nube.
\end{itemize}

\par \emph{Puppet} utiliza un enfoque declarativo, basado en el modelo de automatización:
\begin{itemize}
	\item \emph{Definir} el estado deseado de la configuración de la infraestructura mediante lenguaje de configuración declarativa de Puppet.
	\item \emph{Simular} los cambios de configuración antes de la ejecución.
	\item \emph{Corroborar} el estado final mediante despliegues automáticos comprobando las posibles desviaciones en la configuración.
	\item \emph{Informe} sobre las diferencias entre los estados reales y deseados y cualquier cambio que haya hecho cumplir el estado deseado.
\end{itemize}

\begin{figure}[H]
    \centering
    \includegraphics[width=0.7\textwidth]{howpuppetworks}
    \caption{Ciclo de vida de los módulos Puppet}
    \label{fig:howpuppetworks}
\end{figure}

\par El diseño del aprovisionamiento a través de Puppet se basa en módulos. Cada uno de estos módulos se encarga de gestionar la instalación y configuración del componente definido.

\par Que aporta Puppet ? Proporciona un control sobre la instalación de cada herramienta y la configuración asociada por defecto que se define. De esta forma el proceso se instalación se automatiza permitiendo la replicación del mismo, completo o por módulos en diferentes entornos, locales o virtuales. La instalación a partir de módulos ha de definir una cadena de dependencias entre cada uno de ellos de manera que se vayan habilitando funcionalidades e interacciones.

\par En SidelabCode Stack Puppet es el encargado de gestionar la instalación y configuración de cada uno de los componentes mediante módulos Puppet independientes para completar el proceso de instalación entre 5 y 10 minutos.

%%%%%%%%%%%%%%%%%%%%%%%%%%%%%%%%%%%%%%%%%%%%%%%%%%%%%%%%%%%%%%%%%%%%%%%%%%%%%%%%%%%%%%%%%%%%%%%%%%%%%%%%%%
% Componentes
%%%%%%%%%%%%%%%%%%%%%%%%%%%%%%%%%%%%%%%%%%%%%%%%%%%%%%%%%%%%%%%%%%%%%%%%%%%%%%%%%%%%%%%%%%%%%%%%%%%%%%%%%%

\section{Componentes}
\label{sec:componentes}

\par En este apartado se van a analizar las distintas partes o componentes que integran la Forja SidelabCode Stack y el papel que desempeñan en el funcionamiento del sistema. Partiendo del esquema en donde se refleja a grandes la arquitectura de la Forja SidelabCode ~\ref{fig:arquitectura-scstack} y los componentes descritos en el capítulo \nameref{chap:procesos-desarrollo} a la hora de gestionar el proceso \emph{Iterativo e Incremental}.

\begin{figure}[H]
    \centering
    \includegraphics[width=0.7\textwidth]{arquitectura-base}
    \caption{Arquitectura SidelabCode Stack}
    \label{fir:arquitectura-scstack}
\end{figure}

%%%%%%%%%%%%%%%%%%%%%%%%%%%%%%%%%%%%%%%%%%%%%%%%%%%%%%%%%%%%%%%%%%%%%%%%%%%%%%%%%%%%%%%%%%%%%%%%%%%%%%%%%%
% OpenLDAP
%%%%%%%%%%%%%%%%%%%%%%%%%%%%%%%%%%%%%%%%%%%%%%%%%%%%%%%%%%%%%%%%%%%%%%%%%%%%%%%%%%%%%%%%%%%%%%%%%%%%%%%%%%

\subsection{Usuarios, roles y grupos}
\label{sub:usuarios-roles-grupos}

\begin{figure}[H]
    \centering
    \includegraphics[width=0.7\textwidth]{OpenLDAP-logo}
    \caption{OpenLDAP logo}
    \label{fig:openldap-logo}
\end{figure}

\par Gestión de usuarios, roles, grupos y proyectos a través de \emph{OpenLDAP}\footnote{OpenLDAP - \url{http://www.openldap.org/}}. SCTack utiliza la tecnología de directorios \emph{LDAP} como sistema de autenticación y de información centralizado de la Forja Software. En dicho servidor de directorios se almacena información relativa a todos los usuarios, proyectos software y repositorios de la Forja. Esta tecnología, además, cuenta con la ventaja de que la mayoría de aplicaciones web con sistemas de autenticación ofrecen interfaces que garantizan una completa integración con directorios LDAP, este es el caso de Redmine, Drupal, Wordpress y el propio servidor web Apache. La implementación de este protocolo en la Forja Sidelab se lleva a cabo mediante un servidor de directorios muy estable y de libre distribución que es OpenLDAP.

\subsection{API OpenLDAP}
\label{sub:api-openldap}

\par Debido a que el directorio LDAP es la estructura de información centralizada de la Forja, cualquier tipo de acción que se quiera realizar sobre el sistema requerirá el acceso por parte de la API a este servicio de directorios, bien para la recuperación de datos en las consultas o para la manipulación de registros a la hora de crear, editar o borrar usuarios o proyectos.

\par Todas las acciones de la Consola de Administración pasan a través de la API que comunica con OpenLDAP dando acceso y generando las distintas autenticaciones en cada una de las herramientas interconectadas basándose en los registros de OpenLDAP como generador y autenticador de credenciales.

% subsection api-openldap (end)

% subsection usuarios-roles-grupos (end)

\subsection{Gesti\'on de Requisitos ITS}
\label{sub:its}

\par Gestión de requisitos a través del Issue Tracking System \emph{Redmine}, Backlog plugin para el desarrollo Iterativo e Incremental, documentación a partir de la Wiki y WYSIWYG.

% subsection its (end)

\subsection{Repositorios de Código}
\label{sub:repositorios}

\par Reposiotorios centralizados y distribuidos: SVN y Git.

% subsection repositorios (end)

\subsection{Desarrollo (Maven y TDD)}
\label{sub:mv-tdd}

\par Maven, TDD y desarrollo por ramas con Git.

% subsection mv-tdd (end)

\subsection{Revisión de Código}
\label{sub:gerrit}

\par Gerrit.

% subsection gerrit (end)

\subsection{Integración Continua}
\label{sub:ci-jenkins}

\par Jenkins CI.

% subsection ci-jenkins (end)

\subsection{Gestión de distribuciones y dependencias.}
\label{sub:distribuciones-dependencias}

\par Directorios compartidos, OpenSSH y Archiva para la gestión de dependencias.

% subsection distribuciones-dependencias (end)

% subsection componentes (end)

%%%%%%%%%%%%%%%%%%%%%%%%%%%%%%%%%%%%%%%%%%%%%%%%%%%%%%%%%%%%%%%%%%%%%%%%%%%%%%%%%%%%%%%%%%%%%%%%%%%%%%%%%%
% Interoperabilidad
%%%%%%%%%%%%%%%%%%%%%%%%%%%%%%%%%%%%%%%%%%%%%%%%%%%%%%%%%%%%%%%%%%%%%%%%%%%%%%%%%%%%%%%%%%%%%%%%%%%%%%%%%%

\section{Interoperabilidad}
\label{sec:interoperabilidad}

\par Interoperabilidad entre las herramientas que componen la forja. API Rest.

% section interoperabilidad (end)

%%%%%%%%%%%%%%%%%%%%%%%%%%%%%%%%%%%%%%%%%%%%%%%%%%%%%%%%%%%%%%%%%%%%%%%%%%%%%%%%%%%%%%%%%%%%%%%%%%%%%%%%%%
% Pruebas y Validación
%%%%%%%%%%%%%%%%%%%%%%%%%%%%%%%%%%%%%%%%%%%%%%%%%%%%%%%%%%%%%%%%%%%%%%%%%%%%%%%%%%%%%%%%%%%%%%%%%%%%%%%%%%

\section{Pruebas y Validación}
\label{sec:pruebas-validacion}

\par Pruebas a trav\'es de virtualizaci\'on de sistemas operativos mediante herramientas libres; Vagrant, kvm, puppet.

% section pruebas-validacion (end)
