\chapter{SCStack}
\label{chap:scstack}

* SCStack

\section{Arquitectura}
\label{sec:arquitectura}

\begin{comment}
Este punto sobre la arquitectura de la forja no me queda demasiado claro, es decir, no se definirla.

La arquitectura está definida en el documento de proyecto fin de carrera que te pasé. Básicamente la arquitectura está basada en un LDAP contra el que todas las demás herramientas de la forja se autentican. En tu caso, a esa arquitectura tienes que añadirle un Gerrit que gestiona repositorios git.
\end{comment}

% section arquitectura (end)

\section{Provisionamiento (puppet)}
\label{sec:puppet}

% section puppet (end)

\section{Componentes}
\label{sec:componentes}

\begin{comment}
Componentes, ¿ aquí se definirían las herramientas a instalar ?

Efectivamente, qué herramientas instala la forja (Se supone que las has descrito en la parte de procesos de desarrollo de forma genérica, aquí se dan nombres concretos). 
\end{comment}

\begin{itemize}
	\item Gesti\'on de Requisitos.
	\item Arquitectura.
	\item Desarrollo.
	\item Test.
	\item Issue tracking system.
	\item Continuous Integration.
	\item Release Management.
\end{itemize}

% subsection componentes (end)

\section{Interoperabilidad}
\label{sec:interoperabilidad}

\par Interoperabilidad entre las herramientas que componen la forja. API Rest.

% section interoperabilidad (end)

\section{Pruebas y Validación}
\label{sec:pruebas-validacion}

\par Pruebas a trav\'es de virtualizaci\'on de sistemas operativos mediante herramientas libres; Vagrant, kvm, puppet.

% section pruebas-validacion (end)
