\chapter{Motivaci\'on}
\label{chap:motivacion}

\par En el mismo punto, damos la bienvenida a \emph{SidelabCode Stack}, la herramienta ALM para el proceso de desarrollo.

\par Empezaremos ubicando la forja de desarrollo SidelabCode Stack. ¿ Cual es la motivaci\'on que nos lleva a implementar un nuevo ALM ?

\par Debido a la necesidad de tener un esqueleto para el desarrollo de un proyecto a trav\'es del uso de herramientas de Software Libre actuales generando una nueva herramienta de Software Libre.

\par Partiendo del principio DRY (\emph{Don't Repeat Yourself}) y de la mano de \emph{No reinventar la rueda} el proyecto surgi\'o con la necesidad de crear una forja de desarrollo Libre y usable en distintos entornos.

\par Despu\'es de analizar Las posibles soluciones existentes, se opt\'o por la unificaci\'on de las herramientas evaluadas para la generaci\'on de una nueva forja. 

\par Despu\'es de analizar estas soluciones ALM existentes se opt\'o por crear una nueva Forja a partir de las herramientas necesarias para el proceso de desarrollo de Software de Calidad, siguiendo las metodolog\'ias \'agiles existentes.

\par La integraci\'on del proceso de desarrollo en la implementaci\'on de una Forja ALM es el punto clave de SidelabCode Stack.

En primer lugar se ha elaborado un esquema conceptual de la arquitectura t\'ecnica que
permita la integraci\'on de los servicios, teniendo en consideraci\'on que la arquitectura de la
aplicaci\'on debe lograr los siguientes objetivos:

Proveer un nivel de rendimiento suficiente para cada uno de los requerimientos de
   negocio o subsistemas que lo componen.

Ser escalable para cumplir las demandas de los usuarios durante el ciclo de vida de la
   aplicaci\'on, pudiendo absorber el impacto en el rendimiento fruto de la popularidad de la
   misma en el medio-largo plazo.

Ser efectiva: la modificaci\'on de una capacidad espec\'ifica no requerir\'a el cambio de la
      plataforma completa, ni un cambio en el despliegue de la soluci\'on.

Ser mantenible y que sus funcionalidades sean f\'acilmente extensibles y reutilizables.

Permitir la f\'acil portabilidad de los datos de los proyectos entre forjas.

%%%%%%%%%%%%%%%%%%%%%%%%%%%%%%%%%%%%%%

\chapter{Objetivos}
\label{chap:objetivos}

\par Constancia, metodolog\'ia, facilidad de uso, herramientas al alcance de cualquier desarrollador dentro de un proceso de desarrollo para crear software de calidad.

\par Integrar el desarrollo completo de un proyecto:

\begin{itemize}
	\item Creaci\'on del proyecto, usuarios, permisos y roles.
	\item Carpetas p\'ublicas/privadas
	\item An\'alisis de Requerimientos.
	\item ITS Issue Tracking System.
	\item Repositorio de c\'odigo.
	\item Gesti\'on de librer\'ias.
	\item TDD Test Driven Development.
	\item Gesti\'on del repositorio, parches, actualizaciones, ramas.
	\item Desarrollo e implementaci\'on.
	\item An\'alisis de C\'odigo.
	\item Despliegues en diferentes entornos.
	\item Integraci\'on Continua.
\end{itemize}

\begin{comment}

Obtener una descripci\'on detallada de los servicios a ofrecer por la forja.
Bas\'andose en los requerimientos funcionales definidos en el anterior proyecto se debe
realizar una abstracci\'on de los servicios que debe ofrecer la Forja de Nueva
Generaci\'on conform\'andose as\'i el cat\'alogo de servicios. Para cada uno de estos
   servicios se definir\'a su misi\'on y se describir\'an sus principales funcionalidades y
     caracter\'isticas.

Definir la arquitectura software que permita la cohesi\'on de los diferentes
    servicios. Una vez definidos los servicios a ofrecer por la forja, se debe dar cohesi\'on a
    los mismos dise\~nando un n\'ucleo de servicios internos que permita la integraci\'on
      funcional y t\'ecnica de los diferentes servicios y que provea de servicios comunes de
      base a toda la plataforma, definiendo as\'i servicios de almacenamiento de datos,
      interoperatividad, as\'i como elementos que si bien no son imprescindibles para el
      funcionamiento de la forja, si que van a aportar valor a la misma a\~nadiendo
      posibilidades como la computaci\'on en la nube, el almacenamiento extendido y la
      provisi\'on de servicios de infraestructura de proyectos.

Evaluar las tecnolog\'ias de referencia que puedan dar cobertura total o parcial a
    los requisitos definidos. Una vez definido el cat\'alogo de servicios a ofrecer por la
    forja, as\'i como la arquitectura de soporte a los mismos, se hace necesaria la
    evaluaci\'on de las tecnolog\'ias existentes que puedan dar cobertura parcial o total a los
     requerimientos definidos. Esta evaluaci\'on debe dar respuesta tanto a la viabilidad
     t\'ecnica de los mismos como su viabilidad econ\'omica.

Obtener una Estimaci\'on del alcance econ\'omico del proyecto. Uno de los
     principales objetivos de este documento consiste en la construcci\'on de una matriz con
     los diferentes elementos de coste que marcar\'an el desarrollo y mantenimiento de la
     futura Forja de Nueva Generaci\'on, desglos\'andose \'esta en tres apartados:

    Desarrollos necesarios, tanto adaptaciones de herramientas como nuevos desarrollos.
    
    Hardware m\'inimo necesario para el despliegue de la soluci\'on F:NG.
    
    Administraci\'on y soporte tras implantaci\'on de la F:NG.

Definir los esquemas de despliegue en los diferentes escenarios. Una vez definida
la arquitectura de referencia se deber\'an dise\~nar los posibles escenarios de despliegue
de las diferentes forjas y subforjas
\end{comment}

\par Generar datos aunque no se utilicen, es decir, grano gordo/grano fino.