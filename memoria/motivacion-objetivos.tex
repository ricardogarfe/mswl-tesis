%%%%%%%%%%%%%%%%%%%%%%%%%%%%%%%%%%%%%%%%%%%%%%%%%%%%%%%%%%%%%%%%%%%%%%%%%%%%%%%%%%%%%%%%%%%%%%%%%%%%%%%%%%%
%   \copyright 2013 Ricardo García Fernández - ricardogarfe [at] gmail [dot] com.
%
%    This work is licensed under a Creative Commons 3.0 Unported License.
%    To view a copy of this license visit:
% 
%    http://creativecommons.org/licenses/by/3.0/legalcode
%%%%%%%%%%%%%%%%%%%%%%%%%%%%%%%%%%%%%%%%%%%%%%%%%%%%%%%%%%%%%%%%%%%%%%%%%%%%%%%%%%%%%%%%%%%%%%%%%%%%%%%%%%

\chapter{Motivaci\'on}
\label{chap:motivacion}

\par En el mismo punto, damos la bienvenida a \emph{SidelabCode Stack}, la herramienta ALM para el proceso de desarrollo.

\par Empezaremos ubicando la forja de desarrollo SidelabCode Stack. ¿ Cual es la motivaci\'on que nos lleva a implementar un nuevo ALM ?

\par Debido a la necesidad de tener un esqueleto para el desarrollo de un proyecto a trav\'es del uso de herramientas de Software Libre actuales generando una nueva herramienta de Software Libre.

\par Partiendo del principio DRY (\emph{Don't Repeat Yourself}) y de la mano de \emph{No reinventar la rueda} el proyecto surgi\'o con la necesidad de crear una forja de desarrollo Libre y usable en distintos entornos.

\par Despu\'es de analizar Las posibles soluciones existentes, se opt\'o por la unificaci\'on de las herramientas evaluadas para la generaci\'on de una nueva forja. 

\par Despu\'es de analizar estas soluciones ALM existentes se opt\'o por crear una nueva Forja a partir de las herramientas necesarias para el proceso de desarrollo de Software de Calidad, siguiendo las metodolog\'ias \'agiles existentes.

\par La necesidad de crear una herramienta que aglutinase la gestión de las tareas, el código fuente y la orientación del desarrollo a los Tests. Todo el proceso de desarrollo unificado para facilitar el trabajo.

\par La integraci\'on del proceso de desarrollo en la implementaci\'on de una Forja ALM es el punto clave de SidelabCode Stack. Para crear un entorno adecuado se tuvieron en cuenta distintas características en el proceso de creación:

\begin{itemize}
	\item Ser efectiva: la modificaci\'on de una capacidad espec\'ifica no requerir\'a el cambio de la plataforma completa, ni un cambio en el despliegue de la soluci\'on.
	\item Ser mantenible y que sus funcionalidades sean f\'acilmente extensibles y reutilizables.
	\item Permitir la f\'acil portabilidad de los datos de los proyectos entre distintas forjas para la migración.
\end{itemize}

%%%%%%%%%%%%%%%%%%%%%%%%%%%%%%%%%%%%%%

\chapter{Objetivos}
\label{chap:objetivos}

\par Constancia, metodolog\'ia, facilidad de uso basado en herramientas al alcance de cualquier desarrollador dentro de un proceso de desarrollo para crear software de calidad. Estos son los principios en los que se basa la implementación de SidelabCode Stack como forja de desarrollo teniendo como objetivos tangibles la unificación del uso de las herramientas necesarias para cumplir con éstos como objetivo.

\par Integrar el desarrollo completo de un proyecto a partir de estos objetivos convertidos en los siguientes requerimientos.

\section{Proyecto, usuarios, permisos y roles }
\label{sec:proyecto-usuarios}

\par La gestión del proyecto a partir de la definición de los usuarios, permisos y roles definidos para el mismo. Centralizar esta gestión para así obtener un control fluido y centralizado de los \emph{"poderes"} otorgados según el rango del usuario a través de una herramienta.

% section proyecto-usuarios (end)

\section{Carpetas compartidas}
\label{sec:carpetas-compartidas}

\par Generar un espacio de carpetas p\'ublicas/privadas para cada uno de los proyectos a partir de los permisos de cada usuario. Asegurando el acceso público/privado a los recursos que se publiquen.

% section carpetas-compartidas (end)

\section{ITS}
\label{sec:its}

\par Empezando por el An\'alisis de Requerimientos en un proyecto es necesaria una herramienta que ayude en el proceso de desarrollo, en esta caso un \emph{ITS Issue Tracking System} para la gestión y el mantenimiento de las tareas de cada proyecto. La piedra angular de todo proyecto de software. El ITS otorga un control total del flujo de trabajo y responsabilidades a los usuarios del proyecto, una base de documentación y gestión del tiempo prioritaria en todo proyecto de software.

\par Utilizar el ITS para gestionar las tareas y los posibles errores dentro de cada proyecto para planificar la corrección o la implementación de éstas, dividiendo las entregas del proyecto en versiones durante el proceso continuo de desarrollo.

% section its (end)

\section{Repositorio de código}
\label{sec:repositorio-codigo}

\par El Repositorio de c\'odigo asociado a cada proyecto. Sin repositorio de código no hay proyecto. De esta forma la cooperación entre usuarios de un proyecto se agiliza bajo el manto de un sistema de control de versiones de código fuente. Esta herramienta produce la información necesaria para generar un histórico de cambios y poder evaluar la evolución del proyecto.

\par Además se ha de dotar de una herramienta para la Gesti\'on del repositorio. Es decir, la gestión de parches, actualizaciones y desarrollo por ramas dentro de un entorno controlado.

% section repositorio-codigo (end)

\section{Gestión de librerías}
\label{sec:gestion-librerias}

\par La gesti\'on de librer\'ias centralizada para los proyectos dentro de la forja, ofrece la posibilidad de interconectar distintas librerías dentro de todo el entorno. Publicando las librerías clasificadas a través de las distintas versiones en un repositorio de librerías común. La reutilización del código y la posibilidad de mejora entre distintos proyectos que compartan librerías mejora la calidad del mismo.

% section gestion-librerias (end)

\section{TDD Test Driven Development}
\label{sec:tdd}

\par El desarrollo dirigido por Tests \emph{TDD} ayuda a la generación de código de calidad, legible y reutilizable. Siguiendo este proceso, el tiempo dedicado a la replicación de errores disminuye por lo que se puede dedicar más tiempo a la corrección del error en si.

\par Otorga datos válidos para el análisis del código y controlar la evolución de la cobertura de Tests para poder evaluar la deuda técnica. Genera los datos suficientes para facilitar la corrección de la misma.

% section tdd (end)

\section{CI - Integración Continua}
\label{sec:integracion-continua}

\par El TDD, el desarrollo por ramas y los despliegues en diferentes entornos completan un proceso de desarrollo que culmina en la integración continua del proyecto. Este es el control al más alto nivel dentro de la parte técnica en donde el control de la calidad del código se toma como estandarte. Prevé los errores previos entre distintas versiones del código al unificar las ramas, fallos en los tests y validación de la calidad del código.

\par Crear despliegues en distintos entornos o replicar entornos similares al entorno final desde esta herramienta para minimizar los errores o poder replicar fácilmente los mismos cuando surjan.

% section integracion-continua (end)

\section{Interoperabilidad}
\label{sec:interoperabilidad}

\par La interoperabilidad entre todas las herramientas para gestionar la configuración a través de una herramientas centralizada. Este apartado es donde se muestra el potencial de la forja de desarrollo ya que la configuración se expande o replica a las demás herramientas dando vida a la comunicación entre ellas para crear la forja.

\par Se transforma la configuración inicial y se establecen los protocolos de comunicación entre cada una de las herramientas dentro del proceso de desarrollo habiendo definido los pasos a seguir a partir de las \emph{API}s existentes.

% section interoperabilidad (end)