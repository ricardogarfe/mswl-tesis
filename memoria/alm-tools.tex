%%%%%%%%%%%%%%%%%%%%%%%%%%%%%%%%%%%%%%%%%%%%%%%%%%%%%%%%%%%%%%%%%%%%%%%%%%%%%%%%%%%%%%%%%%%%%%%%%%%%%%%%%%%
%   \copyright 2013 Ricardo García Fernández - ricardogarfe [at] gmail [dot] com.
%
%    This work is licensed under a Creative Commons 3.0 Unported License.
%    To view a copy of this license visit:
% 
%    http://creativecommons.org/licenses/by/3.0/legalcode
%%%%%%%%%%%%%%%%%%%%%%%%%%%%%%%%%%%%%%%%%%%%%%%%%%%%%%%%%%%%%%%%%%%%%%%%%%%%%%%%%%%%%%%%%%%%%%%%%%%%%%%%%%

\chapter{ALM Tools - Forjas de desarrollo}
\label{chap:almtools}

\begin{comment}
* ALM Tools
    * Qué es una forja
    * Objetivos
    * Componentes
        * Estudio del arte de forjas
    * Problemática -> administración, costes...
        * Tablas comparativas
    * Algunos ejemplos y sus limitaciones
    * Conclusiones del estudio de forjas
\end{comment}

\par ALM Tools: \emph{Application Lifecycle Management Tools}. \emph{La gesti\'on del ciclo de vida de una aplicaci\'on}, es decir el conjunto de herramientas encargadas de guiar al desarrollador a trav\'es de un camino basado en metodolog\'ias, que establecen el ciclo de vida, para la desarrollo de un Software hacia su estado del arte.

\par En el desarrollo de SidelabCode Stack se ha buscado la integración del proceso de desarrollo Iterativo e Incremental a trav\'es de las ALM Tools como veh\'iculo. No existe en si mismo una herramienta ALM, sino que la herramienta ALM gobierna a las herramientas que participan en el proceso de desarrollo desarrollando una guía de buenas prácticas.

\section{Historia}
\label{sec:historia}

\par En todas las empresas o comunidades que desarrollan Software siempre se aplica un proceso de desarrollo a la creación del producto. Cada una utiliza distintas herramientas para la gesti\'on de los proyectos de Software, gestor de correo, gestor de incidencias, repositorio de c\'odigo, integraci\'on continua. Pero en la mayor\'ia de los casos de una forma dispar y sin seguir ninguna convención.

\par A veces el no conocimiento de otras herramientas o la no inclusión de nuevas puede hacer que el desarrollo del proyecto no mejore, partiendo de la base de que el desarrollo puede ser óptimo para las herramientas utilizadas, carece de perspectivas de mejora a corto plazo.

\par Si se opta por la integración de una nueva herramienta en el proceso de desarrollo el coste de integración se habría de evaluar ya que se debería dedicar un esfuerzo a la integración ad-hoc de la nueva herramienta para el uso en este mismo entorno con el coste que conlleva, evaluación, test, integración, interoperabilidad, es decir un nuevo proyecto dentro del mismo proyecto.

\par Después de esta integración en el proceso de desarrollo en la empresa habría que hacer un esfuerzo para salvaguardar la información a través de las distintas herramientas por separado.

\par El proceso de unificación y reutilización de herramientas a los desarrolladores nos puede parecer familiar si lo comparamos con el uso de los Frameworks a principios de los años 2000. Muchas empresas o comunidades empezaron a adecuar e implementar sus desarrollos en base a un Framework creado por ellos mismos. Estos Frameworks se adecuan a sus requerimientos pero su uso era interno en la empresa y por lo tanto lejano a los estándares. Uno de los más famosos es sin duda el caso de Spring, proyecto de más de 10 años de edad que goza de buena salud y aceptación, incluso equivoca a algunos entre Java y Spring. En este pequeño paralelismo podemos encontrar el estado de las forjas de desarrollo ALM Tools, cuando el proyecto requiere de herramientas para facilitar su ciclo de vida y se van ensamblando una tras otra, que perfectamente las podemos llamar librerías, en una integración \textbf{ad-hoc} y siguiendo unos pasos repetitivos en cada nuevo proyecto, en los que humanamente todos nos podemos equivocar debido a que depende de cada uno seguir cada uno de los pasos. Estas herramientas tienden a convertirse en pequeños estándares dentro de cada grupo de desarrolladores y a repetirse en futuros proyectos, pero debido a los desarrollos \textbf{ad-hoc} carecen de escalabilidad e integración con nuevas soluciones de una forma ágil, es un escollo actualizar y por otro lado replicar un estándar para la implantación de la forja, no se tiende a dejar puertas abiertas para que más adelante el herramienta mejore. Se podría definir como uno de los casos de inanición en el desarrollo de Software o muerte por éxito.

\par En este punto es donde entra la famosa interoperabilidad entre las herramientas que define una comunicación estándar. El punto clave de las ALM Tools, la \textbf{integración e interacción de las herramientas} dentro de un proceso de desarrollo.

\par En post de evitar la falta de replicación, además de la importancia de la interoperabilidad, se ha de tener en cuenta la replicación del contenido o la gestión de la instalación de un ALM. Siempre se ha de pensar mirando un paso por delante. No es necesario implementar las mejoras pero sí, dejar un espacio o conector para que casen bien. Un ejemplo que puede ilustrar esta frase es la programación basada en Interfaces en Java, ya que las Intefaces ofrecen soluciones para implementar a medida a partir de un \emph{esqueleto}, si se actualiza la Interfaz (en este caso es el esqueleto de la clase) para añadir una nueva funcionalidad con un método, los métodos anteriores mantienen su comportamiento dentro de cada clase que la implementa y adquieren la posibilidad de aumentar la funcionalidad implementando la nueva solución, adecuada a su entorno, de esta forma la interoperabilidad entre las clases que utilicen esta Interfaz también se mantiene ya que todas implementan las funcionalidades de la Interfaz como clase en la que pivotan.

% subsection historia (end)

\section{Qué es una forja}
\label{sec:que-es}

\par Partimos de la definición que ofrece \emph{Cenatic} en el título de su estudio sobre forjas:

\begin{quote}
    \emph{Entorno de desarrollo colaborativo de Software}
\end{quote}

\par Una forja de desarrollo es una herramienta que actúa como elemento catalizador de este proceso abierto de desarrollo. Las forjas juegan un papel clave para aprovechar las ventajas de las metodologías, en este caso a través del proceso de desarrollo Iterativo e Incremental, aportando múltiples ventajas:

\begin{itemize}
	\item Mayor eficiencia.
	\item Mejor calidad en el producto final.
	\item Reutilización de esfuerzos.
	\item Modularidad.
	\item Adaptación a estándares.
	\item Mayor agilidad en el proceso de desarrollo.
\end{itemize}

\par Esta herramienta permite centrar toda la atención y potencial del desarrollador en el mismo desarrollo. Facilita la evolución del código desarrollado y el seguimiento del proyecto a todos los niveles.

\par Además las forjas de desarrollo aportan la sencillez para instaurar este entorno de desarrollo en el grupo de trabajo. Facilita la instalación, la replicación de contenido, la comunicación y la publicación de resultados del desarrollo de cada proyecto al alance de los usuarios de la forja.

\par Una forja es una comunidad en donde convergen proyectos de software (en el caso que nos atañe) a través de las interacciones de los usuarios con el código. Toda la información gira en torno al repositorio de código. El valor aportado por una forja en pos del repositorio por si sólo es la interoperabilidad con otras herramientas para la mejora de la gestión del desarrollo del proyecto; manuales, tickets, diagramas, planificaciones, wiki, revisiones, roles.

\par Tener las herramientas necesarias no es suficiente para obtener un desarrollo fluido, fiable y deseable. Se han de conocer las herramientas que componen la forja, no las herramientas en si, sino su funcionalidad, para que de esta forma se pueda definir un proceso de desarrollo basado en las herramientas existentes o añadiendo nuevas herramientas para que se adapten al proceso de desarrollo elegido. 

\par \emph{La forja ha de trabajar para los usuarios, pero los usuarios han de saber como hacer que la forja trabaje para ellos.}

% section que-es (end)

\section{Objetivos}
\label{sec:objetivos}

\par El objetivo principal de una forja de desarrollo es poder sacar el máximo partido al desarrollo del proyecto ayudando a establecer el proceso de desarrollo elegido para cada proyecto dentro de un entorno integrado de trabajo.

\par Disponer de un entorno integrado de trabajo en el que concentrar los esfuerzos de usuarios permite también obtener una serie de ventajas desde el punto de vista de la potenciación del propio proceso de desarrollo, desde varios puntos de vista:

\begin{itemize}
	\item \emph{Mejor control de esfuerzos}: Ya que todos los usuarios están en contacto a través del entorno de la forja, los responsables de asignación de tareas encuentran más facilidades para identificar los miembros más adecuados para resolver determinadas necesidades, así como detectar de forma temprana cualquier problema importante que pueda comenzar a surgir dentro del proyecto, desde el punto de vista de recursos necesarios para sus mantenibilidad.
    \item \emph{Aspectos legales y de licencias}: Dentro de la forja, hay cabida para poder indicar de forma clara y explícita toda la información necesaria sobre licencias bajo las que se distribuye los productos del proyecto. Este es un aspecto muy importante para poder garantizar la compatibilidad de los productos del proyecto con otras soluciones desarrolladas en otras iniciativas, evitando la posterior aparición de problemas legales o incompatibilidades en fases de integración más avanzadas.
    \item \emph{Homogeneizar las prácticas y estilo desarrollo}: En proyectos de desarrollo la documentación de la forja y el establecimiento por parte de los usuarios de una serie de directrices que marquen el proceso de desarrollo, las herramientas que se debe utilizar, así como la política que debe seguirse en diferentes estratos de los flujos de trabajo es crucial para mantener un entorno de trabajo efectivo y homogéneo que asegure la cohesión de los diferentes elementos generados para dar como resultado productos mucho más estables, integrados y de mayor calidad.
    \item \emph{Establecimiento de guías de evolución}: La forja es el entorno ideal para poder anunciar y consensuar entre todos los usuarios una guía de evolución clara del proyecto, no solo desde el punto de vista del desarrollo formal de código, sino también de los objetivos generales y de utilidad que se persiguen dentro de la iniciativa para así poder mejorar el mismo proceso de desarrollo.
\end{itemize}

\par El proceso Iterativo e Incremental requiere un seguimiento constante de cada una de las iteraciones a partir de los objetivos establecidos en la Lista de Control. 

\begin{itemize}
	\item En cada iteración se ha de desarrollar la solución asociada a cada requerimiento basando el desarrollo en la orientación a test (TDD) partiendo de una nueva rama de desarrollo.
	\item Acto seguido el resultado se ha de implantar en la rama de desarrollo de la iteración, por lo que se maneja a través del servidor de integración continua para poder adjuntar el desarrollo a la iteración dependiendo si el resultado ha sido positivo o no.
	\item De esta manera se completan los pequeños ciclos de vida que ha de manejar la forja.
\end{itemize}

\par La gestión del ciclo de vida del proceso de desarrollo se divide en el uso de las siguientes herramientas que componen la forja ALM \emph{SidelabCode Stack}:

\begin{itemize}
    \item \emph{Usuarios y Roles} - Gestión de los usuarios, permisos y roles para cada proyecto a través de un directorio de identificación.
	\item \emph{Gesti\'on de Requisitos} - A través de un \emph{Issue tracking system}. Herramienta para gestionar los requisitos y estado actual de cada uno de ellos accesible a los desarrolladores del proyecto.
	\item \emph{Gestión de Repositorios} - Herramienta para la gestión de repositorios; centralizados o distribuidos.
	\item \emph{Ciclo de vida} - Desarrollo y seguimiento de las soluciones mediante el ciclo de vida establecido; requisitos, test, desarrollo, integración continua.
	\item \emph{Gestión de Despliegues} - después de cada Iteración finalizada y publicación del resultado.
\end{itemize}

\par La inclusión de estas herramientas para facilitar el proceso de desarrollo Iterativo e Incremental a través del uso de la forja SidelabCode Stack es lo que permite definir cada comportamiento dentro del proceso mediante la comunicación entre cada una de ellas.

\par Esta definición del proceso de desarrollo se plasma en la guía que proporciona SidelabCode Stack para el desarrollo Iterativo e Incremental.

% section objetivos (end)

\section{Estado del arte de forjas}
\label{sec:estado-del-arte}

\par Hoy en d\'ia las forjas ALM abundan, adem\'as de gozar de una gran popularidad entre los proyectos de Software Libre, como podemos ver en los casos de SourceForge, Googlecode, Bitbucket y Github (más adelante discutiremos cada proyecto). En este caso como \emph{SaaS} (Software as a Service) debido al servicio que ofrecen. El punto diferenciador se encuentra en las forjas ALM las que son \emph{FLOSS}, ya que permiten replicar ese mismo entorno en tu propia máquina o poder trasladar el proyecto de una herramienta a otra. Un dato muy importante a tener en cuenta, porque siempre se ha de mirar hacia adelante. Sobretodo destacan porque si se crea una necesidad con respecto a la gestión, al ser FLOSS siempre se puede implementar con el propio conocimiento del desarrollador a diferencia de las que no son FLOSS. En este caso el usuario \emph{depende de la propia compañía}, un tercero que es el único que puede implementar la solución convirtiendo el servicio en \emph{Vendor-Locking}.

\par En esta lista se muestran las forjas de desarrollo ALM más populares:

\begin{itemize}
	\item SourceForge con Allura - \url{http://sourceforge.net/projects/allura/} - \textbf{FLOSS}.
	\item Cloudbees DEV@Cloud \url{http://www.cloudbees.com/dev.cb} - \textbf{SaaS privado}.
	\item CollabNet con CloudForge \url{http://www.cloudforge.com/} - \textbf{SaaS privado}.
	\item Plan.io - \url{http://plan.io/en/} - \textbf{SaaS privado}.
	\item ClinkerHQ \url{http://clinkerhq.com/} - \textbf{SaaS privado}.
	\item Github - \url{http://github.com/} - \textbf{SaaS privado}.
    \item GitlabHQ - \url{http://gitlab.org/} - \textbf{FLOSS}.
	\item GForge - \url{http://gforge.org/gf/} - \textbf{FLOSS}.
	\item Collab.net - url{http://www.collab.net/} - \textbf{SaaS privado}.
	\item Google Code - \url{http://code.google.com/intl/en/}.
	\item Bitbucket - \url{https://bitbucket.org/} - \textbf{SaaS privado}.
\end{itemize}

% subsection estado-del-arte (end)

\subsection{Casos de Uso}
\label{sub:casos-de-uso}

\par Nos vamos a servir de la investigación hecha por \emph{Cenatic}~\cite{cenatic-forjas} sobre el uso de las forjas y el estudio sobre el incremento de la productividad por \emph{Dirk Riehle}~\cite{open-collaboration-forges} en la implantación de una forja de desarrollo.

\par A partir de estos estudios el caso se muestran los casos de \emph{SAP Forge} basado en GForge y \emph{GoogleCode} con su propia forja. Dos casos de éxito de implantación de una forja desde dos planteamientos diferentes en el uso: internamente y como servicio para proyectos FLOSS (además de internamente).

\subsubsection{SAP}
\label{subsub:sap}

\par En el estudio sobre el aumento de la productividad ~\cite{open-collaboration-forges} nos presenta un detalladamente el caso de utilización de una forja para el lanzamiento y desarrollo de proyectos software dentro de la intranet empresarial. En este caso, se eligió GForge como paquete para proporcionar las funcionalidades básicas de una forja, sobre el que se elaboró un entorno ligeramente más personalizado. De este modo, se presenta la forja de SAP como un entorno de desarrollo colaborativo, centralizado, y fácilmente accesible por cualquier miembro de la empresa que desee participar, gracias a su acceso a través de una URL interna y de fácil memorización. Es importante remarcar que cualquier trabajador dentro de los limites de la red interna puede acceder a la forja.

\par Tras su primer año de andadura, según ~\cite{open-collaboration-forges} SAP Forge había atraído más de 100 proyectos y unos 500 usuarios registrados, lo que aproximadamente representa el 5\% de la población total de desarrolladores \emph{SAP}. La inclusión de un proyecto dentro de la forja no era obligatorio, sino potestad del líder del proyecto.

\par Como datos a destacar en este análisis hay tres características que han de ser mencionadas después de la puesta en marcha del proyecto \emph{SAP Forge}:

\begin{itemize}
	\item Búsqueda de proyectos: Un recurso para los usuarios en el que están indexados todos los proyectos a partir de sus metadatos: nombre, descripción, etc. Todos los proyectos están accesibles para los usuarios (miembros de la empresa) para la consulta. Aplicando el modelo de colaboración abierta del Software Libre, cuantos más ojos interesados, mejores resultados.
	
	\item Información de los desarrolladores: Se genera una base de datos de conocimientos asociados a cada uno de los desarrolladores que son usuarios de la forja. Disponen de una lista de aptitudes obtenida a partir de la información que genera la forja de desarrollo. Esta información facilita la elección de, según los requisitos, poder encontrar a la persona idónea para implementar la solución, cercar las búsquedas y obtener mejores referencias de los trabajadores.
	
	\item Publicidad del proyecto: Se puede hacer un seguimiento de la vida del proyecto a partir de los usuarios, listas de correo, tráfico generado. De esta forma se pueden encontrar los proyectos más usados, útiles y activos, definiendo intereses y reconociendo patrones de casos de éxito aplicables a otros proyectos.
	
\end{itemize}

\par El texto ~\cite{open-collaboration-forges} nos expone el caso de un proyecto (\emph{Mobile Retail Demo}) que se introdujo en la \emph{SAP Forge}. A partir de este movimiento el interés y la colaboración en el proyecto se incrementaron por parte de los desarrolladores. El uso, éxito y la calidad crecieron exponencialmente ligados a la colaboración que había facilitado la forja a los usuarios.

\par Con unas mediciones más exactas basadas en una encuesta interna a los desarrolladores participantes el los proyectos de la forja, se puede ver reflejado el grado de satisfacción por los resultados obtenidos en cuanto a los desarrollo de los proyectos y la colaboración entre ellos:

\begin{quotation}
    \emph{De un total de 83 participantes en la encuesta, un 66\% indicó que habían utilizado las herramientas de búsqueda de proyectos de la \emph{SAP Forge}, para localizar otros proyectos que fuesen de su interés. Un 24\% indicaron que su proyecto había recibido ayuda del exterior (dentro del ámbito interno de SAP), principalmente en forma de reportes de error y sugerencias de mejora. Finalmente, un 12\% de los encuestados comentaron que finalmente colaboraron con otros proyectos diferentes basándose en sus preferencias personales para realizar la selección.}
\end{quotation}

\subsubsection{GoogleCode}
\label{subsub:googlecode}

\par Este es un ejemplo del uso de una forja como servicio externo a la compañía. Google proporciona su propia forja de desarrollo para que los usuarios la utilicen bajo la condición de que los proyectos sean FLOSS asociando un tipo \emph{específico} de Licencia: Apache License 2.0, Artistic License/GPLv2, GNU General Public License 2.0, GNU Lesser Public License, MIT License, Mozilla Public License 1.1, New BSD License.

\par A diferencia de una forja de uso interno, la colaboración entre proyectos se multiplica debido a que el número de usuarios "se dispara" para dar paso a una gran cantidad de distintas situaciones e intercambio de información mediante la colaboración.

\par \emph{GoogleCode} es el proyecto insignia de Google en el ámbito de las forjas de desarrollo de código. Se trata de un proyecto para proporcionar espacio web y herramientas de soporte al desarrollo colaborativo de software, abierto a cualquier grupo o desarrollador individual interesado en publicitar su proyecto.

\par El proyecto se publicó en \emph{Julio de 2006} como plataforma y ha recibido un asombroso número de peticiones de alojamiento. Según los datos publicados por Google en 2009~\cite{google-code-open} se pueden encontrar más de 250.000 proyectos alojados en la forja de \emph{GoogleCode}.

\par \emph{GoogleCode} ofrece distintas herramientas para el desarrollo del ciclo del proyecto:

\begin{itemize}
	\item Gestión de tareas mediante un ITS (Issue Tracking System).
	\item Documentación a través de una Wiki.
	\item Repositorio: Centralizados y Distribuidos - Subversion, Mercurial o Git.
	\item Descarga de binarios.
\end{itemize}

\begin{quote}
    \emph{GoogleCode es en la actualidad uno de los repositorios más consultados a la hora de localizar librerías y aplicaciones ya existentes.}
\end{quote}

% subsection casos-de-uso (end)

\subsection{Tablas comparativas}
\label{sub:comparativa-forjas}

\par A partir de los datos obtenidos de la tabla comparativa sobre forjas de desarrollo en wikipedia~ \cite{comparativa-forjas-wiki} se obtiene una visión más amplia de los componentes comunes en las forjas de desarrollo más conocidas:

%%%%%%%%%%%%%%%%%%%%%%%%%%
% Forge Comparison table %
%%%%%%%%%%%%%%%%%%%%%%%%%%

\begin{landscape}
    \begin{table}[H]
    \centering
    \resizebox{1.5\textwidth}{!} {
        \begin{tabular}{|l|l|l|l|l|l|l|l|l|l|l|l|l|l|}
            \hline {\bf Name} & {\bf Code review} & {\bf Bug tracking} & {\bf Web hosting} & {\bf Wiki} & {\bf Translation system} & {\bf Shell server} & {\bf Mailing List} & {\bf Forum} & {\bf Personal branch} & {\bf Private branch} & {\bf Announce} & {\bf Build system} & {\bf Team}\\

            \hline Alioth (Debian) & no \cellcolor{red} & yes \cellcolor{green} & yes \cellcolor{green} & no \cellcolor{red} & no \cellcolor{red} & yes \cellcolor{green} & yes \cellcolor{green} & yes \cellcolor{green} & yes \cellcolor{green} & yes \cellcolor{green} & yes \cellcolor{green} & no \cellcolor{red} & no \cellcolor{red} \\

            \hline Assembla & yes \cellcolor{green} & yes \cellcolor{green} & yes \cellcolor{green} & yes \cellcolor{green} & yes \cellcolor{green} & no \cellcolor{red} & no \cellcolor{red} & no \cellcolor{red} & yes \cellcolor{green} & yes \cellcolor{green} & yes \cellcolor{green} & yes \cellcolor{green} & yes \cellcolor{green}\\

            \hline BerliOS & ? & yes \cellcolor{green} & yes \cellcolor{green} & yes \cellcolor{green} & ? & yes \cellcolor{green} & yes \cellcolor{green} & yes \cellcolor{green} & ? & ? & yes \cellcolor{green} & ? & ?\\

            \hline Bitbucket & yes \cellcolor{green} & yes \cellcolor{green} & no \cellcolor{red} & yes \cellcolor{green} & no \cellcolor{red} & no \cellcolor{red} & no \cellcolor{red} & no \cellcolor{red} & yes \cellcolor{green} & partial|Yes \cellcolor{yellow} & no \cellcolor{red} & no \cellcolor{red} & yes \cellcolor{green}\\

            \hline CloudForge & ? & yes \cellcolor{green} & no \cellcolor{red} & yes \cellcolor{green} & no \cellcolor{red} & no \cellcolor{red} & no \cellcolor{red} & no \cellcolor{red} & ?    & ? & ? & ? & ?\\

            \hline CodeHaus & no \cellcolor{red} & yes \cellcolor{green} & no \cellcolor{red} & yes \cellcolor{green} & no \cellcolor{red} & no \cellcolor{red} & yes \cellcolor{green} & no \cellcolor{red} & no \cellcolor{red} & no \cellcolor{red} & no \cellcolor{red} & yes \cellcolor{green} & ?\\

            \hline CodePlex & no \cellcolor{red} & yes \cellcolor{green} & no \cellcolor{red} & yes \cellcolor{green} & no \cellcolor{red} & no \cellcolor{red} & yes \cellcolor{green} & yes \cellcolor{green} & no \cellcolor{red} & no \cellcolor{red} & no \cellcolor{red} & no \cellcolor{red} & no \cellcolor{red}\\

            \hline Fedora Hosted & yes \cellcolor{green} & yes \cellcolor{green} & no \cellcolor{red} & yes \cellcolor{green} & no \cellcolor{red} & no \cellcolor{red} & no \cellcolor{red} & no \cellcolor{red} & no \cellcolor{red} & no \cellcolor{red} & no \cellcolor{red} & no \cellcolor{red} & no \cellcolor{red}\\

            \hline GitHub & yes \cellcolor{green} & yes \cellcolor{green} & yes \cellcolor{green} & yes \cellcolor{green} & no \cellcolor{red} & no \cellcolor{red} & no \cellcolor{red} & no \cellcolor{red} & yes \cellcolor{green} & partial|Yes \cellcolor{yellow} & no \cellcolor{red} & no \cellcolor{red} & yes \cellcolor{green}\\

            \hline Gitorious & yes \cellcolor{green} & no \cellcolor{red} & no \cellcolor{red} & yes \cellcolor{green} & no \cellcolor{red} & no \cellcolor{red} & no \cellcolor{red} & no \cellcolor{red} & yes \cellcolor{green} & no \cellcolor{red} & no \cellcolor{red} & no \cellcolor{red} & yes \cellcolor{green}\\

            \hline Gna! & ?& yes \cellcolor{green}& yes \cellcolor{green} & no \cellcolor{red} & yes \cellcolor{green}& ?& yes \cellcolor{green}& no \cellcolor{red}& ?& no \cellcolor{red}& ?& No& ?\\ 

            \hline GNU Savannah & yes \cellcolor{green} & yes \cellcolor{green} & yes \cellcolor{green} & no \cellcolor{red} & no \cellcolor{red} & yes \cellcolor{green} & yes \cellcolor{green} & no \cellcolor{red} & no \cellcolor{red} & no \cellcolor{red} & yes \cellcolor{green} & no \cellcolor{red} & yes \cellcolor{green}\\

            \hline Google Code & yes \cellcolor{green} & yes \cellcolor{green} & partial|Yes \cellcolor{yellow} & yes \cellcolor{green} & no \cellcolor{red} & no \cellcolor{red} & partial|Yes \cellcolor{yellow} & no \cellcolor{red} & partial|Yes \cellcolor{yellow} & no \cellcolor{red} & no \cellcolor{red} & no \cellcolor{red} & no \cellcolor{red}\\

            \hline
        \end{tabular}
    }
    \caption{Comparación de forjas de desarrollo.}
    \label{tabla_compartiva}

    \end{table}
\end{landscape}

\newpage

\par Una lista extra sobre forjas actuales (en las que no se basó el análisis):

\begin{itemize}
	\item Cloudbees DEV@Cloud \url{http://www.cloudbees.com/dev.cb} - \textbf{SaaS privado}.
	\item CollabNet con CloudForge \url{http://www.cloudforge.com/} - \textbf{SaaS privado}.
	\item Plan.io - \url{http://plan.io/en/} - \textbf{SaaS privado}.
	\item ClinkerHQ \url{http://clinkerhq.com/} - \textbf{SaaS privado}.
    \item GitlabHQ - \url{http://gitlab.org/} - \textbf{FLOSS}.
	\item GForge - \url{http://gforge.org/gf/} - \textbf{FLOSS}.
	\item Collab.net - url{http://www.collab.net/} - \textbf{SaaS privado}.
\end{itemize}

\par En la tabla se aprecia que las herramientas que más se utilizan en las forjas de desarrollo son el ITS y la wiki. Mientras que los sistemas de traducción y de construcción pasan al último escalafón. La construcción de la solución está presente en todas las fases del proceso Iterativo e Incremental ya que el desarrollo se basa en las construcciones previas, por lo tanto es más que necesaria una herramienta de construcción integrada en la forja.

% subsection comparativa (end)

\subsection{Problemas en algunas forjas}
\label{sub:problemas}

\par Los requisitos más importante a la hora de desarrollar la Forja SidelabCode Stack para adaptar el proceso de desarrollo Iterativo e Incremental fueron:

\begin{itemize}
	\item La gestión de la información a través del ITS.
    \item Gratuidad en la gestión de los repositorios privados y públicos.
    \item La capacidad de tener un entorno instalado en un servidor propio.
    \item Desarrollar nuevas funcionalidades.
    \item Modularizar componentes.
    \item Interconectar nuevos módulos.
\end{itemize}

\par Partiendo de estos primeros requisitos las Forjas existentes Github, Bitbucket, Collab.net, Cloudbees, Plan.io, ClinkerHQ, Google Code, Fedora Hosted, BerliOS, Gitorius, GNU Savannah, Gna!, CodePlex, CodeHaus, CloudForge, caen eliminadas para una posible solución.

\par En la siguiente iteración de los requisitos se busca la gestión de la documentación, wiki, carpetas públicas y revisión del código, por lo que: Alioth y GitlabHQ se descartan.

\par En el último escalafón tenemos a \emph{GForge}. GForge es un modelo orientado a la gestión del proyecto basando en la forja Savannah. Se ha convertido en una compañía dedicada a la gestión de forjas de desarrollo a través en un modelo SaaS. El proyecto se discontinuó creando un nuevo fork llamado FusionForge en 2009~\cite{bitergia-fusionforge-analysis} . Por lo que en ese año no parecía una apuesta segura para crear una forja de desarrollo. En cuanto a cuestiones técnicas, FusionForge no soportaba repositorios distribuidos como es el caso de Git en sus primeras versiones.

\par Como hemos analizado en la comparativa de la sección~\nameref{sub:comparativa-forjas}, el punto correspondiente al sistema de construcción integrado, no se haya muy extendido en las forjas analizadas. Por lo que es necesario que nos permita adjuntar esta herramienta a la forja de desarrollo para completar el proceso.

% subsection problemas (end)

\section{Conclusiones del estudio de forjas}
\label{sec:conclusiones}

\par Después de haber analizado por encima las soluciones existentes, habiendo encontrado soluciones privadas o con pocos recursos, se reafirma la opción construir una forja de desarrollo que integre el proceso Iterativo e Incremental, desde la recogida de los requerimientos (ITS) hasta la construcción de la solución (sistema de construcción integrado). Crear un sistema modular que permita integrar estas herramientas del ciclo de vida de un proyecto como vehículo en la forja de desarrollo.

% subsection conclusiones (end)
