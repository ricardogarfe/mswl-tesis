%%%%%%%%%%%%%%%%%%%%%%%%%%%%%%%%%%%%%%%%%%%%%%%%%%%%%%%%%%%%%%%%%%%%%%%%%%%%%%%%%%%%%%%%%%%%%%%%%%%%%%%%%%%
%   \copyright 2013 Ricardo García Fernández - ricardogarfe [at] gmail [dot] com.
%
%    This work is licensed under a Creative Commons 3.0 Unported License.
%    To view a copy of this license visit:
% 
%    http://creativecommons.org/licenses/by/3.0/legalcode
%%%%%%%%%%%%%%%%%%%%%%%%%%%%%%%%%%%%%%%%%%%%%%%%%%%%%%%%%%%%%%%%%%%%%%%%%%%%%%%%%%%%%%%%%%%%%%%%%%%%%%%%%%

\chapter{Ep\'ilogo}
\label{chap:epilogo}

\section{Conclusiones}
\label{sec:conclusiones}

\par El uso de estas herramientas y su incremento de la calidad en el desarrollo, por encima de todo siendo FLOSS debido a eso la versatilidad que otorga en el momento de unificarlas en una herramienta nueva; SidelabCode Stack.


\section{Lecciones aprendidas}
\label{sec:lecciones}

\par Colaboraci\'on entre distintos proyectos y comunidades, interoperabilidad entre herramientas, Forjas de desarrollo y los elementos m\'as comunes de las mismas.

\section{Trabajo Futuro}
\label{sec:trabajofuturo}

\par Impulso de la comunidad a trav\'es de los canales habituales.

\par Integraci\'on y gesti\'on de nuevas herramientas comunes para los desarrolladores.

\par Centralizaci\'on de la instalaci\'on.

% section epilogo (end)