%%%%%%%%%%%%%%%%%%%%%%%%%%%%%%%%%%%%%%%%%%%%%%%%%%%%%%%%%%%%%%%%%%%%%%%%%%%%%%%%%%%%%%%%%%%%%%%%%%%%%%%%%%%
%   \copyright 2013 Ricardo García Fernández - ricardogarfe [at] gmail [dot] com.
%
%    This work is licensed under a Creative Commons 3.0 Unported License.
%    To view a copy of this license visit:
% 
%    http://creativecommons.org/licenses/by/3.0/legalcode
%%%%%%%%%%%%%%%%%%%%%%%%%%%%%%%%%%%%%%%%%%%%%%%%%%%%%%%%%%%%%%%%%%%%%%%%%%%%%%%%%%%%%%%%%%%%%%%%%%%%%%%%%%

\chapter{Ep\'ilogo}
\label{chap:epilogo}

\section{Conclusiones}
\label{sec:conclusiones}

\par El uso de estas herramientas y su incremento de la calidad en el desarrollo, por encima de todo siendo FLOSS debido a eso la versatilidad que otorga en el momento de unificarlas en una herramienta nueva; SidelabCode Stack.

\par El proyecto gira en torno a un objetivo conciso:

\begin{quote}
    \emph{'Crear un marco de trabajo para Integrar el uso de la metodología Iterativa e Incremental a través de las herramientas necesarias.'}
\end{quote}

\par A partir de la definición de la metodología Iterativa e Incremental se trasladan los requerimientos de la misma como parte inicial del proyecto para plasmarlos dibujando el camino deseado a seguir.

\par Se han incluido las herramientas necesarias dentro del proyecto SCStack además de su integración dentro del flujo de trabajo y su comunicación entre las mismas dentro del entorno SCStack. Unificando un proceso de desarrollo que aglutina a todos los componentes involucrados mediante la configuración y la rápida disponibilidad de las herramientas para el uso la forja creando un marco de trabajo.

\par Acto seguido se traza el camino a seguir para introducir la metodología en el mismo proceso de desarrollo utilizando las cualidades de cada una de las herramientas. Se localizan y gestionan los recursos necesarios para crear el marco de trabajo Iterativo e Incremental.

\par Documentando el proceso de desarrollo para que a través del marco de trabajo creado se facilita su integración y uso por parte de los actores implicados.

\par Por último se ha comprobado a través de casos reales el correcto funcionamiento de los requerimientos del proceso de desarrollo a través de SCStack.

\par Se ha conseguido integrar de forma satisfactoria el proceso de desarrollo Iterativo e Incremental en el marco de trabajo de SCStack además de dotar a la forja de un potente sistema de inicialización y configuración a través de Puppet, permitiendo futuras incorporaciones, actualizaciones o sustituciones de módulos que pueden aportar mejoras a la herramienta.

\par Se ha cumplido con el objetivo principal del proyecto dotando al mismo de un punto de calidad extra reflejado en la gestión de los módulos o herramientas involucrados a través de la unificación centralizada de la autenticación y la gestión.

\section{Lecciones aprendidas}
\label{sec:lecciones}

\par En este apartado cabría destacar en el inicio que este proyecto no hubiera sido posible sin el uso de soluciones FLOSS ya que se trata de un proyecto de integración y comunicación de diferentes tecnologías por lo que se ha tenido que investigar a través de cada módulo involucrado en SCStack su completo funcionamiento para una integración exitosa.

\par Siguiendo en este apartado a colación de uso de soluciones FLOSS, el desarrollo de este proyecto (en todas sus versiones) ha estado ligado a las comunidades de software de cada una de las tecnologías utilizadas, gran parte del éxito se basa en la comunicación en ambos sentidos (upstream/downstream) con las comunidades.

\par En la parte más técnica el uso de las \emph{diferentes tecnologías} conocidas, en mi caso particular por separado, para después completar el proceso de comunicación entre ellas desde la configuración y la puesta en funcionamiento ha supuesto una nueva perspectiva en lo que a construcción de software se refiere ya que este proceso aporta la visión a una herramienta de comunicación para desarrolladores. Este desarrollo ha supuesto una inclusión en la \emph{virtualización} de recursos y configuración de los mismos a través de \emph{Vagrant y Puppet} respectivamente. La escalabilidad de los componentes y el uso de \emph{APIs}, tanto existentes como creadas para la solución como la apuesta de la ligereza del uso del marco \emph{REST} del uso de distintos lenguajes en un mismo proyecto: \emph{Java, Bash, Ruby o Puppet}.

\par Con respecto a la gestión de proyectos, SCStack me ha permitido entrar en el proceso Iterativo e Incremental, investigar los orígenes y las metodologías ágiles tal como SCRUM para el desarrollo de proyectos en los que se intenta implicar a todas las partes a través de la comunicación, acción bastante complicada.

\par El todo, es decir SCStack como forja de Desarrollo ALM me ha aportado una visión completa mediante el uso de herramientas para facilitar un marco de trabajo al proceso de desarrollo de una solución. La especialización en este campo me ha aportado un conocimiento profesional y capaz de defender ante los más agnósticos del lugar conociendo las virtudes y desventajas del uso de estas herramientas. El poder conocer tanto las debilidades como las ventajas de una solución te permite tener una visión más amplia y poder apreciar nuevas soluciones o aportar el conocimiento a otras. Una capacidad analítica mayor dentro de un campo en el que he estado interesado desde el principio del Máster, intentado buscar el modo de implantar metodologías que mejoren la calidad del software desarrollado. Para conseguir llevar a cabo este requerimiento o idea de excelencia se ha de conocer el software que se está desarrollando:

\begin{quote}
    \emph{'Progress, far from consisting in change, depends on retentiveness. When change is absolute there remains no being to improve and no direction is set for possible improvement: and when experience is not retained, as among savages, infancy is perpetual. Those who cannot remember the past are condemned to repeat it.'} \footnote{George Santayana, The Life of Reason, Volume 1, 1905.}
\end{quote}

\section{Trabajo Futuro}
\label{sec:trabajofuturo}

\par Como parte del proceso de desarrollo de una solución basado en fases, el trabajo futuro es una de las más importantes ya que se basa en el trabajo realizado hasta la fecha sentando las bases para los siguientes hitos habiendo ganado conocimiento en este desarrollo para poder definir nuevas metas objetivas.

\par El trabajo futuro inmediato está relacionado con la 'limpieza del código' de la herramienta y su correspondiente licenciamiento correcto respetando las normas para una solución FLOSS, como es el caso de la descripción de la licencia y los \emph{copyright holders} en las cabeceras de los ficheros. De esta forma SCStack muestra un punto extra de profesionalización del que partir para empezar con la refactorización del código fuente, que sería el siguiente paso. En este paso se incluiría el uso de la herramienta Sonarqube\footnote{Sonarqube - \url{www.sonarqube.org}} que ofrece resúmenes humanamente legibles para representar la calidad del código en números objetivos, teniendo como meta de la inclusión de esta herramienta en futuras versiones de SCStack. Después de su uso y comprensión sea más sencilla su integración.

\par Otro paso importante es el referente a la publicidad de la herramienta y sus casos de éxito, partiendo como ejemplo el uso de la misma en un entorno de producción en la empresa TSCompany. Este apartado otorga fiabilidad y ejemplifica situaciones reales reconocibles por los usuarios.

\par Dentro de la publicidad se ha de hacer hincapié en una gestión del código fuente basada en la rápida y sencilla distribución, lo que implicaría la conversión al uso de Git como SCM del proyecto utilizando como lanzadera la plataforma Github a modo de pantalla de comunicación con los usuarios interesados. Así se crea un canal amigable, sencillo y común en debido a su creciente socialización y aceptación entre los desarrolladores, poniendo una primera piedra para establecer una comunidad alrededor de SCStack.

\par Con respecto a las funcionalidades y herramientas de SCStack, se ha de centralizar la instalación de los módulos, es decir dotar al usuario de una herramienta para que él mismo elija los componentes de SCStack que quiere instalar en su servidor. Creando así una solución completamente modular aislando cada componente en a instalación para después integrar su funcionamiento al ejecutar su instalación. Esta funcionalidad la otorga la herramienta Puppet que ha facilitado la instalación de la forja a través de cada uno de sus módulos independientes y en su medida desacoplados que ha de tender al completo desacoplamiento en futuras versiones para la gestión en una nueva consola de administración que incluya estas funcionalidades.

% section epilogo (end)