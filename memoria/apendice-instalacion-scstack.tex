%%%%%%%%%%%%%%%%%%%%%%%%%%%%%%%%%%%%%%%%%%%%%%%%%%%%%%%%%%%%%%%%%%%%%%%%%%%%%%%%%%%%%%%%%%%%%%%%%%%%%%%%%%%
%   \copyright 2013 Ricardo García Fernández - ricardogarfe [at] gmail [dot] com.
%
%    This work is licensed under a Creative Commons 3.0 Unported License.
%    To view a copy of this license visit:
% 
%    http://creativecommons.org/licenses/by/3.0/legalcode
%%%%%%%%%%%%%%%%%%%%%%%%%%%%%%%%%%%%%%%%%%%%%%%%%%%%%%%%%%%%%%%%%%%%%%%%%%%%%%%%%%%%%%%%%%%%%%%%%%%%%%%%%%

\chapter{Instalación SidelabCode Stack}
\label{app:instalacion-sidelab}

\par En la versión 0.4 de SidelabCode Stack se utiliza Puppet en el proceso de instalación. Ahora es extremadamente sencillo instalar SidelabCode Stack usando Puppet o Vagrant con el nuevo instalador.

\par Puppet nos permite automatizar la instalación en un servidor \emph{Ubuntu} mediante una sencilla configuración.

\par El uso de \emph{Vagrant} permite probar SidelabCode Stack en una máquina virtual en 5 minutos.

\section{Requisitos}
\label{sec:requisitos}

\par Para la instalación de SidelabCode Stack es necesario obtener los siguientes recursos.

\par Dependiendo de la instalación que se utilice \textbar{Vagrant} o \textbar{Puppet} se han de seguir instrucciones diferentes.

\subsection{Descarga SidelabCode Stack}
\label{sub:descarga}

\par Descargar el instalador de SidelabCode Stack (módulo scstack) y sus dependencias de la url:

\begin{itemize}
	\item \url{http://code.sidelab.es/public/sidelabcodestack/artifacts/0.3/puppet-installer-0.3-bin.tar.gz}
\end{itemize}

\lstset{style=rubybasico}
\begin{lstlisting}[frame=trbl]
    cd $HOME
    mkdir tmp
    cd $HOME/tmp
    wget http://code.sidelab.es/public/sidelabcodestack/artifacts/0.3/puppet-installer-0.3-bin.tar.gz
\end{lstlisting}

\par Descomprimir y copiar a la carpeta modules:

\lstset{style=rubybasico}
\begin{lstlisting}[frame=trbl]
    cd $HOME/tmp
    tar xvzf puppet-installer-0.3-bin.tar.gz
\end{lstlisting}

\subsection{Configuración de módulos puppet}
\label{sub:conf-modulos-puppet}

\par Creamos un fichero \texttt{default.pp} en la carpeta \texttt{tmp} con el siguiente contenido:

\lstset{style=rubybasico}
\begin{lstlisting}[frame=trbl]
    exec { "apt-update":
        command => "/usr/bin/apt-get update",
    }
    class { "scstack":
        # Superadmin password. Will be used to access the SidelabCode Stack Console re@lity45
        sadminpass => "re@lity45",
        # Or whatever IP specified in Vagrantfile
        ip => "192.168.33.10", 
        domain => "sidelabcode03.scstack.org",
        baseDN => "dc=sidelabcode03,dc=scstack,dc=org",
        # Your company/organization name
        compname => "SidelabCode Stack version 0.3",
        # A name to be displayed within Redmine
        codename => "SCStack ALM Tools",
    }
\end{lstlisting}

\subsection{Apt Cacher}
\label{sub:apt-cacher}

\par Dependiendo de la conexión de red, el proceso de instalación puede tardar más o menos. En general es buena idea configurar un proxy para los paquetes debian. Esta opción es recomendable en el proceso de instalación a través de Vagrant. Para ello, simplemente hay que instalar \textbf{apt-cacher} en el host:

\lstset{style=rubybasico}
\begin{lstlisting}[frame=trbl]
    sudo apt-get install apt-cacher
\end{lstlisting}

\par Modificar las siguientes líneas del fichero \emph{/etc/apt-cacher/apt-cacher.conf}:

\lstset{style=rubybasico}
\begin{lstlisting}[frame=trbl]
    daemon_addr = 192.168.33.1 # No podemos usar localhost si queremos que los clientes se puedan conectar
    allowed_hosts = * # Permitir a todos los clientes conectarse a este proxy.
    generate_reports = 1 # Generar un informe cada 24h
\end{lstlisting}

\par Reiniciar el servicio:

\lstset{style=rubybasico}
\begin{lstlisting}[frame=trbl]
    sudo /etc/init.d/apt-cacher restart
\end{lstlisting}

\par Modificar el fichero \texttt{/etc/hosts} con la ip del host y el dominio asociado que se define en el fichero de configuración \texttt{default.pp} y la ip y el dominio del cliente para la comunicación con \texttt{apt-cacher} definido en el parámetro \texttt{daemon\_addr} del fichero \texttt{/etc/apt-cacher/apt-cacher.conf}:

\lstset{style=rubybasico}
\begin{lstlisting}[frame=trbl]
    192.168.33.10   sidelabcode03.scstack.org
    192.168.33.1    host.scstack.es
\end{lstlisting}

\par Añadir el siguiente trozo de código en la primera línea del fichero \texttt{default.pp} para que utilice el apt-cacher creado:

\lstset{style=rubybasico}
\begin{lstlisting}[frame=trbl]
    file { "/etc/apt/apt.conf.d/01proxy":
        content => 'Acquire::http::Proxy "http://192.168.33.1:3142/apt-cacher";',
    }
\end{lstlisting}

\section{Vagrant}
\label{sec:vagrant}

\par La instalación a través de Vagrant permite probar SidelabCode Stack en una máquina virtual en 5 minutos.

\subsection{Descripción del entorno de instalación}
\label{sub:entorno-instalacion}

\par Básicamente lo que vamos a hacer es indicar a Vagrant que monte una red privada entre la máquina host y la máquina virtual donde instalaremos SidelabCode Stack. Esto nos permite tener acceso a la forja desde el host. Normalmente, Vagrant asigna, dentro de esa red privada, la IP 192.168.33.1 al host y la IP 192.168.33.10 a la máquina virtual. Estos valores se pueden cambiar como veremos posteriormente.

\subsection{Prerequisitos}
\label{sub:prerequisitos}

\begin{itemize}
    \item Instalar Vagrant y Virtualbox (descargar las últimas versiones de las respectivas páginas web)
    \item Añadir Vagrant al path
\end{itemize}

\lstset{style=rubybasico}
\begin{lstlisting}[frame=trbl]
    $ gedit $HOME/.bashrc
    $ PATH=$PATH:/opt/vagrant/bin
\end{lstlisting}

\subsection{Preparación de la VM}
\label{preparar-vm}

\par Utilizaremos Vagrant para provisionar una VM con Java donde poder instalar SidelabCode Stack. Preparamos la carpeta para el proyecto Vagrant.

\lstset{style=rubybasico}
\begin{lstlisting}[frame=trbl]
    mkdir vagrant
    mkdir vagrant/manifests
    mkdir vagrant/modules
\end{lstlisting}

\par Descargar una imagen vagrant (precise64 es la que elegimos en esta documentación, por ser LTS).

\lstset{style=rubybasico}
\begin{lstlisting}[frame=trbl]
    vagrant box add precise64 http://files.vagrantup.com/precise64.box
\end{lstlisting}

\par Si quiere utilizar una versión distinta de una distribución, puede seleccionarla en la lista de \textbf{boxes} que proporciona \href{http://www.vagrantbox.es/}{vagrant para virtualbox}.

\par Crear un proyecto Vagrant:

\lstset{style=rubybasico}
\begin{lstlisting}[frame=trbl]
    vagrant init
\end{lstlisting}

\par Modificar el Vagrantfile que describe el proyecto (VM, provisioner, etc) para que arranque la vm ``precise64'' y utilice Puppet:

\lstset{style=rubybasico}
\begin{lstlisting}[frame=trbl]
config.vm.box = "precise64"
config.vm.network :hostonly, "192.168.33.10" # Si se desea se puede utilizar el modo bridge y asignar una ip por dhcp dentro de la red en la que se encuentra el host
config.vm.provision :puppet, :module_path => "modules"

\end{lstlisting}

\par Copiar el instalador de la forja en la carpeta modules del proyecto Vagrant:

\lstset{style=rubybasico}
\begin{lstlisting}[frame=trbl]
    cp -R puppet-installer-0.3/* $HOME/vagrant/modules
\end{lstlisting}

\par Copiar el fichero \texttt{default.pp} en la carpeta \texttt{manifests} del proyecto Vagrant:

\lstset{style=rubybasico}
\begin{lstlisting}[frame=trbl]
    cp /tmp/default.pp $HOME/vagrant/manifests
\end{lstlisting}

\par Arrancar la vm.

\lstset{style=rubybasico}
\begin{lstlisting}[frame=trbl]
    cd $HOME/vagrant
    vagrant up
\end{lstlisting}

\section{Puppet}
\label{puppet}

\textbf{TBC}: Descripción

\subsection{Pre requisitos}
\label{sub:puppet-pre-requisitos}

\par Puppet se ha de instalar mediante el gestor de paquetes de la distribución, en este caso \texttt{apt} para Ubuntu:

\lstset{style=rubybasico}
\begin{lstlisting}[frame=trbl]
    $ [sudo] apt-get install puppet
\end{lstlisting}

\par La instalación en otras distribuciones se puede consultar en el \href{http://docs.puppetlabs.com/guides/installation.html}{manual de puppet}.

\subsection{Configuración}
\label{sub:configuracion-puppet}

\par Copiar los módulos puppet al directorio de módulos definido:

\lstset{style=rubybasico}
\begin{lstlisting}[frame=trbl]
    $ mkdir -p $HOME/puppet/modules
    $ cp -R puppet-installer-0.3/* $HOME/puppet/modules
\end{lstlisting}

\par Ejecutar puppet para el proceso de instalación mediante sudo:

\lstset{style=rubybasico}
\begin{lstlisting}[frame=trbl]
    $ sudo puppet apply --modulepath=$HOME/puppet/modules default.pp
\end{lstlisting}

\section{Post instalación}
\label{sec:post-instalacion}

\par El proceso de instalación comenzará automáticamente. Una vez finalizado, en la dirección
\href{http://test.scstack.org/redmine}{http://test.scstack.org/redmine} se mostrará Redmine. La consola es accesible a través de la dirección \href{https://test.scstack.org:5555}{https://test.scstack.org:5555}. Es posible administrar scstack accediendo con el usuario \emph{sadmin} y la contraseña especificada en el parámetro \emph{sadminpass}.

\par \textbf{Nota:} La consola de administración se ha comprobado el funcionamiento para los siguientes navegadores:

\begin{itemize}
	\item \href{http://www.mozilla.org/es-ES/firefox/new/}{Firefox}.
	\item \href{https://www.google.com/intl/es/chrome/browser/?hl=es}{Chrome/Chromium}

\end{itemize}

\subsection{Primeros pasos}
\label{sub:primeros-pasos}

\par Después de la instalación automatizada del entorno se ha de acceder a las herramientas \textbf{Redmine} y \textbf{Archiva} para completar el proceso.

\par Antes de nada se ha de reiniciar la máquina para comprobar que se ejecutan todos los procesos en el inicio.

\subsubsection{Redmine}
\label{subs:conf-redmine}

\par odificar la fecha de creación de la api key de admin desde mysql:

\lstset{style=rubybasico}
\begin{lstlisting}[frame=trbl]
mysql -u root -p
> use redminedb;
> update tokens set created_on='2013-01-16 22:16:00' where id='1';
\end{lstlisting}

\par Modificar los permisos de las carpetas en /opt/redmine/tmp para que sean del usuario de apache:

\begin{itemize}
\item
  \texttt{cd /opt/redmine/tmp \&\& chown -R www-data:www-data *}
\end{itemize}
Acceder a la URL de Redmine, con el usuario \texttt{admin} y el password
\texttt{admin}:

\begin{itemize}
\item \href{https://test.scstack.org/redmine}{https://test.scstack.org/redmine}
\end{itemize}

\par Cambiar la contraseña para que no utilice la genérica a través de la ruta:

\begin{itemize}
\item Administración \textgreater{} Users \textgreater{} admin \textgreater{} Authentication \textgreater{} actualizar.
\end{itemize}

\par Cambiar la API key para securizar el acceso a la API rest (scstack instala una por defecto, pero no es segura):

\begin{itemize}
    \item Acceder como admin \textgreater{} My account \textgreater{} API  access key \textgreater{} Reset \textgreater{} Show -\textgreater{} Copiar la key
    \item Pegar la key en el fichero /opt/scstack-service/scstack.conf
    \item Reiniciar el servicio scstack: \texttt{sudo service scstack-service restart}
\end{itemize}

\subsubsection{Gerrit}
\label{subs:conf-gerrit}

\par El primer usuario que accede a Gerrit obtiene privilegios de administrador. Al instalar la forja, se recomienda crear un usuario ``gerritadmin'' y password ``t0rc0zu310'' y acceder con este usuario a Gerrit. Este usuario se convertirá en administrador automáticamente al hacer login. A partir de este momento, este será el usuario con el que crear los grupos y proyectos (repositorios) en Gerrit.

\par Obtener la clave pública del servidor para asignarla al usuario \textbf{gerritadmin}:

\begin{enumerate}
\item \emph{Settings \textgreater{} SSH Public Keys}\textgreater{} Add.
\item Copiar la clave del fichero \texttt{/opt/ssh-keys/gerritadmin\_rsa.pub}.
\end{enumerate}

\par Configurar permisos para creación de proyectos:

\begin{enumerate}
    \item Acceder a \emph{Projects \textgreater{} List}\textgreater{} All-Projects.
    \item Seleccionar \emph{Access}:
    \item \emph{Editar}:
    \begin{itemize}
	    \item \texttt{Bloque} \textbf{refs/} Add Permission \textgreater{} añadir \emph{Push} el grupo \emph{Administrators}.
	    \item \texttt{Bloque} \textbf{refs/meta/config} Add Group \textgreater{} añadir a \emph{Read} el grupo \emph{Administrators}.
	    \item Save Changes.
    \end{itemize}
\end{enumerate}


\subsubsection{Archiva}
\label{subs:conf-archiva}

\par Acceder a la URL de Archiva:

\begin{itemize}
    \item \url{https://test.scstack.org/archiva}
\end{itemize}

\par La primera vez que se configura archiva pide los datos del administrador. Apuntarlos convenientemente para posteriores necesidades de administración. Se recomienda utilizar la contraseña del administrador de la forja definida en el fichero de configuración \texttt{default.pp}.

\paragraph{Repositorios}

\par Por defecto, Archiva trae configurado un \textbf{repositorio internal} que hace de proxy de \emph{Maven Central} y \emph{java.net}. Si hace falta añadir repositorios remotos adicionales, en Repositories, al final de la página se pueden añadir repositorios remotos.

\par Archiva trae configurado un \textbf{repositorio de snapshots}. Se recomienda crear uno de \textbf{releases}.

\par Para ello accedemos a la administración de Archiva \texttt{Administration -\textgreater{} Repositories} y añadimos uno nuevo con los siguientes parámetros:

\lstset{style=rubybasico}
\begin{lstlisting}[frame=trbl]
Identifier*: *releases*
Name*:       *Archiva Managed Releases Repository*
Directory*:  */opt/tomcat/data/repositories/releases*
...
Repository Purge By Days Older Than: *30*
\end{lstlisting}

\par Se crea el repositorio. Si Tomcat nos muestra un error por pantalla al acceder a la URL \texttt{https://test.scstack.org/archiva/admin/addRepository!commit.action} relacionado con \textbf{NullPointerException} no nos debemos preocupar ya que el repositorio está creado correctamente siendo accesible y funcional. Se puede comprobar volviendo a visualizar la lista de repositorios de \emph{Archiva}.

\paragraph{Usuarios}

\par Archiva no lee los usuarios de \emph{OpenLDAP}, por tanto es necesario añadirlos a mano. En principio, debería ser suficiente con un \textbf{usuario de deploy} para \emph{toda la organización}, o como mucho, un usuario por proyecto o grupo de proyectos.

\par El usuario debe ser \textbf{Observer} de los tres repositorios y \emph{\textbf{manager} de} snapshots y releases.

\section{FAQ}
\label{sec:faq}

\par Posibles problemas que se pueden encontrar tras la instalación:

\begin{itemize}
    \item Error al crear un repositorio Git: Reininciar el servicio scstack-service:
    \lstset{style=rubybasico}
    \begin{lstlisting}[frame=trbl]
        $sudo service scstack-service stop
        $sudo service scstack-service start
    \end{lstlisting}

    \item Error al acceder a \textbf{Archiva} o \textbf{Jenkins} \texttt{404 No encontrado}.
    \begin{itemize}
        \item Configurar el dominio de nombres en el ordenador para que acceda a través de la ip correspondiente.

        \lstset{style=rubybasico}
        \begin{lstlisting}[frame=trbl]
            $ sudo vi /etc/hosts
        \end{lstlisting}

        \item Añadir la línea de conversión entre IP y nombre (elegido en la configuración del fichero default.pp).

        \lstset{style=rubybasico}
        \begin{lstlisting}[frame=trbl]
            138.100.156.246 sidelabcode03.scstack.org
        \end{lstlisting}
     \end{itemize}
\end{itemize}
