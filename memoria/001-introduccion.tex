\chapter{Introducci\'on}
\label{chap:introduccion}

\par \textbf{SidelabCode Stack} es una forja de desarrollo de Software para su uso como herramienta \textbf{ALM}. Es una herramienta FLOSS (Free Libre Open Source Software) con Licencia \textbf{TBC}.

\begin{figure}[h]
    \begin{center}	
        \includegraphics[width=1\textwidth]{sidelab}
        \label{fig:sidelab}
    \end{center}
\end{figure}

\par El desarrollo de este proyecto se basa en el dise\~no e implementaci\'on de la forja SidelabCode Stack ahondando en la definici\'on de un proceso de desarrollo de Software para el dise\~no de la forja, unificando herramientas a trav\'es de la interoperabilidad y facilitando la instalaci\'on, la replicaci\'on y la recuperaci\'on de los datos mediante el uso de Software Libre para construir Software de calidad.

\section{Etimolog\'ia}
\label{sec:etimologia}

\par Sidelab es, un laboratorio de software. Partiendo de esta base, encontramos una definici\'on exacta para SidelabCode \footnote{\url{http://code.sidelab.es/projects/sidelab/wiki/Sidelab}}:

\begin{quotation}
        \emph{Sidelab es el "laboratorio de software y entornos de desarrollo integrados" (Software and Integrated Development Environments Laboratory). Es un grupo de entusiastas de la programaci\'on con inter\'es en pr\'acticamente todos los aspectos del desarollo, desde los lenguajes de programaci\'on y los algoritmos avanzados, hasta la ingenier\'ia del software y la seguridad inform\'atica. Nuestros principales intereses se centran en el desarrollo software y la mejora y personalizaci\'on de los entornos de desarrollo integrados (IDEs) y herramientas relacionadas.}
\end{quotation}

\par En el caso de Code se refiere al c\'odigo en s\'i hacia donde se orienta esta herramienta y Stack, es una pila de servicios para la gesti\'on y el desarrollo de proyectos software.

\par Como resultado tenemos \textbf{SidelabCode Stack} una Forja de desarrollo de aplicaciones.

% subsection etimologia (end)

\section{Trabajo en TSCompany}
\label{sec:trabajo-tscompany}

\par Explicaci\'on del trabajo efectuado en TSCompany durante el desarrollo de la implementaci\'on de la Forja.

% subsection trabajo-tscompany (end)