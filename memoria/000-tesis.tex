\documentclass[a4paper, 12pt]{book}
\usepackage[a4paper, left=2.5cm, right=2.5cm, top=3cm, bottom=3cm]{geometry}
\usepackage[scaled]{helvet}
\renewcommand*\familydefault{\sfdefault} %% Only if the base font of the document is to be sans serif
\usepackage[T1]{fontenc}
\usepackage[utf8]{inputenc}
\usepackage[spanish]{babel}
\usepackage[parfill]{parskip}
\usepackage{graphicx}
\usepackage{tabulary}
\usepackage{color}
\usepackage{colortbl}
\usepackage{verbatim}
\usepackage{float}
\usepackage{hyperref}
\usepackage[nottoc, notlot, notlof, notindex]{tocbibind} %% Opciones de \'indice
\usepackage{latexsym}  %% Logo LaTeX
\usepackage{lscape}
\usepackage{nameref}
\usepackage{appendix}

%%%%%%%%%%%%%%%%%%%%%%%%%%%%%%%%%%%%%%%%%%%%%%%%%%%%%%%%%%%%%%%%%%%%%%%%%%%%%%%%%%%%%%%%%%%%%%%%%%%%%%%%%%
% Paquete escribir código fuente
\usepackage{listings}
\lstloadlanguages{Ruby, XML}

% Se definen los estilo a utilizar para el código fuente
\lstdefinestyle{xmlbasico}{language=XML,commentstyle=\textit,basicstyle=\scriptsize,breaklines=true}
\lstdefinestyle{rubybasico}{language=Ruby,commentstyle=\textit,basicstyle=\scriptsize,breaklines=true}

%\lstdefinestyle{(style name)}{(key=value list)}
% Para usarlos
%\begin{lstlisting}{style=xmlbasico}
%\end{lstlisting}

%Se modifican parámetros con \lstset{(key=value list)}
%%%%%%%%%%%%%%%%%%%%%%%%%%%%%%%%%%%%%%%%%%%%%%%%%%%%%%%%%%%%%%%%%%%%%%%%%%%%%%%%%%%%%%%%%%%%%%%%%%%%%%%%%%

\graphicspath{{img/}}

\renewcommand{\baselinestretch}{1.5}  %% Interlineado\underline{}

\title{\textbf{SidelabCode Stack ALM Tools}}
\author{Autor: Ricardo Garc\'ia Fern\'andez
\\Tutor: }

\renewcommand{\baselinestretch}{1.5}  %% Interlineado

\begin{document}

\renewcommand{\refname}{Bibliograf\'ia}  %% Renombrando
\renewcommand{\appendixname}{Apéndice}
\renewcommand{\appendixtocname}{Apéndices}
\renewcommand{\appendixpagename}{Apéndices}

%%%%%%%%%%%%% PORTADA %%%%%%%%%%%%%%%%

%%%%%%%%%%%%% PORTADA %%%%%%%%%%%%%%%%
\begin{titlepage}
\begin{center}

\begin{figure}[h]
    \begin{center}
        \includegraphics{urjc}
        \label{fig:urjc}
    \end{center}
\end{figure}

\begin{center}
\large
M\'aster Universitario en Software Libre
\\ Proyecto Fin de M\'aster
\\Curso Acad\'emico 2012/2013
\end{center}

\vspace{2cm}

\begin{center}

\LARGE
Proceso de Desarrollo Iterativo e Incremental a través de SidelabCode Stack

\large
Autor: Ricardo Garc\'ia Fern\'andez
\\Tutor:
\end{center}

\vfill

\begin{flushright}
\small
    \copyright 2013 Ricardo Garc\'ia Fern\'andez - ricardogarfe [at] gmail [dot] com.

    This work is licensed under a Creative Commons 3.0 Unported License.
    To view a copy of this license visit:
 
    \url{http://creativecommons.org/licenses/by/3.0/legalcode}.
\end{flushright}

\begin{figure}[h]
    \begin{flushright}	
        \includegraphics{by}
        \label{fig:by}
    \end{flushright}
\end{figure}

\end{center}
\end{titlepage}

\newpage

%%%%%%%%%%%%%%%%%%%%%%%%%%%%%%%%%%%%%%

%%%%%%%%%%%%% indices %%%%%%%%%%%%%%%%

\tableofcontents  %% Creando \'indice

\newpage

\listoffigures  %% \'indice de figuras

\newpage

\listoftables %% \'indide de tablas

\newpage

%%%%%%%%%%%%%%%%%%%%%%%%%%%%%%%%%%%%%%

\chapter{Introducci\'on}
\label{chap:introduccion}

\par \textbf{SidelabCode Stack} es una forja de desarrollo de Software para su uso como herramienta \textbf{ALM}. Es una herramienta FLOSS (Free Libre Open Source Software) con Licencia \textbf{TBC}.

\begin{figure}[h]
    \begin{center}	
        \includegraphics[width=1\textwidth]{sidelab}
        \label{fig:sidelab}
    \end{center}
\end{figure}

\par El desarrollo de este proyecto se basa en el dise\~no e implementaci\'on del proceso de desarrollo acorde con las metodolog\'ias \'agiles a trav\'es de la forja \emph{SidelabCode Stack}. Unificando herramientas a trav\'es de las distintas APIs basadas en la interoperabilidad, facilitando la instalaci\'on, la replicaci\'on y la recuperaci\'on de los datos mediante el uso de Software Libre para construir Software de calidad.

\par El uso de metodolog\'ias \'agiles se encuentra intr\'insecamente relacionado a lo largo del proceso de desarrollo e implementaci\'on de la forja. Por otra parte podemos afirmar que no es una herramienta intrusiva para la aplicaci\'on de las distintas metodolog\'ias ligadas a un proceso de desarrollo.

\par Presentando las distintas metodolog\'ias de desarrollo de software definidas analizando sus caracter\'isticas tanto las buenas como las malas y como \'estas se aplican al dise\~no de la herramienta encaminado un proceso de desarrollo fluido y \'agil para un desarrollador o un grupo de desarrolladores.

\par Un punto a tener en cuenta es la ergonom\'ia con la que cuenta la herramienta para el d\'ia a d\'ia y las soluciones que aporta con respecto a otras herramientas que cohabitan en el mismo campo.

\par Todo con un mismo fin; producir software de calidad evaluable, es decir, no es necesario crear un producto perfecto sino se sabe mejorarlo y adaptarlo a nuevas necesidades. Para eso utilizamos las herramientas ALM en los proyectos ya que nos ayudan a tener una visi\'on diaria y a posteriori con perspectiva para poder medir y evaluar las mejoras entre dos puntos de tiempo. Por lo tanto \emph{la evoluci\'on del proyecto y la adaptabilidad a los cambios}.

\section{Etimolog\'ia}
\label{sec:etimologia}

\par Sidelab es, un laboratorio de software. Partiendo de esta base, encontramos una definici\'on exacta para SidelabCode \footnote{\url{http://code.sidelab.es/projects/sidelab/wiki/Sidelab}}:

\begin{quotation}
        \emph{Sidelab es el "laboratorio de software y entornos de desarrollo integrados" (Software and Integrated Development Environments Laboratory). Es un grupo de entusiastas de la programaci\'on con inter\'es en pr\'acticamente todos los aspectos del desarollo, desde los lenguajes de programaci\'on y los algoritmos avanzados, hasta la ingenier\'ia del software y la seguridad inform\'atica. Nuestros principales intereses se centran en el desarrollo software y la mejora y personalizaci\'on de los entornos de desarrollo integrados (IDEs) y herramientas relacionadas.}
\end{quotation}

\par En el otro extremo del nombre se encuentra \emph{Code} se refiere al c\'odigo en s\'i hacia donde se orienta esta herramienta. Por ultimo pero no menos importante tenemos \emph{Stack}, es una pila de servicios para la gesti\'on y el desarrollo de proyectos software.

\par Como resultado tenemos \textbf{SidelabCode Stack} una Forja de desarrollo de aplicaciones orientada a un proceso de desarrollo.

% subsection etimologia (end)

\section{Trabajo en TSCompany}
\label{sec:trabajo-tscompany}

\par TSCompany (trabajo desempeñado a grandes rasgos) -> aquí se puede justificar muy bien el trabajo por la necesidad de implantar en un entorno real una forja funcional.

\par Explicaci\'on del trabajo efectuado en TSCompany durante el desarrollo de la implementaci\'on de la Forja.

\par El pr\'acticum del M\'aster me dio la oportunidad de entrar a trabajar en la empresa TSCompany para cubrir la plaza de becario. Desde el principio he estado interesado en la producción de software de calidad, en las herramientas e control de versiones, el desarrollo orientado a tests, la documentación y de como unificar las herramientas para obtener un rendimiento óptimo a la hora de desarrollar un proyecto, es decir, un desarrollador que desarrollar al 100\%.

\par Esta colaboraci\'on se bas\'o en la aportaci\'on al proyecto de \emph{Software Libre} SidelabCode Stack.

\par Profesionalmente siempre he tenido ejemplos cercanos acerca del desarrollo de software de calidad y me ha han inculcado el desarrollo orientado a tests \emph{TDD}. Siempre he perseguido la simplicidad y de esta forma la replicaci\'on de un entorno, pruebas, instalaciones y la estandarizaci\'on de la programaci\'on para una comprensi\'on sencilla.

\par El objetivo inicial era pulir el funcionamiento de la herramienta SidelabCode Stack unificando el proceso de desarrollo en la instalaci\'on ya que hab\'ian m\'odulos que todav\'ia no estaban controlados, es decir, centralizados y quedaban dispersos para el usuario final. Por ejemplo, el usuario ten\'ia que ejecutar una serie pasos para poder incluir un proyecto en un repositorio distribuido Git.

\par Hab\'ia que dar forma al modelo de desarrollo y promocionarlo dentro de la estructura de la forja para que resultase f\'acil para el usuario empezar a desarrollar su proyecto dentro del entorno SidelabCode.

\par Despu\'es de la formalizaci\'on de las distintas versiones del proyecto, se invirti\'o un tiempo para la implantaci\'on de la forja en la empresa TSCompany para su uso diario. Partiendo de la migraci\'on de la informaci\'on de todos los proyectos y usuarios de la empresa para as\'i poder gestionarlos a trav\'es de un interfaz com\'un, SidelabCode Stack. Debido al uso de herramientas de Software Libre la migraci\'on fue fluida y nada agresiva para los usuarios. El entorno de trabajo segu\'ia siendo el mismo pero en este caso aportando un extra de calidad y m\'etricas para la evaluaci\'on de los proyectos desarrollados.

% subsection trabajo-tscompany (end)

%%%%%%%%%%%%%%%%%%%%%%%%%%%%%%%%%%%%%%

%%%%%%%%%%%%%%%%%%%%%%%%%%%%%%%%%%%%%%%%%%%%%%%%%%%%%%%%%%%%%%%%%%%%%%%%%%%%%%%%%%%%%%%%%%%%%%%%%%%%%%%%%%%
%   \copyright 2013 Ricardo García Fernández - ricardogarfe [at] gmail [dot] com.
%
%    This work is licensed under a Creative Commons 3.0 Unported License.
%    To view a copy of this license visit:
% 
%    http://creativecommons.org/licenses/by/3.0/legalcode
%%%%%%%%%%%%%%%%%%%%%%%%%%%%%%%%%%%%%%%%%%%%%%%%%%%%%%%%%%%%%%%%%%%%%%%%%%%%%%%%%%%%%%%%%%%%%%%%%%%%%%%%%%

\chapter{Motivaci\'on}
\label{chap:motivacion}

\par En el mismo punto, damos la bienvenida a \emph{SidelabCode Stack}, la herramienta ALM para el proceso de desarrollo.

\par Empezaremos ubicando la forja de desarrollo SidelabCode Stack. ¿ Cual es la motivaci\'on que nos lleva a implementar un nuevo ALM ?

\par La necesidad de tener un esqueleto para el desarrollo de un proyecto a trav\'es del uso de herramientas de Software Libre.

\par Partiendo del principio DRY (\emph{Don't Repeat Yourself}) y de la mano de \emph{No reinventar la rueda} el proyecto surgi\'o con la necesidad de crear una forja de desarrollo Libre y usable en distintos entornos.

\par Tras analizar las soluciones ALM existentes se opt\'o por crear una nueva Forja a partir de las herramientas necesarias para el proceso de desarrollo de Software de Calidad para adaptar las metodolog\'ias \'agiles existentes.

\par Para facilitar y unificar el proceso de desarrollo se optó por crear una herramienta que aglutinase la gestión de las tareas, el código fuente y la orientación del desarrollo a los tests.

\par La integraci\'on del proceso de desarrollo en la implementaci\'on de una Forja ALM es el punto clave de SidelabCode Stack. Para crear un entorno adecuado se tuvieron en cuenta distintas características en el proceso de creación:

\begin{itemize}
	\item Ser efectiva: la modificaci\'on de una capacidad espec\'ifica no requerir\'a el cambio de la plataforma completa, ni un cambio en el despliegue de la soluci\'on.
	\item Ser mantenible y que sus funcionalidades sean f\'acilmente extensibles y reutilizables.
	\item Permitir la portabilidad de los proyectos entre distintas forjas facilitando la migración.
\end{itemize}

%%%%%%%%%%%%%%%%%%%%%%%%%%%%%%%%%%%%%%

\chapter{Objetivos}
\label{chap:objetivos}

\par Constancia, metodolog\'ia y facilidad de uso basado en herramientas al alcance de cualquier desarrollador a trav\'es de un proceso de desarrollo para crear software de calidad. Estos son los principios en los que se basa la implementación de SidelabCode Stack teniendo como objetivos tangibles la unificación del uso de las herramientas necesarias para cumplir con éstos.

\par Convertir estos objetivos en los siguientes requerimientos para integrar el desarrollo de un proyecto.

\section{Proyecto, usuarios, permisos y roles }
\label{sec:proyecto-usuarios}

\par La gestión del proyecto a partir de la definición de los usuarios, permisos y roles definidos para el mismo. Centralizar esta gestión para así obtener un control fluido y centralizado de los \emph{"poderes"} otorgados según el rango del usuario a través de una herramienta.

% section proyecto-usuarios (end)

\section{Carpetas compartidas}
\label{sec:carpetas-compartidas}

\par Generar un espacio de carpetas p\'ublicas/privadas para cada uno de los proyectos a partir de los permisos de cada usuario. Asegurando el acceso público/privado a los recursos que se publiquen.

% section carpetas-compartidas (end)

\section{ITS}
\label{sec:its}

\par Empezando por el An\'alisis de Requerimientos en un proyecto es necesaria una herramienta que ayude en el proceso de desarrollo, en esta caso un \emph{ITS Issue Tracking System} para la gestión y el mantenimiento de las tareas de cada proyecto. La piedra angular de todo proyecto de software. El ITS otorga un control total del flujo de trabajo y responsabilidades a los usuarios del proyecto, una base de documentación y gestión del tiempo prioritaria en todo proyecto de software.

\par Utilizar el ITS para gestionar las tareas y los posibles errores dentro de cada proyecto para planificar la corrección o la implementación de éstas, dividiendo las entregas del proyecto en versiones durante el proceso continuo de desarrollo.

% section its (end)

\section{Repositorio de código}
\label{sec:repositorio-codigo}

\par El Repositorio de c\'odigo asociado a cada proyecto. Sin repositorio de código no hay proyecto. De esta forma la cooperación entre usuarios de un proyecto se agiliza bajo el manto de un sistema de control de versiones de código fuente. Esta herramienta produce la información necesaria para generar un histórico de cambios y poder evaluar la evolución del proyecto.

\par Además se ha de dotar de una herramienta para la Gesti\'on del repositorio. Es decir, la gestión de parches, actualizaciones y desarrollo por ramas dentro de un entorno controlado.

% section repositorio-codigo (end)

\section{Gestión de librerías}
\label{sec:gestion-librerias}

\par La gesti\'on de librer\'ias centralizada para los proyectos dentro de la forja, ofrece la posibilidad de interconectar distintas librerías dentro de todo el entorno. Publicando las librerías clasificadas a través de las distintas versiones en un repositorio de librerías común. La reutilización del código y la posibilidad de mejora entre distintos proyectos que compartan librerías mejora la calidad del mismo.

% section gestion-librerias (end)

\section{TDD Test Driven Development}
\label{sec:tdd}

\par El desarrollo dirigido por Tests \emph{TDD} ayuda a la generación de código de calidad, legible y reutilizable. Siguiendo este proceso, el tiempo dedicado a la replicación de errores disminuye por lo que se puede dedicar más tiempo a la corrección del error en si.

\par Otorga datos válidos para el análisis del código y controlar la evolución de la cobertura de Tests para poder evaluar la deuda técnica. Genera los datos suficientes para facilitar la corrección de la misma.

% section tdd (end)

\section{CI - Integración Continua}
\label{sec:integracion-continua}

\par El TDD, el desarrollo por ramas y los despliegues en diferentes entornos completan un proceso de desarrollo que culmina en la integración continua del proyecto. Este es el control al más alto nivel dentro de la parte técnica en donde el control de la calidad del código se toma como estandarte. Prevé los errores previos entre distintas versiones del código al unificar las ramas, fallos en los tests y validación de la calidad del código.

\par Crear despliegues en distintos entornos o replicar entornos similares al entorno final desde esta herramienta para minimizar los errores o poder replicar fácilmente los mismos cuando surjan.

% section integracion-continua (end)

\section{Interoperabilidad}
\label{sec:interoperabilidad}

\par La interoperabilidad entre todas las herramientas para gestionar la configuración a través de una herramientas centralizada. Este apartado es donde se muestra el potencial de la forja de desarrollo ya que la configuración se expande o replica a las demás herramientas dando vida a la comunicación entre ellas para crear la forja.

\par Se transforma la configuración inicial y se establecen los protocolos de comunicación entre cada una de las herramientas dentro del proceso de desarrollo habiendo definido los pasos a seguir a partir de las \emph{API}s existentes.

% section interoperabilidad (end)

%%%%%%%%%%%%%%%%%%%%%%%%%%%%%%%%%%%%%%

%%%%%%%%%%%%%%%%%%%%%%%%%%%%%%%%%%%%%%%%%%%%%%%%%%%%%%%%%%%%%%%%%%%%%%%%%%%%%%%%%%%%%%%%%%%%%%%%%%%%%%%%%%%
%   \copyright 2013 Ricardo García Fernández - ricardogarfe [at] gmail [dot] com.
%
%    This work is licensed under a Creative Commons 3.0 Unported License.
%    To view a copy of this license visit:
% 
%    http://creativecommons.org/licenses/by/3.0/legalcode
%%%%%%%%%%%%%%%%%%%%%%%%%%%%%%%%%%%%%%%%%%%%%%%%%%%%%%%%%%%%%%%%%%%%%%%%%%%%%%%%%%%%%%%%%%%%%%%%%%%%%%%%%%

\chapter{Procesos de desarrollo}
\label{chap:procesos-desarrollo}

\par Un proceso de desarrollo en el mundo del software se define como:

\begin{quote}
\emph{El proceso de transformación de unos requerimientos en una solución. Un conjunto de pautas a seguir para completar el ciclo de vida de la solución de una manera organizada, sistemática y que ayude a las personas a completar los objetivos fijados}
\end{quote}

\par En SidelabCode Stack se optó por implementar el proceso de desarrollo de software \emph{iterativo e incremental} para crear el dise\~no de la forja SidelabCode Stack a través de metodolog\'ias ágiles.

\par El proceso de desarrollo infiere directamente en la calidad del software que se construye. Es la parte más importante en el software ya que un proceso de desarrollo óptimo para una solución otorga las herramientas necesarias para una mejor evolución del mismo. Cada proceso de desarrollo ha de aplicarse a la solución según los requisitos de la misma, no todos los procesos de desarrollo son válidos para todos los proyectos.

\section{Software de Calidad}
\label{sec:software-calidad}

\par Desarrollar Software de Calidad. En este paso nos encontramos con la palabra \emph{'Calidad'}, tan subjetiva en muchos ámbitos pero que en el desarrollo puede ser bastante objetiva ya que se trata de Software, una ciencia evaluable. La Calidad del Software es el conjunto de cualidades que lo caracterizan y determinan su viabilidad y utilidad; Mantenibilidad, Fiabilidad, Eficiencia y Seguridad.

\begin{figure}[H]
    \begin{center}	
        \includegraphics[width=0.9\textwidth]{SoftwareQuality}
        \caption{Software de Calidad}
        \label{fig:softwarequality}
    \end{center}
\end{figure}

\par Un software hecho para ejecutarse una sola vez no requiere el mismo nivel de calidad mientras que un software para ser explotado durante un largo necesita ser fiable, seguro, mantenible y flexible para disminuir los costes.

\begin{itemize}
	\item \emph{Mantenibilidad}: El software debe ser diseñado de tal manera, que permita ajustarlo a los cambios en los requerimientos. Esta característica es crucial, debido al inevitable cambio del contexto en el que se desempeña un software.
	\item \emph{Fiabilidad}: Incluye varias características además de la fiabilidad, como la aplicación de estándares, complejidad, tratamiento de errores.
	\item \emph{Eficiencia}: Tiene que ver con el uso eficiente de los recursos que necesita un sistema para su funcionamiento.
	\item \emph{Seguridad}: La evaluación de la seguridad requiere un control sobre la arquitectura, el diseño y las buenas prácticas.
\end{itemize}

% section software-calidad (end)

\section{Proceso iterativo}
\label{sec:proc-iterativo}

\par ¿ Que es el proceso iterativo ?

\begin{quote}
    \emph{La primera versión debe contener todos los requerimientos del usuario y lo que se va a hacer en las siguientes versiones es ir mejorando aspectos como la funcionalidad o el tiempo de respuesta.}\footnote{Procesos Iterativos e Incrementales - \url{http://esalas334.blogspot.es/1193761920/}}
\end{quote}

\par Se centra más en la inmediatez de la primera versión y en las mejoras posteriores que se van creando enfocadas a la solución final. En el proceso también juega una parte fundamental la comunicación con el cliente a través de la visualización de los resultados por iteraciones. De esta forma se consigue una buena coordinación entre el cliente y el equipo de desarrollo para la consecución de los objetivos.

\begin{figure}[H]
    \centering
    \includegraphics[width=0.9\textwidth]{DevelopmentIterative}
    \caption{Desarrollo Iterativo}
    \label{fig:desarrollo-iterativo}
\end{figure}

\par Se han de tener en cuenta posibles cambios entre iteraciones pero nunca del resultado completo, para que de esta forma se pueda controlar a tiempo la \emph{desviación} que pueda existir en el proceso de la creación de producto.

\begin{quote}
    \emph{Como la idea que representa la palabra iterativo, un proceso de desarrollo de software iterativo es aquel al que se lo piensa, como una serie de tareas agrupadas en pequeñas etapas repetitivas. Estas "pequeñas etapas repetitivas" son las iteraciones.}\footnote{Proceso de Desarrollo Iterativo - Fernando Soriano - \url{http://fernandosoriano.com.ar/?p=13}}
\end{quote}

\par La base el proceso de desarrollo Iterativo provee un conjunto de pasos para el desarrollo de la solución que se repiten iteración tras iteración para la creación de mejoras tangibles y/o evaluables. 

\begin{figure}[H]
    \centering
    \includegraphics[width=0.9\textwidth]{ProcesoIterativo}
    \caption{Proceso Iterativo}
    \label{fig:ProcesoIterativo}
\end{figure}

\par En cada iteración se construye una pieza funcional del producto final, completa, testeada, documentada e integrada en la solución final. La visión completa de este proceso muestra una línea de iteraciones separadas funcionalmente unas de otras que en conjunto, forman la solución final. Iteraciones independientes unas de otras a través de un desarrollo lineal agrupando pequeños ciclos de desarrollo.

\begin{itemize}
	\item \emph{Duración fija}, quiere decir que una vez establecidos los tiempos o planificación de la iteración, la iteración termina en la fecha exacta establecida. Si el equipo no pudo cumplir lo planificado, el desarrollo pendiente pasa a otra iteración.
	\item Estimación de tiempos cortos, las \emph{"buenas prácticas"} hablan de que una iteración debiera durar entre 2 y 6 semanas.
	\item Es como un ciclo de desarrollo completo, ya que en una iteración se realizan actividades de análisis, diseño, implementación, pruebas, etc.
\end{itemize}

% section proc-iterativo (end)

\section{Proceso incremental}
\label{sec:proc-incremental}

\par El Proceso Incremental fue propuesto por \emph{Harlan D. Mills} en 1980.

\par Sugirió el enfoque incremental de desarrollo como una forma de reducir la repetición del trabajo en el proceso de desarrollo y dar oportunidad de retrasar la toma de decisiones en los requisitos hasta adquirir experiencia con el sistema.

\par El modelo incremental combina elementos del modelo lineal secuencial (aplicados repetidamente) con la filosofía interactiva de construcción de prototipos. El modelo incremental aplica secuencias lineales de forma escalonada mientras progresa el tiempo en el calendario.

\par Cada secuencia lineal produce un \emph{incremento} en el desarrollo de la solución. Por ejemplo, en relación a la forja SidelabCode Stack; en la primera versión estaba accesible el módulo de Jenkins, en el siguiente incremento la configuración de Jenkins se ligaba automáticamente a la configuración del los usuarios por proyecto, el siguiente incremento se publicaban las instrucciones para gestionar Jenkins a partir de una cuenta y facilitar la configuración para los distintos entornos.

\begin{figure}[H]
    \centering
    \includegraphics[width=0.9\textwidth]{modelo_incremental}
    \caption{Modelo Incremental}
    \label{fig:modelo-incremental}
\end{figure}

\par Al iniciar el desarrollo, los clientes o los usuarios, identifican a grandes rasgos las funcionalidades que proporcionará el sistema. Se define un bosquejo de requisitos funcionales y será el cliente quien se encarga de priorizar que funcionalidades son más importantes. Con las prioridades definidas, se puede confeccionar el plan de incrementos, en donde cada incremento se compone de un subconjunto de funcionalidades a desarrollar.

% section proc-incremental (end)

\section{Iterativo e Incremental}
\label{sec:iterativo-incremental}

\par Desarrollo iterativo e incremental. La conjunción de estos dos tipos de desarrollo aúnan las mejores cualidades de ambos para gestión de un equipo de trabajo en la construcción de una solución.

\par El proceso iterativo e incremental se basa en incrementos por cada una de las iteraciones en el proceso de desarrollo. La idea básica de este proceso es desarrollar una solución a través de las iteraciones de ciclos a partir de los incrementos en la funcionalidad para que los desarrolladores mejoren su productividad en torno al proyecto a partir de pequeños hitos que completan versiones usables de la solución desde la primera implementación.

\begin{figure}[H]
    \centering
    \includegraphics[width=0.9\textwidth]{iterativo-incremental-larman}
    \caption{Proceso iterativo e Incremental por Larman}
    \label{fig:iterativo-incremental-larman}
\end{figure}

\par La evolución de la solución se basa en las iteraciones pasadas añadiendo los nuevos requisitos/objetivos (pueden ser mejoras o nuevas funcionalidades). Se basa en incrementar el valor del trabajo hecho para tener un control del proceso más exhaustivo priorizando los objetivos. Por cada iteración existen modificaciones en el diseño y en las funcionalidades.

\par La comunicación y la implicación en el desarrollo del proyecto con el usuario/cliente desde el inicio del proyecto es crucial ya que el proceso parte desde una solución inicial para incluir versiones usables con el usuario/cliente. De esta forma todas las partes aportan sus distintos puntos de vista de una manera continua implicándose en el proceso y midiendo el crecimiento de la solución paso a paso, incremento a incremento.

\par Este proceso de desarrollo cercano a las \emph{Metodologías Ágiles}, como ellas tiene el mismo fin, la implicación, desarrollo, fiabilidad, confianza, aprendizaje, versatilidad, responsabilidad, comunicación con la solución desarrollada y el usuario/cliente durante el proceso.

\par Una de las claves en este proceso es la retroalimentación y el aprendizaje del grupo de trabajo a partir de las iteraciones. De esta forma el trabajo hecho repercute en las iteraciones futuras aportando nuevos conocimientos sobre el proceso y el desarrollo. Creando mejores iteraciones y evoluciones de la solución, \emph{profesionalizando el trabajo}.

\par Las Iteraciones han de ser de una duración corta de \emph{2 a 6 semanas} para que la comunicación a todos los niveles del proyecto siga siendo fluida y para que si en algún caso se haya de desechar una Iteración no se pierda mucho trabajo desarrollado en ella. Esto no ocurre muy a menudo pero se contempla por diferentes causas:

\begin{itemize}
	\item Abandono del proyecto.
	\item Cambio de Cliente/Usuario.
	\item Falta de recursos.
\end{itemize}

\par Las Iteraciones \emph{cortas} otorgan al modelo un alto nivel de versatilidad a la hora de evolucionar y evaluar el recorrido del trabajo hecho en el proyecto, para así poder predecir la organización de las futuras iteraciones.

\subsection{Fases del proceso}
\label{sub:fases-proceso}

\par El proceso de desarrollo Iterativo e Incremental está basado en tres fases:

\begin{itemize}
	\item Iniciación.
	\item Iteración.
	\item Lista del Control del Proyecto.
\end{itemize}

\par El objetivo de la \emph{Iniciación} es la implementación inicial para crear un producto con el cual el usuario/cliente pueda interactuar y tener las primeras impresiones. El equipo de desarrollo y el usuario/cliente toman como punto base esta fase de \emph{Iniciación}.

\par Esta primera implementación ha de servir de guía para la evolución del desarrollo en cada una de las iteraciones, la base.

\par La \emph{Lista de Control del Proyecto} es el lugar donde se definen las tareas que han de cumplimentarse durante el proceso de desarrollo. Todos los aspectos relacionados con la implementación de la solución se encuentran definidos en la lista, funcionalidades, diseño, errores, mejoras. La Lista de Control está en constante evolución, no es un muro estático, ya que se van adjuntando las funcionalidades y/o mejoras y posibles nuevas funcionalidades. De esta forma se evalúan las prioridades en la fase de análisis por cada iteración y se decide que tareas han de implementarse y cuales son desechadas para la siguiente iteración.

\par La \emph{Iteración} es un conjunto modular de acciones a llevar a cabo para cumplir con las tareas que se definen para evolucionar la solución por incrementos. Debe estar sujeta a cambios en el diseño, nuevas tareas añadidas a la lista de control y sobretodo, ser simple.

% subsection fases-proceso (end)

\subsection{Desarrollo Iteración}
\label{sub:desarrollo-iteracion}

\par El desarrollo del proyecto viene medido por las Iteraciones que se conectan una a otra secuencialmente.

\par La iteración comienza a partir del análisis basado en la retroalimentación de los usuarios y los servicios de análisis disponibles. Los elementos a tener en cuenta en el análisis son:

\begin{itemize}
	\item Estructura.
	\item Modularidad.
	\item Ergonomía.
	\item Eficiencia.
	\item Objetivos logrados.
\end{itemize}

\par Se han tener en cuenta los posibles riesgos que puedan surgir en la Iteración para evaluarlos y eliminarlos en el momento de definir la línea base de la arquitectura. Además el equipo de desarrollo ha de dominar y estar al tanto del lenguaje empleado en los requisitos, el problema que se va a abordar y de esta manera ser capaces de asumir los posibles riesgos o imprevistos que puedan surgir.

\par Los resultados del análisis se reflejan en la Lista de Control del proyecto para añadir, modificar y ordenar por prioridades para la siguiente Iteración.

\par En las Iteraciones posteriores ha de aumentar la capacidad de reducir los riesgos, desarrollar los componentes e ir evolucionando incremento a incremento hacia la versión final para el usuario/cliente.

\par Cada Iteración se reduce a un \emph{miniproyecto} (Se les llama miniproyectos porque no es algo que el usuario haya pedido) que consta del proceso de requisitos, análisis, diseño, implementación y prueba.

\begin{figure}[H]
    \centering
    \includegraphics[width=0.9\textwidth]{valor-negocio-iterativo-incremental}
    \caption{Valor de negocio Iterativo e Incremental}
    \label{fig:valor-negocio-iterativo-incremental}
\end{figure}

\par El proceso de desarrollo en una Iteración se reduce a una serie de guías o pasos a seguir en torno a las modificaciones que surgen a la medida que se avanza en el desarrollo. El proceso Iterativo e Incremental se basa en la flexibilidad por lo tanto las modificaciones de la Iteración han de ser otra herramienta más para lograr los objetivos y como tales:

\begin{itemize}
	\item Cualquier dificultad encontrada en el diseño, desarrollo y prueba de una modificación puede alertar de la necesidad de cambiar el diseño o la implementación. Se han de desarrollar las modificaciones con una estructura modular y aislada (en su medida) para poder trabajar en un problema o solución concreta y que no afecte al resto de la implementación.
	\item Las modificaciones han de ser sencillas de implementar, sino se ha de rediseñar el la solución.
	\item Las modificaciones han de ser más sencillas conforme se van completando iteraciones. Si esto no ocurre existe un problema de diseño y puede incurrir en el exceso de soluciones \emph{ad-hoc}, parches.
	\item Los parches se contemplan como soluciones temporales en distintas iteraciones para que no sea necesario un cambio en el diseño, pero sólo para casos excepcionales. Si los parches proliferan se ha de replantear el diseño.
	\item La implementación existente ha de ser analizada constantemente para certificar que sigue el camino marcado por los objetivos a corto y largo plazo del proyecto. De esta forma se controlan las posibles desviaciones de tiempo y el trabajo hecho por el grupo de trabajo.
	\item Los herramientas de análisis se han de utilizar para validad los análisis y/o funcionalidades de las implementaciones parciales de la solución.
	\item La participación del usuario/cliente en el proceso ha de ser solicitada y analizada para contemplar posibles deficiencias o errores en la implementación actual. De esta forma es como se ha de crear una canal de comunicación para interactuar con el grupo de trabajo.
\end{itemize}

% subsection desarrollo-iteracion (end)


% section iterativo-incremental (end)
\section{Gestión de tareas}
\label{sec:gestion-tareas}

\par¿ Qué hay que hacer ? ¿ Quién tiene que hacerlo ? ¿ Cuando hay que hacerlo ?

\par \emph{David Allen} propulsor de la metodología \emph{GTD} (Getting Things Done):

\begin{quote}
    \emph{El método GTD se basa en la idea de trasladar las tareas y los proyectos previstos de la mente mediante el registro de forma externa y luego dividiéndolos en los elementos de trabajo viables. Esto permite centrar la atención en la adopción de medidas en las tareas, en lugar de en recordarlos}
\end{quote}

\par La organización centra la parte más importante en el desarrollo Iterativo e Incremental. Es el apartado donde se ha de dedicar más esfuerzo pero en donde se reciben mejores recompensas al trabajo bien hecho. Cuanto mejor se organiza la información con respecto al proceso, más y mejor crece la capacidad de afrontar nuevas iteraciones en el grupo de trabajo partiendo de la información generada a través del mismo.

\begin{figure}[H]
    \centering
    \includegraphics[width=0.6\textwidth]{todolist}
    \caption{TODO List}
    \label{fig:todolist.jpg}
\end{figure}

\par En este apartado se introduce el concepto de \emph{Gestión de Tareas} para una óptima organización. Un gestor de tareas no es otra cosa que una agenda. Limitándonos a la definición de la \emph{RAE} (Real Academia de la lengua Española):

\begin{quote}
    \emph{"Libro o cuaderno en que se apunta, para no olvidarlo, aquello que se ha de hacer."}
\end{quote}

\par Una agenda es la herramienta que permite apuntar los trabajos, citas, recordatorios, listas de la compra para organizarlos según un orden de prioridad en base a las necesidades y el tiempo.

\par El Gestor de Tareas nos permite \textbf{organizar} las tareas que se han de hacer con respecto al proyecto. Es la herramienta que contiene la \emph{Lista de Control del Proyecto} y la gestiona con respecto a cada iteración para adjuntar las tareas y repartirlas entre los miembros del grupo de trabajo.

\par El gestor de tareas va un nivel más allá y dota de vida a las tareas creando un seguimiento:

\begin{itemize}
	\item Añadiendo información. En que se basa cada tarea.
	\item Cambios de encargados de las tareas. Pueden pasar de un usuario a otro.
	\item Ciclo de vida de las tareas. Desde que se crea hasta que se finaliza (diferentes motivos).
	\item Evaluar la prioridad. Dando un peso específico a cada una de las tareas dentro del proyecto.
	\item Visibilidad a todo el grupo de trabajo. Todo el mundo puede acceder a la información dependiente de cada tareas.
	\item Permite la planificación de la iteración. Desde el nivel más bajo al más alto se puede hacer un seguimiento minucioso del estado del proyecto.
\end{itemize}

\par La gestión de la tareas depende de todo el grupo de trabajo, desde el cliente (Lista de Control), el encargado de definir las iteraciones (Obteniendo las tareas de la Lista de Control) y el grupo de desarrolladores (Gestionando las tareas conforme avanza su trabajo). En cada uno de los estados se ha de hacer una planificación de cada una de las tareas y a cada iteración de la solución la planificación será mejor con respecto a la anterior, ya que el histórico ayuda a aprender del trabajo hecho hasta la fecha.

\par Como características principales de un Gestor de Tareas podemos definir:

\begin{itemize}
	\item Gestión de recursos.
	\item Gestión de tiempo.
	\item Gestión de iteraciones.
	\item Histórico de información.
\end{itemize}

% section gestion-tareas (end)

\section{Código versionado}
\label{sec:codigo-versionado}

\par El código fuente de la solución es la clave de la misma por lo tanto ha de existir un mecanismo que permita el control y salvaguarde la información a través de los cambios durante el tiempo. Estamos hablando de un \textbf{VCS} (En Inglés \emph{Version Control System}) - \emph{Sistema de Control de Versiones}.

\par \textbf{VCS}: Version Control System. Un sistema de control de versiones es una herramienta para la gestión de archivos y su ciclo de vida dentro de un proyecto. Gestiona todas las acciones realizadas en los archivos, crear, guardar, copiar, eliminar, mover. La información se refleja en una base de datos y proporciona un histórico dotando una gestión dinámica de los recursos a los usuarios.

\par Mediante el uso de una herramienta para versionar el código se hace un seguimiento de la evolución del desarrollo de la solución. En este caso pasaremos a utilizar las siglas \textbf{SCM} (Source Code Management) - \emph{Gestión del Código Fuente}.

\par Las herramientas de versionado de código nos permiten asociar las tareas específicas para cada iteración a los cambios efectuados en el código. Agrupar los cambios por usuario, proyecto, fecha, iteración o tarea asociada. Estas dos herramientas multiplican la productividad del grupo de trabajo si se hace un buen uso de ellas, aplicando como es el caso, el uso de la metodología Iterativa e Incremental.

\par Existen dos tipos de repositorios:

\begin{itemize}
	\item \emph{Centralizados}: Los usuarios dependen de una estructura centralizada que salvaguarda toda la información. No puede trabajar sin conexión al repositorio.
	\item \emph{Distribuidos}: Cada usuario gestiona una copia completa del repositorio y unifica los cambios con otros desarrolladores estableciendo un repositorio centralizado. Puede trabajar ya que tiene una copia del repositorio en su poder.
\end{itemize}

\par En este caso, se han integrado ambos tipos de repositorio en la forja de desarrollo, como hemos dicho antes, no es excluyente a otros procesos de desarrollo. En el caso que nos ocupa, se utilizará el repositorio de tipo \emph{Distribuido} ya que otorga mayor libertad al grupo de trabajo.

\par El desarrollo de la solución se adapta al uso del Repositorio distribuido, ya que éste aporta una flexibilidad para la gestión de bifurcaciones del código que en un repositorio centralizado no tenemos fácilmente\footnote{Git Svn Comparison - \url{https://git.wiki.kernel.org/index.php/GitSvnComparison}}. 

\subsection{Desarrollo por Ramas}
\label{sub:desarrollo-ramas}

\par Los repositorios de código son una herramientas más en el proceso de desarrollo de software por lo tanto hemos de utilizar aquel que se adapte a las necesidades del proceso. En este caso un repositorio distribuido y a su vez, aplicar una metodología de desarrollo a la herramienta.

\par El \emph{Desarrollo por Ramas} (Feature branch\footnote{Martin Fowler - Feature branch - \url{http://martinfowler.com/bliki/FeatureBranch.html}}) o \emph{Desarrollo por Canales}.

\begin{figure}[H]
    \centering
    \includegraphics[width=0.8\textwidth]{feature_branch_simple}
    \caption{Desarrollo por Ramas}
    \label{fig:feature_branch_simple}
\end{figure}

\par El Desarrollo por ramas ayuda a la creación de código de una forma ordenada y coordinada dentro de un grupo de trabajo. Se basa en la creación y la integración continua de nuevas funcionalidades o solventar problemas derivados de otras implementaciones de una manera eficiente, ágil y distribuida por canales.

\par El desarrollo basado en canales o ramas se ha definido dentro del proceso de desarrollo Iterativo e Incremental en donde ayuda a la integración del código en la rama principal de desarrollo. Como se ha definido en el punto referente al proceso de desarrollo, cada iteración consta de un incremento en el que se han agrupado las tareas (de la Lista de Control) a implementar. Por lo tanto cada tarea se corresponde al desarrollo de una funcionalidad o cobertura de un error. Estamos hablando de desarrollos pequeños y que fácilmente se integran en la rama principal del proyecto.

\begin{quote}
    \emph{Divide y Vencerás}: resolver un problema difícil, dividiéndolo en partes más simples tantas veces como sea necesario, hasta que la resolución de las partes se torna obvia.
\end{quote}

\par Fragmentar la solución en pequeños módulos organizados facilitando el desarrollo y aislando el ruido y además poder controlar las posibles \emph{excepciones} o \emph{errores} que se produzcan minimizando el daño y el alcance. Pero para ser eficientes ha de existir una herramienta que ayude a coordinar los cambios y el esfuerzo invertido para no desvirtuar el proceso de desarrollo.

% subsection desarrollo-ramas (end)

% section codigo-versionado (end)

\section{TDD y CI}
\label{sec:tdd-ci}

\par En este punto se introduce un nuevo apéndice del proceso junto a la herramienta que ayuda a que el desarrollo del código no se desvirtúe mediante el desarrollo por ramas. La \emph{Integración Continua} (CI) y el \emph{Desarrollo Dirigido por Tests} (TDD).

\subsection{TDD}
\label{sub:tdd}

\par El TDD se basa en la repetición de un ciclo de desarrollo corto en el que el desarrollador escribe un caso de prueba (test). Este ha de definir el comportamiento que se busca en la nueva funcionalidad y ha de fallar. A partir de este instante ya están definidos los requerimientos de la funcionalidad en caso de prueba y se ha escribir el código necesario para que a través del test se obtenga el resultado esperado. Acto seguido se ha de refactorizar el código y se da por terminada la tarea.

\begin{figure}[H]
    \centering
    \includegraphics[width=0.7\textwidth]{TDD-Diagram}
    \caption{Diagrama de los estados del proceso TDD}
    \label{fig:TDD-Diagram}
\end{figure}

\par Este pequeño ciclo de vida se repite constantemente para orientar el desarrollo basado en las necesidades y de esta forma controlar el resultado de los requerimientos y generar una trazabilidad alrededor del código. La solución desarrollada queda integrada en la solución global en donde se incluye, por lo tanto se aumenta la seguridad a través del aislamiento de cada una de las funcionalidades o pasos para que no interfieran entre ellos.

\par Este proceso de desarrollo proporciona una estructura adecuada para la evolución del código desarrollado ya que facilita el proceso de 'refactoring', control de la deuda técnica en un proyecto y la cobertura de tests.

\par TDD significa un desarrollo responsable de la solución fraccionando cada implementación dentro de las iteraciones correspondientes. El TDD es \emph{el día a día} de un desarrollador.

\par Es necesario saber cuando se han de implementar los test ya que hay niveles en los que no existe funcionalidad ni complejidad alguna y \emph{no merece la pena} dedicar esfuerzo a ello. Como puede ser el caso de los test en métodos \emph{getter y setter} en un objeto Java.

\par Las buenas prácticas para el TDD:
\begin{itemize}
	\item Tener el código separado de los tests, en carpetas diferentes.
	\item Los test forman parte de cada iteración.
	\item Los desarrolladores son responsables de testar el código que escriben.
	\item Las herramientas y los procesos están altamente automatizados.
	\item El equipo de QA se encarga de mejorar las herramientas.
	\item Los tests deben fallar la primera vez que son escritos.
	\item Los nombre de los tests deben ir acorde con la intención, deben ser nombres expresivos.
	\item Refactorizar para eliminar código duplicado después de pasar los tests.
	\item Repetir las pruebas después de cada refactorización.
	\item Solo se debe escribir nuevo código, cuando algún test falla. Cada test debe comprobar un nuevo comportamiento o diferente.
	\item Escribe primero el \emph{assert}.
	\item Minimiza los \emph{asserts} en cada test.
	\item Todos los tests deben ser pasados antes de escribir otro test.
	\item Solo se refactoriza cuando todos los tests pasan.
	\item Escribe el mínimo y simple código para pasar las pruebas.
	\item No usar las dependencias entre tests. Los tests deben pasar en cualquier orden.
	\item Los tests deben ser rápidos.
	\item Usa Mock objects para testear código cuando haya alguna limitación, y así ejecutar los tests más rápido.
\end{itemize}

% subsection tdd (end)

\subsection{Integración Continua}
\label{sub:integracion-continua}

\begin{quote}
    \emph{Continuous Integration is a software development practice where members of a team integrate their work frequently, usually each person integrates at least daily - leading to multiple integrations per day. Each integration is verified by an automated build (including test) to detect integration errors as quickly as possible}\footnote{Martin Fowler Continuous Integration - \url{http://martinfowler.com/articles/continuousIntegration.html}}
\end{quote}

\par La \emph{Integración Continua} (CI - Continuous Integration) es un \emph{modelo informático} propuesto inicialmente por \emph{Martin Fowler} que consiste en hacer integraciones automáticas de un proyecto lo más a menudo posible para así poder detectar fallos cuanto antes.

\par La CI está basada en dos principios básicos dentro del proceso de desarrollo de software, la \emph{integración de la compilación} y la \emph{ejecución de tests de todo un proyecto}. De esta manera la fiabilidad de la solución desarrollada aumenta y se obtiene una visión a grandes rasgos del código fuente. Permite la localización temprana de bugs (errores), gestionando de una forma eficiente la evolución de la solución desarrollada.

\par El resultado del proceso se proporciona a los desarrolladores para continuar con el trabajo en la tarea correspondiente:

\begin{itemize}
	\item Si el resultado es \textbf{positivo}: el siguiente paso es cerrar la tarea y empezar con la siguiente.
	\item Por otra parte, un resultado \textbf{negativo}: la herramienta de CI alerta al desarrollador para que evalúe de nuevo la solución desarrollada. A partir de los informes generados el desarrollador tiene más información para subsanar el error. Al volver a completar el ciclo se envía de nuevo el código a la herramienta CI para comprobar el correcto funcionamiento.
\end{itemize}

\par Este proceso de automatiza para evitar la repetición de tareas repetitivas (DRY) y agilizar el desarrollo de la solución. Se ha de invertir tiempo para la gestión de esta herramienta ya que si la herramienta no se utiliza correctamente va a traer más dolores de cabeza que alegrías.

\begin{figure}[H]
    \centering
    \includegraphics[width=0.8\textwidth]{continuous-integration-graphic}
    \caption{Ciclo de la Integración Continua}
    \label{fig:continuous-integration-graphic}
\end{figure}

\par Este apartado culmina el proceso de desarrollo Iterativo e Incremental ya que en el confluyen todas las herramientas utilizadas unificando los resultados, haciendo un seguimiento de la evolución y generando la solución final por cada iteración.

\par En el proceso de desarrollo por ramas, cada desarrollador se encarga de su desarrollo asociado a una tarea dentro de la iteración, generación de test e integración en el sistema de CI. A partir de este instante el sistema de CI es el encargado de evaluar el código proporcionado por el desarrollador, evaluando de forma automática cada uno de los pasos descritos:

\begin{itemize}
	\item Compilación.
	\item Tests.
	\item Integración del código fuente en la rama de desarrollo de la iteración.
	\item Despliegue de la solución.
	\item Tests de integración.
	\item Generación de una solución funcional (Continuous Delivery).
\end{itemize}

\par Esta nueva iteración dentro del proceso de desarrollo mantiene la viabilidad del proyecto activa a través de una rama de desarrollo preparada para cada iteración. En esta rama se unifican los desarrollos que han pasado las pruebas anteriores desde la compilación a la generación de una solución funcional. \emph{El resultado está listo para entregar al cliente}.

\begin{figure}[H]
    \centering
    \includegraphics[width=0.8\textwidth]{dev_graphs_agilemethodologies}
    \caption{Proceso desarrollo con Integración Continua}
    \label{fig:dev_graphs_agilemethodologies}
\end{figure}

\par Se cierra el circulo de una iteración a través de la herramienta de CI automatizando los pasos repetitivos del ciclo particular hasta completar una solución que aúne los requerimientos establecidos en la iteración.

% subsection integracion-continua (end)

% section tdd-ci (end)


%%%%%%%%%%%%%%%%%%%%%%%%%%%%%%%%%%%%%%

\chapter{ALM Tools - Forjas de desarrollo}
\label{chap:almtools}

\begin{comment}
* ALM Tools
    * Qué es una forja
    * Objetivos
    * Componentes
        * Estudio del arte de forjas
    * Problemática -> administración, costes...
        * Tablas comparativas
    * Algunos ejemplos y sus limitaciones
    * Conclusiones del estudio de forjas
\end{comment}

\par ALM Tools: \emph{Application Lifecycle Management Tools}. \emph{La gesti\'on del ciclo de vida de una aplicaci\'on}, es decir el conjunto de herramientas encargadas de guiar al desarrollador a trav\'es de un camino basado en metodolog\'ias, que establecen el ciclo de vida, para la desarrollo de un Software hacia su estado del arte.

\par En el desarrollo de SidelabCode Stack se ha buscado la integración del proceso de desarrollo Iterativo e Incremental a trav\'es de las ALM Tools como veh\'iculo. No existe en si mismo una herramienta ALM, sino que la herramienta ALM gobierna a las herramientas que participan en el proceso de desarrollo desarrollando una guía de buenas prácticas.

\section{Historia}
\label{sec:historia}

\par En todas las empresas o comunidades que desarrollan Software siempre se aplica un proceso de desarrollo a la creación del producto. Cada una utiliza distintas herramientas para la gesti\'on de los proyectos de Software, gestor de correo, gestor de incidencias, repositorio de c\'odigo, integraci\'on continua. Pero en la mayor\'ia de los casos de una forma dispar y sin seguir ninguna convención.

\par A veces el no conocimiento de otras herramientas o la no inclusión de nuevas puede hacer que el desarrollo del proyecto no mejore, partiendo de la base de que el desarrollo puede ser óptimo para las herramientas utilizadas, carece de perspectivas de mejora a corto plazo.

\par Si se opta por la integración de una nueva herramienta en el proceso de desarrollo el coste de integración se habría de evaluar ya que se debería dedicar un esfuerzo a la integración ad-hoc de la nueva herramienta para el uso en este mismo entorno con el coste que conlleva, evaluación, test, integración, interoperabilidad, es decir un nuevo proyecto dentro del mismo proyecto.

\par Después de esta integración en el proceso de desarrollo en la empresa habría que hacer un esfuerzo para salvaguardar la información a través de las distintas herramientas por separado.

\par El proceso de unificación y reutilización de herramientas a los desarrolladores nos puede parecer familiar si lo comparamos con el uso de los Frameworks a principios de los años 2000. Muchas empresas o comunidades empezaron a adecuar e implementar sus desarrollos en base a un Framework creado por ellos mismos. Estos Frameworks se adecuan a sus requerimientos pero su uso era interno en la empresa y por lo tanto lejano a los estándares. Uno de los más famosos es sin duda el caso de Spring, proyecto de más de 10 años de edad que goza de buena salud y aceptación, incluso equivoca a algunos entre Java y Spring. En este pequeño paralelismo podemos encontrar el estado de las forjas de desarrollo ALM Tools, cuando el proyecto requiere de herramientas para facilitar su ciclo de vida y se van ensamblando una tras otra, que perfectamente las podemos llamar librerías, en una integración \textbf{ad-hoc} y siguiendo unos pasos repetitivos en cada nuevo proyecto, en los que humanamente todos nos podemos equivocar debido a que depende de cada uno seguir cada uno de los pasos. Estas herramientas tienden a convertirse en pequeños estándares dentro de cada grupo de desarrolladores y a repetirse en futuros proyectos, pero debido a los desarrollos \textbf{ad-hoc} carecen de escalabilidad e integración con nuevas soluciones de una forma ágil, es un escollo actualizar y por otro lado replicar un estándar para la implantación de la forja, no se tiende a dejar puertas abiertas para que más adelante el herramienta mejore. Se podría definir como uno de los casos de inanición en el desarrollo de Software o muerte por éxito.

\par En este punto es donde entra la famosa interoperabilidad entre las herramientas que define una comunicación estándar. El punto clave de las ALM Tools, la \textbf{integración e interacción de las herramientas} dentro de un proceso de desarrollo.

\par En post de evitar la falta de replicación, además de la importancia de la interoperabilidad, se ha de tener en cuenta la replicación del contenido o la gestión de la instalación de un ALM. Siempre se ha de pensar mirando un paso por delante. No es necesario implementar las mejoras pero sí, dejar un espacio o conector para que casen bien. Un ejemplo que puede ilustrar esta frase es la programación basada en Interfaces en Java, ya que las Intefaces ofrecen soluciones para implementar a medida a partir de un \emph{esqueleto}, si se actualiza la Interfaz (en este caso es el esqueleto de la clase) para añadir una nueva funcionalidad con un método, los métodos anteriores mantienen su comportamiento dentro de cada clase que la implementa y adquieren la posibilidad de aumentar la funcionalidad implementando la nueva solución, adecuada a su entorno, de esta forma la interoperabilidad entre las clases que utilicen esta Interfaz también se mantiene ya que todas implementan las funcionalidades de la Interfaz como clase en la que pivotan.

% subsection historia (end)

\section{Qué es una forja}
\label{sec:que-es}

\par Partimos de la definición que ofrece \emph{Cenatic} en el título de su estudio sobre forjas:

\begin{quote}
    \emph{Entorno de desarrollo colaborativo de Software}
\end{quote}

\par Una forja de desarrollo es una herramienta que actúa como elemento catalizador de este proceso abierto de desarrollo. Las forjas juegan un papel clave para aprovechar las ventajas de las metodologías, en este caso a través del proceso de desarrollo Iterativo e Incremental, aportando múltiples ventajas:

\begin{itemize}
	\item Mayor eficiencia.
	\item Mejor calidad en el producto final.
	\item Reutilización de esfuerzos.
	\item Modularidad.
	\item Adaptación a estándares.
	\item Mayor agilidad en el proceso de desarrollo.
\end{itemize}

\par Esta herramienta permite centrar toda la atención y potencial del desarrollador en el mismo desarrollo. Facilita la evolución del código desarrollado y el seguimiento del proyecto a todos los niveles.

\par Además las forjas de desarrollo aportan la sencillez para instaurar este entorno de desarrollo en el grupo de trabajo. Facilita la instalación, la replicación de contenido, la comunicación y la publicación de resultados del desarrollo de cada proyecto al alance de los usuarios de la forja.

\par Una forja es una comunidad en donde convergen proyectos de software (en el caso que nos atañe) a través de las interacciones de los usuarios con el código. Toda la información gira en torno al repositorio de código. El valor aportado por una forja en pos del repositorio por si sólo es la interoperabilidad con otras herramientas para la mejora de la gestión del desarrollo del proyecto; manuales, tickets, diagramas, planificaciones, wiki, revisiones, roles.

\par Tener las herramientas necesarias no es suficiente para obtener un desarrollo fluido, fiable y deseable. Se han de conocer las herramientas que componen la forja, no las herramientas en si, sino su funcionalidad, para que de esta forma se pueda definir un proceso de desarrollo basado en las herramientas existentes o añadiendo nuevas herramientas para que se adapten al proceso de desarrollo elegido. 

\par \emph{La forja ha de trabajar para los usuarios, pero los usuarios han de saber como hacer que la forja trabaje para ellos.}

% section que-es (end)

\section{Objetivos}
\label{sec:objetivos}

\par El objetivo principal de una forja de desarrollo es poder sacar el máximo partido al desarrollo del proyecto ayudando a establecer el proceso de desarrollo elegido para cada proyecto dentro de un entorno integrado de trabajo.

\par Disponer de un entorno integrado de trabajo en el que concentrar los esfuerzos de usuarios permite también obtener una serie de ventajas desde el punto de vista de la potenciación del propio proceso de desarrollo, desde varios puntos de vista:

\begin{itemize}
	\item \emph{Mejor control de esfuerzos}: Ya que todos los usuarios están en contacto a través del entorno de la forja, los responsables de asignación de tareas encuentran más facilidades para identificar los miembros más adecuados para resolver determinadas necesidades, así como detectar de forma temprana cualquier problema importante que pueda comenzar a surgir dentro del proyecto, desde el punto de vista de recursos necesarios para sus mantenibilidad.
    \item \emph{Aspectos legales y de licencias}: Dentro de la forja, hay cabida para poder indicar de forma clara y explícita toda la información necesaria sobre licencias bajo las que se distribuye los productos del proyecto. Este es un aspecto muy importante para poder garantizar la compatibilidad de los productos del proyecto con otras soluciones desarrolladas en otras iniciativas, evitando la posterior aparición de problemas legales o incompatibilidades en fases de integración más avanzadas.
    \item \emph{Homogeneizar las prácticas y estilo desarrollo}: En proyectos de desarrollo la documentación de la forja y el establecimiento por parte de los usuarios de una serie de directrices que marquen el proceso de desarrollo, las herramientas que se debe utilizar, así como la política que debe seguirse en diferentes estratos de los flujos de trabajo es crucial para mantener un entorno de trabajo efectivo y homogéneo que asegure la cohesión de los diferentes elementos generados para dar como resultado productos mucho más estables, integrados y de mayor calidad.
    \item \emph{Establecimiento de guías de evolución}: La forja es el entorno ideal para poder anunciar y consensuar entre todos los usuarios una guía de evolución clara del proyecto, no solo desde el punto de vista del desarrollo formal de código, sino también de los objetivos generales y de utilidad que se persiguen dentro de la iniciativa para así poder mejorar el mismo proceso de desarrollo.
\end{itemize}

\par El proceso Iterativo e Incremental requiere un seguimiento constante de cada una de las iteraciones a partir de los objetivos establecidos en la Lista de Control. 

\begin{itemize}
	\item En cada iteración se ha de desarrollar la solución asociada a cada requerimiento basando el desarrollo en la orientación a test (TDD) partiendo de una nueva rama de desarrollo.
	\item Acto seguido el resultado se ha de implantar en la rama de desarrollo de la iteración, por lo que se maneja a través del servidor de integración continua para poder adjuntar el desarrollo a la iteración dependiendo si el resultado ha sido positivo o no.
	\item De esta manera se completan los pequeños ciclos de vida que ha de manejar la forja.
\end{itemize}

\par La gestión del ciclo de vida del proceso de desarrollo se divide en el uso de las siguientes herramientas que componen la forja ALM \emph{SidelabCode Stack}:

\begin{itemize}
    \item \emph{Usuarios y Roles} - Gestión de los usuarios, permisos y roles para cada proyecto a través de un directorio de identificación.
	\item \emph{Gesti\'on de Requisitos} - A través de un \emph{Issue tracking system}. Herramienta para gestionar los requisitos y estado actual de cada uno de ellos accesible a los desarrolladores del proyecto.
	\item \emph{Gestión de Repositorios} - Herramienta para la gestión de repositorios; centralizados o distribuidos.
	\item \emph{Ciclo de vida} - Desarrollo y seguimiento de las soluciones mediante el ciclo de vida establecido; requisitos, test, desarrollo, integración continua.
	\item \emph{Gestión de Despliegues} - después de cada Iteración finalizada y publicación del resultado.
\end{itemize}

\par La inclusión de estas herramientas para facilitar el proceso de desarrollo Iterativo e Incremental a través del uso de la forja SidelabCode Stack es lo que permite definir cada comportamiento dentro del proceso mediante la comunicación entre cada una de ellas.

\par Esta definición del proceso de desarrollo se plasma en la guía que proporciona SidelabCode Stack para el desarrollo Iterativo e Incremental.

% section objetivos (end)

\section{Estado del arte de forjas}
\label{sec:estado-del-arte}

\par Hoy en d\'ia las forjas ALM abundan, adem\'as de gozar de una gran popularidad entre los proyectos de Software Libre, como podemos ver en los casos de SourceForge, Googlecode, Bitbucket y Github (más adelante discutiremos cada proyecto). En este caso como \emph{SaaS} (Software as a Service) debido al servicio que ofrecen. El punto diferenciador se encuentra en las forjas ALM las que son \emph{FLOSS}, ya que permiten replicar ese mismo entorno en tu propia máquina o poder trasladar el proyecto de una herramienta a otra. Un dato muy importante a tener en cuenta, porque siempre se ha de mirar hacia adelante. Sobretodo destacan porque si se crea una necesidad con respecto a la gestión, al ser FLOSS siempre se puede implementar con el propio conocimiento del desarrollador a diferencia de las que no son FLOSS. En este caso el usuario \emph{depende de la propia compañía}, un tercero que es el único que puede implementar la solución convirtiendo el servicio en \emph{Vendor-Locking}.

\par En esta lista se muestran las forjas de desarrollo ALM más populares:

\begin{itemize}
	\item SourceForge con Allura - \url{http://sourceforge.net/projects/allura/} - \textbf{FLOSS}.
	\item Cloudbees DEV@Cloud \url{http://www.cloudbees.com/dev.cb} - \textbf{SaaS privado}.
	\item CollabNet con CloudForge \url{http://www.cloudforge.com/} - \textbf{SaaS privado}.
	\item Plan.io - \url{http://plan.io/en/} - \textbf{SaaS privado}.
	\item ClinkerHQ \url{http://clinkerhq.com/} - \textbf{SaaS privado}.
	\item Github - \url{http://github.com/} - \textbf{SaaS privado}.
    \item GitlabHQ - \url{http://gitlab.org/} - \textbf{FLOSS}.
	\item Bitnami - \url{http://bitnami.com/} - \textbf{FLOSS}.
	\item GForge - \url{http://gforge.org/gf/} - \textbf{FLOSS}.
	\item Collab.net - url{http://www.collab.net/} - \textbf{SaaS privado}.
	\item Google Code - \url{http://code.google.com/intl/en/}.
	\item Bitbucket - \url{https://bitbucket.org/} - \textbf{SaaS privado}.
\end{itemize}

% subsection estado-del-arte (end)

\subsection{Casos de Uso}
\label{sub:casos-de-uso}

\par Nos vamos a servir de la investigación hecha por \emph{Cenatic}~\cite{cenatic-forjas} sobre el uso de las forjas y el estudio sobre el incremento de la productividad por \emph{Dirk Riehle}~\cite{open-collaboration-forges} en la implantación de una forja de desarrollo.

\par A partir de estos estudios el caso se muestran los casos de \emph{SAP Forge} basado en GForge y \emph{GoogleCode} con su propia forja. Dos casos de éxito de implantación de una forja desde dos planteamientos diferentes en el uso: internamente y como servicio para proyectos FLOSS (además de internamente).

\subsubsection{SAP}
\label{subsub:sap}

\par Como ya hemos indicado anteriormente, ~\cite{open-collaboration-forges} nos presenta un estudio detallado del caso de utilización de una forja para el lanzamiento y desarrollo de proyectos software dentro de la intranet empresarial. En este caso, se eligió GForge como paquete para proporcionar las funcionalidades básicas de una forja, sobre el que se elaboró un entorno ligeramente más personalizado. De este modo, se presenta la forja de SAP como un entorno de desarrollo colaborativo, centralizado, y fácilmente accesible por cualquier miembro de la empresa que desee participar, gracias a su acceso a través de una URL interna y de fácil memorización. Es importante remarcar que cualquier trabajador dentro de los limites de la red interna puede acceder a la forja.

\par Tras su primer año de andadura, según ~\cite{open-collaboration-forges} SAP Forge había atraído más de 100 proyectos y unos 500 usuarios registrados, lo que aproximadamente representa el 5\% de la población total de desarrolladores \emph{SAP}. La inclusión de un proyecto dentro de la forja no era obligatorio, sino potestad del líder del proyecto.

\par Como datos a destacar en este análisis hay tres características que han de ser mencionadas después de la puesta en marcha del proyecto \emph{SAP Forge}:

\begin{itemize}
	\item Búsqueda de proyectos: Un recurso para los usuarios en el que están indexados todos los proyectos a partir de sus metadatos: nombre, descripción, etc. Todos los proyectos están accesibles para los usuarios (miembros de la empresa) para la consulta. Aplicando el modelo de colaboración abierta del Software Libre, cuantos más ojos interesados, mejores resultados.
	
	\item Información de los desarrolladores: Se genera una base de datos de conocimientos asociados a cada uno de los desarrolladores que son usuarios de la forja. Disponen de una lista de aptitudes obtenida a partir de la información que genera la forja de desarrollo. Esta información facilita la elección de, según los requisitos, poder encontrar a la persona idónea para implementar la solución, cercar las búsquedas y obtener mejores referencias de los trabajadores.
	
	\item Publicidad del proyecto: Se puede hacer un seguimiento de la vida del proyecto a partir de los usuarios, listas de correo, tráfico generado. De esta forma se pueden encontrar los proyectos más usados, útiles y activos, definiendo intereses y reconociendo patrones de casos de éxito aplicables a otros proyectos.
	
\end{itemize}

\par El texto ~\cite{open-collaboration-forges} nos expone el caso de un proyecto (\emph{Mobile Retail Demo}) que se introdujo en la \emph{SAP Forge}. A partir de este movimiento el interés y la colaboración en el proyecto se incrementaron por parte de los desarrolladores. El uso, éxito y la calidad crecieron exponencialmente ligados a la colaboración que había facilitado la forja a los usuarios.

\par Con unas mediciones más exactas basadas en una encuesta interna a los desarrolladores participantes el los proyectos de la forja, se puede ver reflejado el grado de satisfacción por los resultados obtenidos en cuanto a los desarrollo de los proyectos y la colaboración entre ellos:

\begin{quotation}
    \emph{De un total de 83 participantes en la encuesta, un 66\% indicó que habían utilizado las herramientas de búsqueda de proyectos de la \emph{SAP Forge}, para localizar otros proyectos que fuesen de su interés. Un 24\% indicaron que su proyecto había recibido ayuda del exterior (dentro del ámbito interno de SAP), principalmente en forma de reportes de error y sugerencias de mejora. Finalmente, un 12\% de los encuestados comentaron que finalmente colaboraron con otros proyectos diferentes basándose en sus preferencias personales para realizar la selección.}
\end{quotation}

\subsubsection{GoogleCode}
\label{subsub:googlecode}

GoogleCode es el proyecto insignia de Google en el ámbito de las forjas de desarrollo de
código. Se trata de un proyecto para proporcionar espacio web y herramientas de soporte
al desarrollo colaborativo de software, abierto a cualquier grupo o desarrollador individual
interesado en publicitar su proyecto.
Desde su apertura en julio de 2006, la plataforma ha recibido un asombroso número de
peticiones de alojamiento, y en la actualidad podemos encontrar más de 170.000
proyectos diferentes hospedados en GoogleCode. Además, una proporción significativa
de estos proyectos (unos 40.000, lo que supone el 23.5% del total) ha registrado algún
tipo de actividad durante los últimos 90 días.
La tarea de clasificar cuáles de estos proyectos han sido originados por empresas, o bien
están directamente ligados a su aplicación en entornos corporativos es un tanto
complicada. Sin embargo, podemos encontrar algunos proyectos muy conocidos entre el
listado disponible en GoogleCode:

Digg1: El conocido portal de indexación, agregación de contenidos y comentarios
 ha liberado cierto número de herramientas que utiliza dentro de su código. Entre
  ellas, destacan librerías que proporcionan funcionalidades adicionales con
  JavaScript o automatización de actualizaciones de grandes cantidades de
  información en bases de datos.
JQuery2: Una conocida librería que permite aumentar las funcionalidades de
    JavaScript para crear sitios web basados en tecnologías Ajax.
Firebug3: Herramienta que se integra en Firefox para permitir el acceso dinámico a
    utilidades de desarrollo para editar y depurar código HTML, estilos CSS o código
     JavaScript en una página web.
SWFObject4: Se trata de un potente reproductor de Flash integrado, mucho más
    versátil que las opciones tradicionales y con una API que permite integración más
      completa con código JavaScript.
FriendFeed5: Librería que proporciona una API para interactuar con el sito web
    FriendFeed.

Cabe destacar que GoogleCode solamente acepta hospedaje de proyectos que sean
liberados bajo algún tipo de licencia libre, concretamente soporta siete tipos de licencias
diferentes: Apache License 2.0, Artistic License/GPLv2, GNU General Public License 2.0, GNU Lesser Public License, MIT License, Mozilla Public License 1.1, New BSD License.
Además, se puede licenciar el contenido de la documentación que acompaña a los
proyectos bajo licencia Creative Commons.

Desde el punto de vista de interfaz gráfica, Google sigue fielmente su política de entornos
minimalistas que no distraigan al usuario con funcionalidades innecesarias, y ofrece un
esquema muy sencillo y directo. Prueba de la gran acogida de esta iniciativa es el
elevadísimo número de proyectos que han decidido liberar su código en esta plataforma, y
que además ofrece las clásicas funcionalidades adicionales soportadas por otras
plataformas de la compañía, como el sistema de búsquedas avanzadas. Simplemente por
el hecho de la gran cantidad de código que alberga, GoogleCode es en la actualidad uno
de los repositorios más consultados a la hora de localizar librerías y aplicaciones ya
existentes.

% subsection casos-de-uso (end)

\subsection{Problemas en algunas forjas}
\label{sub:problemas}

% subsection problemas (end)

\subsection{Tablas comparativas}
\label{sub:comparativa}

% subsection comparativa (end)

\section{Conclusiones del estudio de forjas}
\label{sec:conclusiones}

% subsection conclusiones (end)


%%%%%%%%%%%%%%%%%%%%%%%%%%%%%%%%%%%%%%

%%%%%%%%%%%%%%%%%%%%%%%%%%%%%%%%%%%%%%%%%%%%%%%%%%%%%%%%%%%%%%%%%%%%%%%%%%%%%%%%%%%%%%%%%%%%%%%%%%%%%%%%%%%
%   \copyright 2013 Ricardo García Fernández - ricardogarfe [at] gmail [dot] com.
%
%    This work is licensed under a Creative Commons 3.0 Unported License.
%    To view a copy of this license visit:
% 
%    http://creativecommons.org/licenses/by/3.0/legalcode
%%%%%%%%%%%%%%%%%%%%%%%%%%%%%%%%%%%%%%%%%%%%%%%%%%%%%%%%%%%%%%%%%%%%%%%%%%%%%%%%%%%%%%%%%%%%%%%%%%%%%%%%%%

%%%%%%%%%%%%%%%%%%%%%%%%%%%%%%%%%%%%%%%%%%%%%%%%%%%%%%%%%%%%%%%%%%%%%%%%%%%%%%%%%%%%%%%%%%%%%%%%%%%%%%%%%%
% SCStack
%%%%%%%%%%%%%%%%%%%%%%%%%%%%%%%%%%%%%%%%%%%%%%%%%%%%%%%%%%%%%%%%%%%%%%%%%%%%%%%%%%%%%%%%%%%%%%%%%%%%%%%%%%

\chapter{SCStack}
\label{chap:scstack}

\par SCStack es el conjunto de herramientas que componen la Forja de desarrollo. La palabra \emph{stack} en Inglés significa \emph{pila}, con respecto a la forja se trata del conjunto de herramientas apiladas y conectadas que dan forma a la Forja.

\begin{figure}[H]
    \centering
    \includegraphics[width=0.7\textwidth]{stackstorage}
    \caption{Ejemplo de pila a partir de cajones}
    \label{fig:stackstorage}
\end{figure}

\section{Arquitectura}
\label{sec:arquitectura}

\par La arquitectura del Sistema de Gestión de la Forja se ve reflejada en el siguiente esquema:

\begin{figure}[H]
    \centering
    \includegraphics[width=\textwidth]{scstack-diagram}
    \caption{Diagrama de arquitectura SCStack}
    \label{fig:scstack-diagram}
\end{figure}

\par El proyecto consta de tres componentes conectados entre sí para la comunicación del usuario con las herramientas para la gestión de los proyectos: La Consola de Administración, el servidor de servicios web REST y el API.

%%%%%%%%%%%%%%%%%%%%%%%%%%%%%%%%%%%%%%%%%%%%%%%%%%%%%%%%%%%%%%%%%%%%%%%%%%%%%%%%%%%%%%%%%%%%%%%%%%%%%%%%%%
% Consola de Administración
%%%%%%%%%%%%%%%%%%%%%%%%%%%%%%%%%%%%%%%%%%%%%%%%%%%%%%%%%%%%%%%%%%%%%%%%%%%%%%%%%%%%%%%%%%%%%%%%%%%%%%%%%%

\subsection{Consola de Administración}
\label{sub:consola-admin}

\par \textbf{Consola de Administración}: Se trata de la capa de la vista encargada de interactuar con el usuario. A través de la interfaz que se ejecuta sobre un navegador web, el usuario administrador gestiona los usuarios y los proyectos de la Forja: crear, editar, modificar, eliminar y buscar), el típico sistema CRUD-F\footnote{\url{http://en.wikipedia.org/wiki/Create,\_read,\_update\_and\_delete}} (\emph{CRUD: Create, Read, Update and Delete also Find}). Está diseñada con el framework Javascript \emph{jQuery} por lo que no requiere ninguna librería externa para su uso, únicamente el navegador del cliente. Las operaciones que se realizan en la capa UI se trasladan al servidor REST a través de peticiones \emph{http} asíncronas mediante las interfaces REST definidas.

% subsection consola-admin (end)

%%%%%%%%%%%%%%%%%%%%%%%%%%%%%%%%%%%%%%%%%%%%%%%%%%%%%%%%%%%%%%%%%%%%%%%%%%%%%%%%%%%%%%%%%%%%%%%%%%%%%%%%%%
% Servicio web REST
%%%%%%%%%%%%%%%%%%%%%%%%%%%%%%%%%%%%%%%%%%%%%%%%%%%%%%%%%%%%%%%%%%%%%%%%%%%%%%%%%%%%%%%%%%%%%%%%%%%%%%%%%%

\subsection{Servicio web REST}
\label{sub:rest-ws}

\par \emph{REST}: Representational State Transfer se trata de una arquitectura acuñada por \emph{Roy Fielding}\footnote{\url{http://roy.gbiv.com/}}, uno de los autores de la especificación del protocolo \emph{HTTP}. Esta arquitectura de comunicación se basa en cuatro principios:

\begin{itemize}
	\item Un protocolo cliente/servidor \emph{sin estado}: cada mensaje HTTP contiene toda la información necesaria para comprender la petición. Como resultado, ni el cliente ni el servidor necesitan recordar ningún estado de las comunicaciones entre mensajes. Sin embargo, en la práctica, muchas aplicaciones basadas en HTTP utilizan cookies y otros mecanismos para mantener el estado de la sesión.

	\item Un \emph{conjunto de operaciones} bien definidas que se aplican a todos los recursos de información: HTTP en sí define un conjunto pequeño de operaciones, las más importantes son \textbf{POST}, \textbf{GET}, \textbf{PUT} y \textbf{DELETE}.

	\item Una sintaxis \emph{universal} para identificar los recursos. En un sistema REST, cada recurso es direccionable únicamente a través de su URI.

	\item El uso de \emph{hipermedios}: tanto para la información de la aplicación como para las transiciones de estado de la aplicación. Como resultado de esto, es posible navegar de un recurso REST a muchos otros, simplemente siguiendo enlaces sin requerir el uso de registros u otra infraestructura adicional.
\end{itemize}

\par El uso de los servicios REST para la comunicación mediante el protocolo http con la Consola de Administración agiliza la adaptación de la funcionalidades al cliente, el tráfico y las posibles adaptaciones de nuevos clientes para el acceso a través de cualquier plataforma.

\begin{figure}[H]
    \centering
    \includegraphics[width=1\textwidth]{rest-scope}
    \caption{Diseño de un API RESTful}
    \label{fig:rest-scope}
\end{figure}

\par \textbf{Servicio web REST}: Es el componente encargado de gestionar las peticiones http a los servicios REST de la Consola de Administración y el API de SidelabCode. Define e implementa la lógica a seguir para la construcción de un proyecto a través de la UI mediante las llamadas ordenadas al API de cada una de las herramientas involucradas en el servicio como componentes de la forja.

\par Los servicios REST ofrecen tres interfaces distintas accesibles a través de http: XML, JSON y HTML para la comunicación la API. Además, hay que desatacar que el servicio web es servido a través del framework \emph{Restlet} a través de Internet de forma continua y remota a cualquier usuario o proceso cliente.

\par Se encarga de la seguridad y validación para las operaciones permitidas o denegadas del usuario de la UI, proporcionando, según su rol, determinados servicios a través de la Consola de Administración, buscar proyectos, crear usuarios, editar proyectos, crear proyectos, etc.

% subsection rest-ws (end)

%%%%%%%%%%%%%%%%%%%%%%%%%%%%%%%%%%%%%%%%%%%%%%%%%%%%%%%%%%%%%%%%%%%%%%%%%%%%%%%%%%%%%%%%%%%%%%%%%%%%%%%%%%
% API
%%%%%%%%%%%%%%%%%%%%%%%%%%%%%%%%%%%%%%%%%%%%%%%%%%%%%%%%%%%%%%%%%%%%%%%%%%%%%%%%%%%%%%%%%%%%%%%%%%%%%%%%%%

\subsection{API}
\label{sub:api}

\par \textbf{API}: La API es el núcleo funcional de SCStack, coordina y realiza las tareas para cada componente de la forja con respecto a los usuarios y proyectos involucrados. Traslada las órdenes ejecutadas en la capa UI a las distintas herramientas para configurar su funcionamiento. Está conectada a todos los servicios que ofrece pivotando a través del directorio LDAP: Control de Versiones, Directorios, Integración Continua, ITS, Revisiones, Gestión de Dependencias, Seguridad y Autenticación que analizaremos extensamente más adelante componente a componente.

% subsection api (end)

% section arquitectura (end)

%%%%%%%%%%%%%%%%%%%%%%%%%%%%%%%%%%%%%%%%%%%%%%%%%%%%%%%%%%%%%%%%%%%%%%%%%%%%%%%%%%%%%%%%%%%%%%%%%%%%%%%%%%
% Aprovisionamiento (Puppet)
%%%%%%%%%%%%%%%%%%%%%%%%%%%%%%%%%%%%%%%%%%%%%%%%%%%%%%%%%%%%%%%%%%%%%%%%%%%%%%%%%%%%%%%%%%%%%%%%%%%%%%%%%%

\section{Aprovisionamiento (Puppet)}
\label{sec:puppet}

\par \emph{Aprovisionamiento}: 'Accción o efecto de aprovisionar', \emph{aprovisionar}: abastecer\footnote{Definición del diccionario de la RAE - \url{http://buscon.rae.es/drae/?type=3&val=aprovisionar&val_aux=&origen=REDRAE}}. Aplicado a la Ingeniería del Software el Aprovisionamiento nos provee de los componentes necesarias para construir una solución.

\par El aprovisionamiento trata de la automatización de tareas para construir el entorno deseado, en este caso la instalación de SidelabCode Stack. La evolución en el modelo de instalación para facilitar la ejecución de módulos, configuración y comunicación entre los distintos componentes.

\par La herramienta empleada para el aprovisionamiento es \textbf{Puppet} de la compañía \emph{PuppetLabs}\footnote{\url{https://puppetlabs.com/}}.

\begin{figure}[H]
    \centering
    \includegraphics[width=0.3\textwidth]{puppet-labs-logo}
    \caption{Puppet Labs Logo}
    \label{fig:puppet-labs}
\end{figure}

\par \emph{Puppet}: es una herramienta \emph{Software Libre}\footnote{\url{https://puppetlabs.com/puppet/puppet-open-source/}} de aprovisionamiento desarrollada en \emph{Ruby}\footnote{\url{https://github.com/puppetlabs/puppet}}. Gestiona la infraestructura a través de su ciclo de vida, desde el aprovisionamiento y la configuración de parches automatizando la ejecución de órdenes para la instalación y configuración del entorno.

\begin{itemize}
	\item Automatizar tareas repetitivas.
	\item Desplegar rápidamente aplicaciones críticas.
	\item Gestionar proactivamente el cambio.
	\item Escalar de 10 servidores para 1000.
	\item Instalaciones locales o en la nube.
\end{itemize}

\par \emph{Puppet} utiliza un enfoque declarativo, basado en el modelo de automatización:
\begin{itemize}
	\item \emph{Definir} el estado deseado de la configuración de la infraestructura mediante lenguaje de configuración declarativa de Puppet.
	\item \emph{Simular} los cambios de configuración antes de la ejecución.
	\item \emph{Corroborar} el estado final mediante despliegues automáticos comprobando las posibles desviaciones en la configuración.
	\item \emph{Informe} sobre las diferencias entre los estados reales y deseados y cualquier cambio que haya hecho cumplir el estado deseado.
\end{itemize}

\begin{figure}[H]
    \centering
    \includegraphics[width=0.7\textwidth]{howpuppetworks}
    \caption{Ciclo de vida de los módulos Puppet}
    \label{fig:howpuppetworks}
\end{figure}

\par El diseño del aprovisionamiento a través de Puppet se basa en módulos. Cada uno de estos módulos se encarga de gestionar la instalación y configuración del componente definido.

\par Que aporta Puppet ? Proporciona un control sobre la instalación de cada herramienta y la configuración asociada por defecto que se define. De esta forma el proceso se instalación se automatiza permitiendo la replicación del mismo, completo o por módulos en diferentes entornos, locales o virtuales. La instalación a partir de módulos ha de definir una cadena de dependencias entre cada uno de ellos de manera que se vayan habilitando funcionalidades e interacciones (Apéndice ~\ref{app:apendice-puppet}).

\par En SidelabCode Stack Puppet es el encargado de gestionar la \emph{instalación y configuración} de cada uno de los componentes mediante módulos Puppet independientes para completar el proceso de instalación (Apéndice ~\ref{app:instalacion-sidelab}) entre 5 y 10 minutos.

%%%%%%%%%%%%%%%%%%%%%%%%%%%%%%%%%%%%%%%%%%%%%%%%%%%%%%%%%%%%%%%%%%%%%%%%%%%%%%%%%%%%%%%%%%%%%%%%%%%%%%%%%%
% Componentes
%%%%%%%%%%%%%%%%%%%%%%%%%%%%%%%%%%%%%%%%%%%%%%%%%%%%%%%%%%%%%%%%%%%%%%%%%%%%%%%%%%%%%%%%%%%%%%%%%%%%%%%%%%

\section{Componentes}
\label{sec:componentes}

\par En este apartado se van a analizar los distintos componentes que integran la Forja SidelabCode Stack y el papel que desempeñan en el funcionamiento del sistema partiendo del esquema en donde se refleja la arquitectura de la Forja SidelabCode ~\ref{fig:scstack-diagram} y la relación que mantienen con las necesidades descritas en el capítulo \nameref{chap:procesos-desarrollo} a la hora de gestionar el proceso \emph{Iterativo e Incremental}.

%%%%%%%%%%%%%%%%%%%%%%%%%%%%%%%%%%%%%%%%%%%%%%%%%%%%%%%%%%%%%%%%%%%%%%%%%%%%%%%%%%%%%%%%%%%%%%%%%%%%%%%%%%
% OpenLDAP
%%%%%%%%%%%%%%%%%%%%%%%%%%%%%%%%%%%%%%%%%%%%%%%%%%%%%%%%%%%%%%%%%%%%%%%%%%%%%%%%%%%%%%%%%%%%%%%%%%%%%%%%%%

\subsection{Usuarios, roles y grupos}
\label{sub:usuarios-roles-grupos}

\par Gestión de usuarios, roles, grupos y proyectos a través de \emph{OpenLDAP}\footnote{OpenLDAP - \url{http://www.openldap.org/}}.

\begin{figure}[H]
    \centering
    \includegraphics[width=0.3\textwidth]{OpenLDAP-logo}
    \caption{OpenLDAP logo}
    \label{fig:openldap-logo}
\end{figure}

\par SCStack utiliza la tecnología de directorios \emph{LDAP} como sistema de autenticación y de información centralizado de la Forja Software. En dicho servidor de directorios se almacena información relativa a todos los usuarios, proyectos software y repositorios de la Forja. Esta tecnología, además, cuenta con la ventaja de que la mayoría de aplicaciones web con sistemas de autenticación ofrecen interfaces que garantizan una completa integración con directorios LDAP, este es el caso de Redmine, Drupal, Wordpress y el propio servidor web Apache. La implementación de este protocolo en la Forja Sidelab se lleva a cabo mediante un servidor de directorios muy estable y de libre distribución que es OpenLDAP.

\subsection{API OpenLDAP}
\label{sub:api-openldap}

\par Debido a que el directorio LDAP es la estructura de información centralizada de la Forja, cualquier tipo de acción que se quiera realizar sobre el sistema requerirá el acceso por parte de la API a este servicio de directorios, bien para la recuperación de datos en las consultas o para la manipulación de registros a la hora de crear, editar o borrar usuarios o proyectos.

\par Todas las acciones de la Consola de Administración pasan a través de la API que comunica con OpenLDAP dando acceso y generando las distintas autenticaciones en cada una de las herramientas interconectadas basándose en los registros de OpenLDAP como generador y autenticador de credenciales.

% subsection api-openldap (end)

% subsection usuarios-roles-grupos (end)

%%%%%%%%%%%%%%%%%%%%%%%%%%%%%%%%%%%%%%%%%%%%%%%%%%%%%%%%%%%%%%%%%%%%%%%%%%%%%%%%%%%%%%%%%%%%%%%%%%%%%%%%%%
% Gestión de Requisitos - Redmine
%%%%%%%%%%%%%%%%%%%%%%%%%%%%%%%%%%%%%%%%%%%%%%%%%%%%%%%%%%%%%%%%%%%%%%%%%%%%%%%%%%%%%%%%%%%%%%%%%%%%%%%%%%

\subsection{Gestión de Requisitos ITS}
\label{sub:its}

\par La gestión de requisitos en SidelabCode Stack se lleva a cabo a través del Issue Tracking System \emph{Redmine}\footnote{Redmine - \url{http://www.redmine.org/}}.

\begin{figure}[H]
    \centering
    \includegraphics[width=0.3\textwidth]{redmine}
    \caption{Redmine logo}
    \label{fig:redmine-logo}
\end{figure}

\par Este apartado es el más cuidado e importante en el conjunto de SCStack ya que se trata de la herramienta encargada de centralizar la gestión de \emph{Lista de Control del Proyecto} por cada Iteración. El ITS Redmine es una aplicación web de gestión de proyectos Software multiplataforma desarrollada en \emph{Ruby} a través de \emph{Ruby on Rails}. Por supuesto se trata de una herramienta FLOSS.

\par Redmine proporciona la gestión de tareas (de cualquier tipo; \emph{features, bugs, parches\ldots}) para cada uno de los proyectos creados. Dentro de SCStack se encuentra enlazado a la configuración de usuarios a través de OpenLDAP para la validación y cuenta con una base de datos propia \emph{MySQL}. Un apartado importante ya que esto permite gestionar las migraciones de Redmine de manera rápida, eficiente y fluida de una versión a otra o incluir la información de otro Redmine\footnote{Upgrading Redmine - \url{http://www.redmine.org/projects/redmine/wiki/RedmineUpgrade}} en una nueva instalación de la Forja, únicamente haciendo un backup de la base de datos y algunos directorios clave. Incluso la migración a Redmine desde otros gestores de tareas\footnote{Migrate to Redmine - \url{http://www.redmine.org/projects/redmine/wiki/RedmineMigrate}}.

\par Proporciona una interfaz adaptada para cada proyecto de la forja asociado a un repositorio de código fuente con un visor integrado. Se visualizan los cambios entre distintas versiones del código y comparaciones entre distintas ramas de desarrollo para seguir la evolución del proyecto.

\begin{figure}[H]
    \centering
    \includegraphics[width=0.7\textwidth]{redmine-demo-landing-page}
    \caption{Redmine página de bienvenida}
    \label{fig:redmine-demo-landing-page}
\end{figure}

\par Dentro del proceso Iterativo e Incremental como herramientas nos proporciona informes de seguimiento de actividad dinámicos (\emph{día, persona, proyecto\ldots}), un \emph{Wiki} para la documentación y una página de noticias.

\par Es la interfaz web de entrada al proyecto para los desarrolladores, por ello es la parte más importante y en donde se centran gran parte de los esfuerzos para la fluidez y la importancia que tiene la misma.

\par Por otra parte Redmine proporciona una API Rest para facilitar la interoperabilidad entre los distintos componentes. Esta API se maneja a través de la capa de negocio de la Consola de Administración de SCStack para así a través del API se registren en Redmine los usuarios, proyectos y grupos con sus respectivos permisos partiendo de los valores introducidos a través de la Consola de Administración. De esta forma, Redmine pasa a gestionar a través de su BBDD MySQL los usuarios y proyectos después que se hayan autenticado a través de OpenLDAP para así cargar la configuración obtenida de su BBDD.

\par Las operaciones de gestión administrativa en Redmine permanecen desactivadas, de modo que ningún administrador de proyectos puede añadir ni borrar miembros de sus proyectos, tampoco pueden crearse ni borrarse usuarios o proyectos, etc. \emph{Obligando} así que todas las operaciones de gestión de la Forja se lleven a cabo desde el Software diseñado específicamente para ello, la Consola de Administración, garantizando así la consistencia.

\subsection{Plugins}
\label{sub:redmine-plugins}

\par A la hora de aplicar el desarrollo Iterativo e Incremental se intenta incrementar la eficacia, la visibilidad, la rapidez y la comprensión del proceso de desarrollo. Por eso se han utilizado distintos plugins para Redmine que ayudan a la comprensión y agilizan el proceso Iterativo e Incremental para los usuarios (desarrolladores, gestores, etc\ldots) a través del plugin FLOSS \emph{Backlogs}\footnote{Redmine Backlogs - \url{http://www.redminebacklogs.net/}}.

\par \emph{Backlogs} aporta una gestión visual en modo tablón de las Iteraciones activas en el proceso. Gestiona 'manualmente' a través de \emph{Drag and Drop} la organización de las tareas del proyecto y agiliza la comprensión del estado de la iteración, proyecto, desarrollo en sí, en un sólo vistazo ~\ref{fig:backlogs-plugin}.

\begin{figure}[H]
    \centering
    \includegraphics[width=0.7\textwidth]{backlogs-plugin}
    \caption{Redmine Backlogs plugin: tareas Redmine agrupadas por estados dentro de una historia de usuario.}
    \label{fig:backlogs-plugin}
\end{figure}

\par Otro apartado importante en el desarrollo Iterativo e Incremental es la gestión de la documentación, en este caso Redmine nos proporciona una Wiki por cada proyecto. Una Wiki es una herramienta para la documentación colaborativa que mantiene su histórico de cambios (asociados a usuario y tiempo) y que mediante el lenguaje de marcado wiki desde el cual se exportan a distintos formatos como: \emph{html, pdf, etc}, a un coste de recursos bajo ya que se trata de texto plano. El caso más famoso del uso de Wikis como documentación colaborativa es la \emph{Wikipedia}\footnote{Wikipedia - \url{http://www.wikipedia.org}}.

% subsection redmine-plugins (end)

% subsection its (end)

%%%%%%%%%%%%%%%%%%%%%%%%%%%%%%%%%%%%%%%%%%%%%%%%%%%%%%%%%%%%%%%%%%%%%%%%%%%%%%%%%%%%%%%%%%%%%%%%%%%%%%%%%%
% Repositorios de Código
%%%%%%%%%%%%%%%%%%%%%%%%%%%%%%%%%%%%%%%%%%%%%%%%%%%%%%%%%%%%%%%%%%%%%%%%%%%%%%%%%%%%%%%%%%%%%%%%%%%%%%%%%%

\subsection{Repositorios de Código}
\label{sub:repositorios}

\par Los repositorios de código (VCS) son los encargados de gestionar el ciclo de vida del código fuente como hemos visto en el apartado de Código versionado~\ref{sec:codigo-versionado} y existen dos tipos: centralizados y distribuidos. Para el proceso de desarrollo Iterativo e Incremental hemos seleccionado el repositorio distribuido \emph{Git}\footnote{Git - \url{http://git-scm.com/}}.

\begin{figure}[H]
    \centering
    \includegraphics[width=0.3\textwidth]{git-logo}
    \caption{Git SCM Logo}
    \label{fig:git-scm-logo}
\end{figure}

\par Atendiendo a los requerimientos del proceso:

\begin{quote}
    \emph{El desarrollo de la solución se adapta al uso del Repositorio distribuido, ya que éste aporta una flexibilidad para la gestión de bifurcaciones del código que en un repositorio centralizado no tenemos fácilmente}
\end{quote}

\par En el proceso Iterativo e Incremental abundan las ramificaciones del código en el repositorio debido a ello se opta por la elección del repositorio distribuido Git en pro del repositorio centralizado SVN, que también se incluye en la Forja SCStack. Las ramificaciones, la cantidad de fusiones entre ramas~\cite{featurebranch}, entornos de desarrollo sumando la facilidad, agilidad y el bajo coste en Git nos proporcionan la herramienta adecuada para la gestión del código fuente en SCStack.

\par En el repositorio Git no se guardan diferencias se guardan snapshots en comparación con Subversion.

\begin{figure}[H]
    \centering
    \includegraphics[width=0.7\textwidth]{svn-git-comparison}
    \caption{Comparación de incrementos entre SVN (completo) y Git (incrementos en archivos)}
    \label{fig:svn-git-comparison}
\end{figure}

\begin{quotation}
    \emph{For example the Mozilla repository is reported to be almost 12 Gb when stored in SVN using the fsfs backend. Previously, the fsfs backend also required over 240,000 files in one directory to record all 240,000 commits made over the 10 year project history. This was fixed in SVN 1.5, where every 1000 revisions are placed in a separate directory. The exact same history is stored in Git by only two files totaling just over 420 Mb. This means that SVN requires 30x the disk space to store the same history}\footnote{Git Svn Comparison Smaller Space Requirements - \url{https://git.wiki.kernel.org/index.php/GitSvnComparison\#Smaller\_Space\_Requirements}}
\end{quotation}

\par Asociada a cada Iteración se ha de gestionar la viabilidad de una rama del repositorio. En esta rama se trabaja con respecto a las tareas a desarrollar para la iteración. Se hacen las pruebas necesarias para cada desarrollo para después integrar la solución en la rama principal y así crear una nueva versión del proyecto. 

\par La ramificación del desarrollo permite gestionar por iteraciones incrementales aisladas en una nueva rama, de esta forma, el contenido de la rama principal es fiable, ya que para añadir contenido a la rama ha tenido que pasar una iteración nueva y el proceso de desarrollo que conlleva; planificación, test, implementación e integración. El resultado está altamente controlado y distribuido en base a módulos para poder acotar posibles errores posteriores o revertir cambios a través de un proceso minuciosamente controlado.

\subsubsection{Integridad}
\label{subs:git-integridad}

\par Los commits se identifican por un \texttt{hash sha1}

\begin{itemize}
    \item \texttt{Svn}: \emph{rev 33}
    \item \texttt{Git}: \emph{d025a7b3217f05110ebbf48065b8d02a0ad22ae3}
\end{itemize}

\par O más amigablemente: \emph{d025a7b}

\par Los ficheros también se identifican por su sha1 de esta forma si un fichero se corrompe durante la transmisión por la red se detecta inmediatamente

\subsubsection{Los 3 estados}
\label{subs:git-3-estados}

\par Los ficheros en Git pueden estar en tres estados:

\begin{itemize}
    \item \texttt{Modificado}: el fichero ha cambiado desde el último checkout
    \item \texttt{Staged}: un fichero modificado ha sido marcado para ser añadido en el próximo commit
    \item \texttt{Committed}: el fichero se encuentra en la base de datos de git
\end{itemize}

\par Hay un 4º estado: \textbf{untracked}.

\subsubsection{Las 3 áreas de un proyecto git}
\label{subs:git-3-areas}

\begin{enumerate}
    \item El directorio git (git directory):
        \begin{itemize}
            \item Contiene los metadatos y la base de datos de git
            \item Es lo que se copia cuando se clona un repositorio
            \item Normalmente es una carpeta .git en algún directorio
        \end{itemize}
    \item La carpeta de trabajo (working directory):
        \begin{itemize}
            \item Es un checkout de una versión específica del proyecto
            \item Se extrae del directorio git
            \item Es el espacio donde modificamos los ficheros
        \end{itemize}
    \item Staging area:
        \begin{itemize}
            \item Fichero en el directorio git que indica qué cambios van en el próximo commit
        \end{itemize}
\end{enumerate}

\begin{figure}[H]
    \centering
    \includegraphics[width=0.7\textwidth]{git-3-areas}
    \caption{Tres áreas en Git}
    \label{fig:git-3-areas}
\end{figure}

\subsubsection{La identidad}
\label{subs:git-identidad}

\par Git necesita conocer algunos datos del desarrollador (aparecen en los commits para identificar al autor)

\begin{itemize}
    \item Nombre
    \item Email
\end{itemize}

\par Si no están correctamente configurados pueden aparecer varios problemas:

\begin{itemize}
    \item Los commits \textbf{fallan} porque el usuario no está autorizado
    \item Commits del mismo usuario \emph{'físico'} \textbf{no son considerados como del mismo usuario} porque el nombre \emph{'lógico'} cambia.
\end{itemize}

\par Por lo que se ha de tener muy en cuenta este apartado y configurar correctamente en el archivo.

% subsection repositorios (end)

\subsection{Revisión de Código}
\label{sub:gerrit}

\par La gestión del repositorio Git está orientada a través de la herramienta \emph{Gerrit}\footnote{\url{http://code.google.com/p/gerrit/}}.

\begin{figure}[H]
    \centering
    \includegraphics[width=0.3\textwidth]{gerrit-diffy}
    \caption{Gerrit Kunfu Review Cuckoo}
    \label{fig:gerrit-logo}
\end{figure}

\par Gerrit es una herramienta de revisión de código basada en web. Facilita el control online de las revisiones de código para los proyectos a través de la herramienta Git.

\par Gestiona la autenticación y la gestión de permisos, roles y grupos para cada uno de los repositorios existentes a través de una interfaz ligera. Un punto a destacar es la orientación hacia un repositorio centralizado de Git, es decir, el control de código se centraliza en un repositorio para la integración dependiente de los repositorios de los desarrolladores\footnote{Git Flujo de trabajo centralizado - \url{http://git-scm.com/book/es/Git-en-entornos-distribuidos-Flujos-de-trabajo-distribuidos\#Flujo-de-trabajo-centralizado}}.

\begin{figure}[H]
    \centering
    \includegraphics[width=0.7\textwidth]{git-centralizado}
    \caption{Git: Flujo de trabajo centralizado}
    \label{fig:git-centralizado}
\end{figure}

\par Centralizar la rama principal del proyecto en el repositorio descentralizado casa perfectamente con el proceso Iterativo e Incremental ya que a cada iteración cerrada añadimos los cambios a la rama estable del proyecto desarrollado centralizando el código a empaquetar para convertir en la solución final, incremento a incremento.

\par La configuración viene a través del API de SCStack y una consola interactiva SSH\footnote{Gerrit Command Line Tools - \url{http://gerrit.googlecode.com/svn/documentation/2.0.34/cmd-index.html}} que proporciona Gerrit. En SCStack esta función recae sobre los valores obtenidos del servidor \emph{OpenLDAP} a partir de los datos introducidos en la Consola de Administración a la hora de gestionar la creación de un proyecto y el grupo de usuarios asociados al proyecto siguiendo el flujo de datos centralizado a través de la API Rest central. La creación del proyecto, la gestión de usuarios y permisos sobre el repositorio Git creado.

\par Este procedimiento se encapsula a través del envío de órdenes a la consola a través del API habiendo configurado la autenticación a través de un conjunto de clave SSH en la instalación de SCStack.

% subsection gerrit (end)

%%%%%%%%%%%%%%%%%%%%%%%%%%%%%%%%%%%%%%%%%%%%%%%%%%%%%%%%%%%%%%%%%%%%%%%%%%%%%%%%%%%%%%%%%%%%%%%%%%%%%%%%%%
% Integración Continua
%%%%%%%%%%%%%%%%%%%%%%%%%%%%%%%%%%%%%%%%%%%%%%%%%%%%%%%%%%%%%%%%%%%%%%%%%%%%%%%%%%%%%%%%%%%%%%%%%%%%%%%%%%

\subsection{Integración Continua}
\label{sub:ci-jenkins}

\par  La integración continua dentro del proceso de desarrollo Iterativo e Incremental es la encargada de agilizar la comunicación entre los distintos actores involucrados. Evalúa la evolución de la solución y provee una respuesta firme y profesional para afianzar los resultados de cada incremento.

\begin{figure}[H]
    \centering
    \includegraphics[width=0.3\textwidth]{jenkins}
    \caption{Jenkins CI Logo}
    \label{fig:jenkins-logo}
\end{figure}


\par En este aspecto, la herramienta que SCStack necesita es un gestor de integración continua como \emph{Jenkins-CI}\footnote{Jenkins-CI - \url{http://jenkins-ci.org}}:

\begin{quote}
    \emph{In a nutshell Jenkins CI is the leading open-source continuous integration server. Built with Java, it provides over 400 plugins to support building and testing virtually any project.}
\end{quote}

\par Jenkins-CI\footnote{Meet Jenkins-CI - \url{https://wiki.jenkins-ci.org/display/JENKINS/Meet+Jenkins}} proporciona la integración de la integración continua (valga la redundancia) en el proceso de desarrollo de software de una forma modular e interoperable. No se trata de un servidor intrusivo ni para el proceso de desarrollo ni para el desarrollador en si.

\par En este aspecto SCStack incluye el módulo de Jenkins-CI en la instalación para facilitar el trabajo de las pruebas, integración y generación de versiones mediante la virtualización de escenarios más cercanos al sistema de producción. Des esta forma se perfila mejor el rendimiento, la replicación de errores y el control de la evolución del código fuente. Dentro del desarrollo, el código fuente es el centro de la calidad y debido a esta afirmación, se necesita un servidor de CI adaptado al proceso de desarrollo.

\begin{itemize}
	\item Tags.
	\item Construir las versiones vivas.
	\item Branches de releases.
	\item Desplegar una versión específica con un clic.
\end{itemize}

\par Jenkins-CI se instala integrado en el marco de trabajo SCStack interoperando entre el repositorio de código fuente Git, la herramienta de gestión Gerrit y el despliegue de los tests en servidores como Apache o Tomcat dependiendo del producto final (esto no tiene nada que ver con el desarrollo, sería un requisito del proyecto). Jenkins-CI mantiene diferentes versiones vivas a la vez en el repositorio del proyecto para cumplir con los objetivos:

\begin{itemize}
	\item Asegurar la calidad.
	\item Hacer el despliegue ágil.
	\item Minimizar el riesgo.
\end{itemize}

\par Jenkins-CI trabaja en base a \emph{Jobs}. Un Job en Jenkins-CI describe el proceso de integración, pruebas o despliegue que va a ejecutarse cuando se cumpla un requisito definido o manualmente. La integración de las distintas ramas de desarrollo en cada iteración en el ciclo de vida~\ref{fig:simple-lifecycle} las gestiona Jenkins-CI a través de los Jobs definidos:

\begin{itemize}
	\item Jobs de \emph{integración}.
	    \begin{itemize}
        	\item Descargan el código del repositorio Git.
        	\item Construyen.
        	\item Pasan tests.
        	\item Despliegan la versión construida en "local".
        \end{itemize}
	\item Jobs de \emph{release}.
	    \begin{itemize}
        	\item Realizan los pasos anteriores y además.
        	\item Tag si los tests pasaron.
        	\item Push del tag al repositorio remoto.
        \end{itemize}
	\item Jobs de \emph{despliegue}.
	    \begin{itemize}
        	\item Descargan el binario del repositorio de binarios
        	\item Desplegar en un servidor de aplicaciones.
        \end{itemize}
\end{itemize}

\begin{figure}[H]
    \centering
    \includegraphics[width=0.7\textwidth]{simple-lifecycle}
    \caption{Ciclo de vida Jenkins CI}
    \label{fig:simple-lifecycle}
\end{figure}

\par Dentro de la forja SCStack se configura el uso de Jenkins-CI a partir de usuarios dedicados a la gestión de jobs. Se plantean tres opciones:

\begin{itemize}
	\item \textbf{Opción I}: Un usuario \emph{jenkinsci} con acceso de \emph{lectura/escritura} a \emph{todos} los repositorios.
        \begin{itemize}

        	\item \emph{Pros}; Simplicidad: sólo hay que gestionar un usuario. Una \emph{única} clave ssh \emph{/home/tomcat/.ssh/id\_rsa.pub}.
        	\item \emph{Contras}; Un usuario para dominarlos a todos. Cualquier error en un job para el proyecto \emph{X} puede afectar a los repositorios del proyecto \emph{Y}.
        \end{itemize}

	\item \textbf{Opción II}: Un usuario @jenkinsci@ con acceso de \emph{lectura/escritura} a todos los repositorios y  otro \emph{jenkinsci\_read} con acceso sólo de lectura	
        \begin{itemize}
        	\item \emph{Pros}; Sólo hay que gestionar \emph{dos usuarios}: basta con asociarlos a \emph{todos} los proyectos.
        	\item \emph{Contras}; Seguimos teniendo un usuario para dominarlos a todo: jenkinsci. Cualquier error en un job para el proyecto \emph{X} puede afectar a los repositorios del proyecto \emph{Y}. Hay que gestionar dos claves ssh. No es trivial.
        \end{itemize}

	\item \textbf{Opción III}: Un \emph{usuario por proyecto} para integración continua.
        \begin{itemize}
        	\item \emph{Pros}; El usuario de ci de un proyecto no tiene acceso a los repositorios de otro proyecto.
        	\item \emph{Contras}; Hay que gestionar múltiples \emph{claves ssh}.
        \end{itemize}

\end{itemize}

\par Se decide utilizar un usuario por proyecto definida en la \textbf{tercera opción}, después de haber evaluado las distintas opciones.
 
\par La gestión de los usuarios se gestiona a través de la consola de administración de SCStack. De esta manera la herramienta nos aporta una vía única de entrada dotando de la funcionalidad específica para distintos estados del proceso de desarrollo nada intrusivos en el desarrollo del proyecto. Separando a través del API de SCStack la gestión y partiendo de los roles/usuarios/grupos definidos como base en OpenLDAP.

\par El repositorio remoto debería contener versiones más o menos estables y distinguidas para el desarrollo por ramas establecido el proceso:

\begin{itemize}
	\item \textbf{Develop}: Pueden fallar algunos tests, no es problema.

	\item \textbf{Release-X}: Los tests deberían pasar, eventualmente se podrían desactivar.
	
	\item \textbf{Nightly builds}: Construcciones que comprueban la \emph{'salud'} del proyecto. Se hacen sobre los branches \emph{development} y \emph{release-X} (siendo X la versión más reciente).

    \begin{itemize}
	    \item Se ejecutan los tests.
	    \item Se despliegan dos versiones por cada rama: \texttt{Limpia}: una bbdd nueva y \texttt{Migración}: con actualización de bbdd ya existente sobre la bbdd que ya hubiera para ese despliegue.
    \end{itemize}

	\item \textbf{Preproducción}: Las versiones que van a preproducción son aquellas de las que se ha hecho \texttt{tag}: 
	\begin{itemize}
	    \item Pasan los tests.
	    \item El entorno de pre puede estar en \emph{local} o en \emph{Remoto}.
	    \item Se despliegan dos versiones: \texttt{Limpia} y \texttt{Migración}.
    \end{itemize}

	\item \textbf{Producción}: Las versiones que van a producción son aquellas de las que se ha hecho \texttt{tag}.
	\begin{itemize}
	    \item Pasan los tests.
	    \item El despliegue se realiza en \emph{Remoto}.
	    \item Se despliegan dos versiones: \texttt{Limpia} y \texttt{Migración}.
    \end{itemize}

\end{itemize}

\begin{figure}[H]
    \centering
    \includegraphics[width=0.7\textwidth]{diagrama-integracion-continua}
    \caption{Integración Continua de versiones a partir de distintas ramas}
    \label{fig:diagrama-integracion-continua}
\end{figure}

% subsection ci-jenkins (end)

%%%%%%%%%%%%%%%%%%%%%%%%%%%%%%%%%%%%%%%%%%%%%%%%%%%%%%%%%%%%%%%%%%%%%%%%%%%%%%%%%%%%%%%%%%%%%%%%%%%%%%%%%%
% Gestión de distribuciones y dependencias
%%%%%%%%%%%%%%%%%%%%%%%%%%%%%%%%%%%%%%%%%%%%%%%%%%%%%%%%%%%%%%%%%%%%%%%%%%%%%%%%%%%%%%%%%%%%%%%%%%%%%%%%%%

\subsection{Gestión de distribuciones y dependencias}
\label{sub:distribuciones-dependencias}

\par Gestión de recursos compartidos, comunicación segura a través de \emph{OpenSSH} y gestión de dependencias y distribuciones a través de \emph{Archiva} como repositorio \emph{Maven}.

\subsubsection{OpenSSH}
\label{ssub:openssh}

\par \emph{OpenSSH}\footnote{\url{http://www.openssh.org/}} es un conjunto de aplicaciones de libre distribución que permiten realizar comunicaciones cifradas a través de la red usando el protocolo SSH.

\begin{figure}[H]
    \centering
    \includegraphics[width=0.3\textwidth]{openssh_logo}
    \caption{OpenSSH logo}
    \label{fig:openssh_logo}
\end{figure}

\par La función fundamental que brinda este protocolo a la Forja Software consiste en garantizar que los usuarios de los distintos proyectos puedan acceder, por ejemplo a través de SFTP\footnote{\url{http://en.wikipedia.org/wiki/SFTP}} o Secure Shell \emph{SSH}\footnote{\url{http://en.wikipedia.org/wiki/Secure\_Shell}}, a las carpetas privadas y públicas de los proyectos en los cuales participan, a través de una conexión cifrada, y gestionar sus ficheros de una forma sencilla, ágil y segura.

% subsubsection openssh (end)

\subsubsection{Archiva}
\label{ssub:archiva}

\par \emph{Archiva}\footnote{\url{http://archiva.apache.org/index.cgi}} es una aplicación FLOSS para la gestión de repositorios de artefactos de construcción, desarrollado por \emph{Apache Software Foundation}. Este el compañero perfecto para herramientas de construcción como Maven.

\begin{figure}[H]
    \centering
    \includegraphics[width=0.3\textwidth]{apache-archiva-logo}
    \caption{Apache Archiva logo}
    \label{fi:apache-archiva}
\end{figure}

\par Entre las funcionalidades que podemos destacar de esta herramienta están, el proxy para repositorio remoto (reduce el tráfico de red en equipos de trabajo), la gestión del control de acceso a los repositorios definidos, la construcción de artefactos de almacenamiento y, la gestión y mantenimiento de repositorios Maven facilitando la indexación, búsquedas, informes, estadísticas.

\par Posee una interfaz web sencilla de configurar en la que publica el contenido del repositorio Maven accesible a los usuarios autenticados. Además de el acceso al repositorio Maven habiendo configurado en los proyectos de desarrollo la url en la que está publicado:

\lstset{style=xmlbasico}
\begin{lstlisting}[frame=trbl]
<project>
  <!-- omitted xml -->
  <distributionManagement>
    <repository>
      <id>archiva.internal</id>
      <name>Internal Release Repository</name>
      <url>http://reposerver.mycompany.com:8080/archiva/repository/internal/</url>
    </repository>
    <snapshotRepository>
      <id>archiva.snapshots</id>
      <name>Internal Snapshot Repository</name>
      <url>http://reposerver.mycompany.com:8080/archiva/repository/snapshots/</url>
    </snapshotRepository>
  </distributionManagement>
  <!-- omitted xml -->
</project>
\end{lstlisting}

\par El uso de Archiva como gestor de dependencias Maven puede ser público y/o privado dependiendo del repositorio que se cree. De esta forma los equipos de desarrolladores automatizan y centralizan las descargas de las dependencias (librerías) a través de Archiva dentro de la red local, disminuyendo el tráfico de red y agilizando la puesta en marcha de proyectos y los jobs de CI, reduciendo el consumo.

% subsubsection archiva (end)

% subsection distribuciones-dependencias (end)

% subsection componentes (end)

%%%%%%%%%%%%%%%%%%%%%%%%%%%%%%%%%%%%%%%%%%%%%%%%%%%%%%%%%%%%%%%%%%%%%%%%%%%%%%%%%%%%%%%%%%%%%%%%%%%%%%%%%%
% Desarrollo Eclipse y Maven
%%%%%%%%%%%%%%%%%%%%%%%%%%%%%%%%%%%%%%%%%%%%%%%%%%%%%%%%%%%%%%%%%%%%%%%%%%%%%%%%%%%%%%%%%%%%%%%%%%%%%%%%%%

\subsection{Desarrollador: Eclipse y Maven}
\label{sub:eclipse-mvn-tdd}

\par El entorno de desarrollo para el desarrollador en el que se integra el proceso Iterativo e Incremental consta de dos herramientas y un concepto.

\par El IDE \emph{Integrated Development Environment} a utilizar es Eclipse, concretamente la distribución del framework Spring \emph{Eclipse STS}\footnote{Eclipse STS - \url{http://www.springsource.org/sts}}. 

\begin{figure}[H]
    \centering
    \includegraphics[width=0.3\textwidth]{STS}
    \caption{Spring Tool Suite logo}
    \label{fig:sts}
\end{figure}

\par STS otorga y proporciona un conjunto de herramientas preparadas para trabajar con los componentes de STStack: Redmine, Git, Gerrit, Jenkins, Apache, OpenLDAP, Maven. Se basa en unan distribución de Eclipse orientada al desarrollo para trabajar con diferentes componentes externos facilitando la puesta a punto para empezar a desarrollar.

\par Esta distribución incluye la gestión de un proyecto a través \emph{Maven}\footnote{Apache Maven - \url{http://maven.apache.org/}}. Maven es un Software de gestión de proyectos basado en la definición del proyecto POM (Project Object Model). Es el encargado de compilar, ejecutar test, desplegar, gestionar las dependencias (locales y remotas) y generar distribuciones a partir de una configuración definida.

\begin{figure}[H]
    \centering
    \includegraphics[width=0.3\textwidth]{maven}
    \caption{Maven logo}
    \label{fig:maven-logo}
\end{figure}

\par Esta herramienta se integra en el ciclo de vida del Software para automatizar los procesos tediosos manuales, proveer una estructura básica inicial a los proyectos (iteración 0) para ir incrementando iteración tras iteración a través de los arquetipos.

\par Incorpora componentes en la instalación por defecto para facilitar el trabajo del desarrollador con la forja SCSTack:

\begin{itemize}
	\item \emph{Egit} - Plugin de Eclipse para trabajar con repositorios Git - \url{http://www.eclipse.org/egit/}.
	\item \emph{Mylyn} - Plugin para gestionar las tareas del ITS, en este caso Redmine, a través de la interfaz de Eclipse. Automatizando las pruebas, grabando sesiones de trabajo e incluso compartirlas para poder reproducir un procedimiento en otro sistema - \url{http://www.eclipse.org/mylyn/}
	\item \emph{Maven} - Plugin \emph{2eclipse} para la gestión de proyectos Maven a través de su ciclo de vida. Ayuda a facilitar la configuración de los builds a través de los pom de Maven. Genera esqueletos de proyectos a través de los arquetipos de una forma intuitiva - \url{http://www.sonatype.org/m2eclipse}
\end{itemize}

\par A través de Maven y Eclipse se definen las líneas del desarrollo de cada iteración facilitando el diseño TDD~\ref{sub:tdd} a través de una sencilla configuración. El esqueleto de los proyectos Maven proporciona unas sencillas guías estándar para el desarrollo de proyectos de Software. Esta agrupación de funcionalidades se reutiliza en Jenkins-CI a través de los \emph{jobs} de Maven para así poder replicar cualquier entorno durante la integración continua partiendo de una base robusta.

% subsection eclipse-mvn-tdd (end)

%%%%%%%%%%%%%%%%%%%%%%%%%%%%%%%%%%%%%%%%%%%%%%%%%%%%%%%%%%%%%%%%%%%%%%%%%%%%%%%%%%%%%%%%%%%%%%%%%%%%%%%%%%
% Pruebas y Validación
%%%%%%%%%%%%%%%%%%%%%%%%%%%%%%%%%%%%%%%%%%%%%%%%%%%%%%%%%%%%%%%%%%%%%%%%%%%%%%%%%%%%%%%%%%%%%%%%%%%%%%%%%%

\section{Pruebas y Validación}
\label{sec:pruebas-validacion}

\par Todo proyecto de software necesita ser probado para comprobar que funcione correctamente si existen fallos para así poder atajarlos sin que las sorpresas aparezcan.

\par El lenguaje utilizado para el desarrollo del API del proyecto ha sido \emph{Java}.

\begin{figure}[H]
    \centering
    \includegraphics[width=0.5\textwidth]{sidelabcodestack-api-lib}
    \caption{SidelabCode Stack Librería API}
    \label{fig:sidelabcodestack-api-lib}
\end{figure}

\par Para las pruebas del proyecto de la generación del API se ha utilizado en framework \emph{JUnit}\footnote{JUnit - \url{http://junit.org/}}. La cobertura de tests del proyecto en JUnit no es muy halagüeña ya que es del 3,5\%~\ref{fig:sidelabcodestack-sonar} pero en su defensa se puede justificar debido a que se trata de una librería que accede a APIs de terceros. Las APIs o librerías de terceros no han de desarrollar test unitarios sino test test de aprendizaje por lo que no son tan útiles~\cite{clean-code}.

\par Por parte de la publicación del API como servicio Web REST se ha utilizado framework Restlet pero en este caso vemos que la cobertura de test es del 0\%~\ref{fig:sidelabcodestack-sonar-rest-service}. A partir de un proyecto nuevo y dependiente que utiliza la libraría API para publicarla como Servicio Web REST ofreciendo las funcionalidades a la Consola de Administración de la Forja.

\par He utilizado la herramienta \emph{SonarQube}\footnote{SonarQube - \url{http://www.sonarqube.org/}} para obtener una visión más empírica del estado de la cobertura del proyecto. Utilizando las reglas básicas para poder fijarnos en la cantidad de pruebas definidas para SidelabCode Stack \emph{API Lib}~\ref{fig:sidelabcodestack-sonar} y \emph{REST Services}~\ref{fig:sidelabcodestack-sonar-rest-service} ya que para obtener este dato no hace falta un gran refinamiento en la configuración.

\begin{figure}[H]
    \centering
    \includegraphics[width=\textwidth]{sidelabcodestack-sonar}
    \caption{SidelabCode Stack API lib Análisis de cobertura con Sonar}
    \label{fig:sidelabcodestack-sonar}
\end{figure}

\begin{figure}[H]
    \centering
    \includegraphics[width=\textwidth]{sidelabcodestack-sonar-rest-service}
    \caption{SidelabCode Stack REST Service Análisis de cobertura con Sonar}
    \label{fig:sidelabcodestack-sonar-rest-service}
\end{figure}

\par El an\'alisis del seguimiento a trav\'es de SonarQube tiene como objetivo mostrar el estado del proyecto y definir el horizonte al que se quiere llegar:

\begin{itemize}
	\item Incrementar la calidad dentro de un rango. Partiendo de una base como primer an\'alisis.
	\item Controlar el avance la de la calidad por iteraciones.
	\item Mejorar el desarrollo.
\end{itemize}

\subsection{Pruebas instalación}
\label{sub:pruebas-instalacion}

\par Las pruebas más importantes en el proceso de desarrollo de la forja SidelabCodeStack han sido las de instalación y configuración de la forja que es donde se encuentra el valor del proyecto. 

\par Previamente a la versión de la forja 0.2 la instalación se ejecutaba a partir de un instalador Java y en consecuencia las pruebas y la replicación de los errores en diferentes entornos acarreaban un trabajo que no podía asumirse con sencillez. A partir de la citada versión (0.2) se introdujo Puppet como herramienta de aprovisionamiento y configuración de los distintos módulos que componen la forja. Este framework ofrece un nuevo camino dentro de las pruebas en el proceso de desarrollo, \emph{la virtualización de las instalaciones}. Puppet nos proporciona un seguimiento completo de la instalación paso a paso a partir de la configuración definida en el proceso. Facilita la ejecución de la instalación de forja en entornos virtuales y seguros para poder asegurar el correcto funcionamiento y descubrir los posibles errores.

\begin{figure}[H]
    \centering
    \includegraphics[width=0.5\textwidth]{sctack-puppet}
    \caption{Estructura de módulo Puppet de SCStack}
    \label{sctack-puppet}
\end{figure}

\par Pero no, no es suficiente ya que este manera el desarrollador ha de estar preocupado de invertir el tiempo en la gestión de las distintas máquinas virtuales, asignando espacio, potencia, configuración inicial, sobrecarga de la máquina de desarrollo y cumplir con una serie de tareas repetitivas que le alejan del camino marcado, el proyecto SidelabCode Stack.

\par En este punto aparece \emph{Vagrant}\footnote{Vagrant - \url{http://www.vagrantup.com/}}.

\begin{figure}[H]
    \centering
    \includegraphics[width=0.3\textwidth]{vagrant}
    \caption{Vagrant Logo}
    \label{fir:vagrant}
\end{figure}

\begin{quote}
    \emph{Create a single file for your project to describe the type of machine you want, the software that needs to be installed, and the way you want to access the machine. Store this file with your project code.}
\end{quote}

\par Vagrant proporciona la gestión de los entornos de desarrollo sobre máquinas virtuales, ligeras, sencillas y replicables. Existe un repositorio de imágenes de virtuales\footnote{VagrantBox.es - \url{http://www.vagrantbox.es/}} preparadas para Vagrant que se han de instalar la máquina del desarrollador con un simple comando:

\lstset{style=bashbasico}
\begin{lstlisting}[frame=trbl]
config.vm.box = "precise64"
config.vm.network :hostonly, "192.168.33.10" # Si se desea se puede utilizar el modo bridge y asignar una ip por dhcp a la maquina dentro de la red en la que se encuentra el host
config.vm.provision :puppet, :module_path => "modules"
\end{lstlisting}

\par Se puede afirmar que Vagrant es la navaja suiza para las pruebas en este proyecto. A través de Vagrant se define el sistema operativo a instalar, la capacidad de la máquina y lo más importante: el \textbf{provisionador}, en nuestro caso Puppet y el módulo SidelabCode Stack.

\par A través de dos comandos se inicia el proceso completo de la instalación de SCStack en una nueva máquina virtual a partir de la configuración de una plantilla con las propiedades necesarias (Apéndice ~\ref{app:instalacion-sidelab}).

\lstset{style=bashbasico}
\begin{lstlisting}[frame=trbl]
$ vagrant box add base http://files.vagrantup.com/precise64.box
$ vagrant init
$ vagrant up
\end{lstlisting}

\par Permite la ejecución de pruebas para la instalación de SCStack de una manera eficiente y controlada aunque también es pesada debido a las características del proyecto.

\par De esta forma se completa el proceso de desarrollo con respecto a las herramientas y módulos utilizados para el desarrollo de la forja SidelabCode Stack ofreciendo las soluciones y herramientas necesarias para poder facilitar el uso del proceso desarrollo Iterativo e Incremental dentro de un marco seguro y fiable.

% subsection pruebas-instalacion (end)

% section pruebas-validacion (end)


%%%%%%%%%%%%%%%%%%%%%%%%%%%%%%%%%%%%%%

\chapter{Comunidades FLOSS}
\label{chap:comunidades}

\par El punto diferenciador en el proyecto haciendo hincapi\'e en la interacci\'on con las comunidades de software de cada una de las herramientas.

%%%%%%%%%%%%%%%%%%%%%%%%%%%%%%%%%%%%%%

%%%%%%%%%%%%%%%%%%%%%%%%%%%%%%%%%%%%%%%%%%%%%%%%%%%%%%%%%%%%%%%%%%%%%%%%%%%%%%%%%%%%%%%%%%%%%%%%%%%%%%%%%%%
%   \copyright 2013 Ricardo García Fernández - ricardogarfe [at] gmail [dot] com.
%
%    This work is licensed under a Creative Commons 3.0 Unported License.
%    To view a copy of this license visit:
% 
%    http://creativecommons.org/licenses/by/3.0/legalcode
%%%%%%%%%%%%%%%%%%%%%%%%%%%%%%%%%%%%%%%%%%%%%%%%%%%%%%%%%%%%%%%%%%%%%%%%%%%%%%%%%%%%%%%%%%%%%%%%%%%%%%%%%%

\chapter{Desarrollo de un proyecto}
\label{chap:desarrollo}

\par Cómo llevar a cabo el desarrollo de un proyecto con SCStack. Desarrollo en paralelo de un proyecto mediante github + travis-ci vs gerrit + jenkins.

\section{Alta usuarios}
\label{sec:alta-usuarios}

\par Dar de alta desarrolladores/Project Owners (usuarios de la forja).

% section alta-usuarios (end)

\section{Crear un proyecto}
\label{sec:crear-proyecto}

\subsection{Proyecto en redmine}
\label{sub:proyeto-redmine}

% subsection proyeto-redmine (end)

\subsection{Repositorio git}
\label{sub:repo-git}

\par Configurar usuario Git del desarrollador del proyecto para la forja:

\lstset{style=bashbasico}
\begin{lstlisting}[frame=trbl]
$ cat .gitconfig 
[user]
    name = patxigortazar
    email = patxi.gortazar@gmail.com
$ git config --global user.name "ricardogarfe"
$ git config --global user.email "ricardogarfe@gmail.com"
$ cd [path-to-gitrepo]
[path-to-gitrepo]$ git config user.name "ricardogarfe"
[path-to-gitrepo]$ git config user.email "ricardogarfe@gmail.com"
\end{lstlisting}

\par Comprobar las credenciales de Git en el ordenador del desarrollador:

\lstset{style=bashbasico}
\begin{lstlisting}[frame=trbl]$ git config --list
user.name=patxigortazar
user.email=patxi.gortazar@gmail.com
\end{lstlisting}

% subsection repo-git (end)

\subsection{Configuración de Jenkins}
\label{sub:jenkins-configuracion}

\par Configuracón de Jenkins para realizar determinadas tareas de forma automática:

\begin{itemize}
	\item Tags
	\item Construir las versiones vivas
\end{itemize}

\par También proveerá tareas para ser ejecutadas manualmente:

\begin{itemize}
	\item Branches de releases
	\item Desplegar una versión específica con un clic
\end{itemize}

\par Las versiones vivas vivirán en su propia máquina virtual

\begin{figure}[H]
    \centering
    \includegraphics[width=0.6\textwidth]{jenkins-git}
    \caption{Jenkisn-Git relación en el proceso de integración}
    \label{fig:jenkins-git}
\end{figure}

\par Se mantendrán diferentes versiones vivas a la vez para cumplir unos objetivos con forme al desarrollo Iterativo e Incremental: \emph{'Release early, release often'}:

\begin{itemize}
	\item Asegurar la calidad
	\item Hacer el despliegue ágil
	\item Minimizar el riesgo
\end{itemize}

\par Para orientar la integración continua al proceso de desarrollo se ha de configurar \emph{Jenkins} para acceder a múltiples repositorios Git para esto se han de tener en cuenta los siguientes requerimientos:

\begin{itemize}
    \item \emph{Jenkins} construye diferentes proyectos en donde en proyecto puede tener su propio repositorio git.
    \item \emph{Jenkins} debe tener permisos de lectura o lectura/escritura a todos los repositorios de aquellos proyectos que vaya a construir.
    \item Por defecto \emph{Jenkins} usa la clave del usuario en \texttt{\ensuremath{\sim.ssh}} para autenticarse
    \item ¿Con qué usuario se ejecuta Jenkins? con el usuario \textbf{tomcat}.
\end{itemize}

\par El acceso de Jenkins a los distintos repositorios se configura a partir de un usuario por Jenkins repositorio, aislando los problemas descritos en la sección~\ref{sub:ci-jenkins} del capítulo \nameref{chap:procesos-desarrollo}.

\par Configurar \emph{Jenkins} para acceso a \emph{múltiples repositorios git} creando el usuario necesario en la Consola de Administración.

\begin{figure}[H]
    \centering
    \includegraphics[width=\textwidth]{gestion-jenkins-usuarios-000}
    \caption{Crear usuarios para Jenkins a través de la consola de Administración}
    \label{fig:gestion-jenkins-usuarios-000}
\end{figure}

\par Generar un par de claves pública/privada con \texttt{ssh-keygen} para cada usuario en diferentes ficheros y acceder a Gerrit con cada usuario creado.

\begin{figure}[H]
    \centering
    \includegraphics[width=\textwidth]{gestion-jenkins-usuarios-001}
    \caption{Configuración de usuarios Jenkisn en Gerrit}
    \label{fig:gestion-jenkins-usuarios-001}
\end{figure}

\par Añadir la clave pública para este usuario para copiar las claves al servidor de Jenkins en el directorio\texttt{'/opt/ssh-keys'}.

\lstset{style=bashbasico}
\begin{lstlisting}[frame=trbl]
$ git config --list
$ cd /opt/ssh-keys
$ ll
total 24
drwxr-xr-x  2 tomcat tomcat 4096 Jan  4 09:46 ./
drwxr-xr-x 14 root   root   4096 Jan  4 09:42 ../
- rw-------  1 tomcat tomcat 1679 Jan  4 09:46 filetransferci_rsa
- rw-r--r--  1 tomcat tomcat  398 Jan  4 09:46 filetransferci_rsa.pub
- rw-------  1 tomcat tomcat 1679 Jan  4 09:44 samplegitci_rsa
- rw-r--r--  1 tomcat tomcat  396 Jan  4 09:44 samplegitci_rsa.pub
</code>
\end{lstlisting}

\par Configurar SSH para que utilice la clave correcta en cada caso creando el fichero \texttt{/home/tomcat/.ssh/config}.

\lstset{style=bashbasico}
\begin{lstlisting}[frame=trbl]
$ git config --list
$ cd /home/tomcat/.ssh
$ cat config
Host samplegit.ricardogarfe.sidelab.es
    HostName ricardogarfe.sidelab.es
    User samplegitci
    IdentityFile /opt/ssh-keys/samplegitci_rsa
Host filetransfer.ricardogarfe.sidelab.es
    HostName ricardogarfe.sidelab.es
    User filetransferci
    IdentityFile /opt/ssh-keys/filetransferci_rsa
\end{lstlisting}

\subsection{Configuración de builds}
\label{sub:jenkins-build-jobs}

\par Los builds de \emph{Jenkins} funcionan a través de la configuración de \textbf{jobs}. Por lo que vamos a definir los jobs necesarios para el proceso de integración continua. Se dividen en tres grupos:

\begin{itemize}
    \item Jobs de \textbf{integración} (read only).
        \begin{itemize}
            \item Descargan el código (checkout).
            \item Construyen.
            \item Pasan tests.
            \item Despliegan la versión construida en ``local''.
        \end{itemize}
    \item Jobs de \textbf{release} (read/write).
        \begin{itemize}
            \item Realizan los pasos anteriores y además.
            \item Tag si los tests pasaron.
            \item Push del tag al repositorio remoto.
        \end{itemize}
    \item Jobs de \textbf{despliegue} (read only).
        \begin{itemize}
            \item Descargan el binario del repositorio de binarios
            \item Desplegar
        \end{itemize}
\end{itemize}

\subsubsection{Job de integración}
\label{subs:jenkins-job-integracion}

\par Configurar el job de integración mediante Maven:

\begin{itemize}
	\item Crear un \textbf{job} \emph{Maven}.
	\item Configurar el repositorio git:
        \begin{itemize}
	        \item \emph{ssh://filetransferci@filetransferci.code.tscompany.es/filetransfer}
	        \item Ssh leerá el fichero config y utilizará el fichero de claves correspondiente el host \texttt{filetransferci.code.tscompany.es}.
        \end{itemize}
	\item Añadir las ramas a construir (añadir nuevas ramas con el botón ``Add'')
        \begin{itemize}
	        \item development
	        \item release-0.1.1
        \end{itemize}
	\item Añadir el \texttt{user.email} y \texttt{user.name} que usará \emph{Jenkins}.
        \begin{figure}[H]
            \centering
            \includegraphics[width=0.7\textwidth]{jenkins-job-integracion}
            \caption{Crear Job integración en Jenkins}
            \label{fg:jenkins-job-integracion}
        \end{figure}
	\item Los resultados del build se pueden comprobar en: 
        \begin{itemize}
	        \item \texttt{/opt/jenkins/jobs/filetransfer/workspace}
	        \item Si es un proyecto \emph{Maven}, dentro del proyecto en la carpeta target estará el artefacto generado.
	        \item También se puede acceder vía web y descargar el workspace como un zip.
	        \item Los \textbf{tests} están en la carpeta \texttt{surfire-reports} del proyecto \emph{Maven}.
	        \item También pueden consultarse vía web accediendo al build y seleccionando \emph{'Resultado de los tests'}.
        \begin{figure}[H]
            \centering
            \includegraphics[width=\textwidth]{jenkins-job-resultados}
            \caption{Resultados de los test a través de la interfaz de Jenkins}
            \label{fig:jenkins-job-resultados}
        \end{figure}
        \end{itemize}
\end{itemize}

\subsection{Maven}
\label{sub:jenkins-maven}

\par \emph{Jenkins} permite construir proyectos \emph{Maven}.

\par En determinadas ocasiones los proyectos requieren configuraciones específicas. La información sensible suele ir en el fichero \texttt{settings.xml} en el \texttt{home} del usuario en su máquina de desarrollo.

\begin{itemize}
    \item Info de \textbf{autenticación para Archiva}.
    \item \textbf{Profiles}
\end{itemize}

\par En Jenkins esto se puede gestionar con el plugin \emph{``Config File Provider Plugin''}.

\begin{figure}[H]
    \centering
    \includegraphics[width=\textwidth]{jenkins-config-file-management}
    \caption{Configuración de Maven a través de Jenkins}
    \label{fig:jenkins-config-file-management}
\end{figure}

\par Podemos añadir cualquiera de los ficheros creados con \emph{Config File Management} en un \textbf{job}.

\begin{figure}[H]
    \centering
    \includegraphics[width=\textwidth]{jenkins-config-file-management-settings}
    \caption{Seleccionar archivo de configuración de Maven}
    \label{fig:jenkins-config-file-management-settings}
\end{figure}

\par Para los deploys, si el certificado es autofirmado es \textbf{necesario} generar un \textbf{truststore} a partir del certificado generado por el servidor\footnote{\url{http://www.liferay.com/web/neil.griffin/blog/-/blogs/fixing-suncertpathbuilderexception-caused-by-maven-downloading-from-self-signed-repository}}.

\par Este truststore debe incluirse en todos los \texttt{jdk} que utilice \emph{Jenkins} en la ruta indicada en el enlace anterior.

% subsection jobs-jenkins (end)

% section crear-proyecto (end)

\section{Proceso de desarrollo basado en ramas}
\label{sec:desarrollo-en-ramas}

\par Proceso de desarrollo basado en ramas partiendo de 2 ramas de forma continua:

\begin{itemize}
    \item \textbf{master:} Desarrollo limpio. Sólo versiones estables.
    \item \textbf{develop:} El desarrollo inicial de la versión actual tiene lugar aquí.
\end{itemize}

\par Gestión de las ramas para estabilización de las versiones:

\begin{itemize}
    \item \textbf{release-0.1}, \textbf{release-0.2}; una rama de estabilización cada vez   
\end{itemize}

\subsection{Proceso de estabilización}

\par El proceso de estabilización se gestiona a través de las \emph{ramas de estabilización}:

\begin{itemize}
    \item Estabilización del código (\emph{RC release candidates})
    \item Arreglar bugs (hotfixes)
    \item Cuando la versión se considera estable se procede al siguiente paso:
        \begin{itemize}
            \item Tag
            \item Mezclar (merge) con development
            \item Mezclar (merge) con master
        \end{itemize}
    \item Si surgen nuevos bugs se vuelve a repetir el \textbf{proceso de estabilización}:
        \begin{itemize}
            \item Se arreglan en la misma rama (release-0.1)
            \item Nuevo tag y mezcla
        \end{itemize}
\end{itemize}

\subsection{Releasing}

\par Para crear una Release se define un proceso de gestión a través las ramas:

\begin{itemize}
    \item Checkout del tag
    \item Build (Jenkins)
    \item Deploy (Jenkins)
\end{itemize}

\subsection{Diagrama de desarrollo}

\par El flujo de trabajo a través de las ramas se representa en este diagrama:

\begin{figure}[H]
\centering
\includegraphics[width=0.6\textwidth]{flujo-desarrollo-git-000}
\caption{Flujo de desarrollo Git.}
\label{fig:flujo-desarrollo-git-000}
\end{figure}

% section desarrollo-en-ramas (end)

%%%%%%%%%%%%%%%%%%%%%%%%%%%%%%%%%%%%%%

%%%%%%%%%%%%%%%%%%%%%%%%%%%%%%%%%%%%%%

%%%%%%%%%%%%%%%%%%%%%%%%%%%%%%%%%%%%%%%%%%%%%%%%%%%%%%%%%%%%%%%%%%%%%%%%%%%%%%%%%%%%%%%%%%%%%%%%%%%%%%%%%%%
%   \copyright 2013 Ricardo García Fernández - ricardogarfe [at] gmail [dot] com.
%
%    This work is licensed under a Creative Commons 3.0 Unported License.
%    To view a copy of this license visit:
% 
%    http://creativecommons.org/licenses/by/3.0/legalcode
%%%%%%%%%%%%%%%%%%%%%%%%%%%%%%%%%%%%%%%%%%%%%%%%%%%%%%%%%%%%%%%%%%%%%%%%%%%%%%%%%%%%%%%%%%%%%%%%%%%%%%%%%%

\chapter{Ep\'ilogo}
\label{chap:epilogo}

\section{Conclusiones}
\label{sec:conclusiones}

\par El uso de estas herramientas y su incremento de la calidad en el desarrollo, por encima de todo siendo FLOSS debido a eso la versatilidad que otorga en el momento de unificarlas en una herramienta nueva; SidelabCode Stack.


\section{Lecciones aprendidas}
\label{sec:lecciones}

\par Colaboraci\'on entre distintos proyectos y comunidades, interoperabilidad entre herramientas, Forjas de desarrollo y los elementos m\'as comunes de las mismas.

\section{Trabajo Futuro}
\label{sec:trabajofuturo}

\par Impulso de la comunidad a trav\'es de los canales habituales.

\par Integraci\'on y gesti\'on de nuevas herramientas comunes para los desarrolladores.

\par Centralizaci\'on de la instalaci\'on.

% section epilogo (end)

%%%%%%%%%%%%%%%%%%%%%%%%%%%%%%%%%%%%%%

\appendix
\clearpage % o \cleardoublepage
\addappheadtotoc
\appendixpage
% Puppet proyecto de ejemplo.
\chapter{Puppet}
\label{app:apendice-puppet}

\par Instalación de Puppet siguiendo la documentación ofrecida en su página web\footnote{Puppet Documentation - \url{http://docs.puppetlabs.com/guides/installation.html\#debian-and-ubuntu}}.

\par El proceso automatiza la instalación y configuración manual de un componente, un servidor web.

\begin{itemize}
	\item Primero se ha de crear la instalación manual del servidor en un entorno definido (Sistema operativo Ubuntu). Configurar los permisos, definir archivos de configuración, directorios a servir, públicos/privados.
	\item Recapitular todos los pasos seguidos en la instalación y la información sobre la configuración para reescribirla a un módulo Puppet.
	\item Crear el esqueleto del módulo Puppet.
	\item Definir las órdenes a ejecutar organizando el módulo en distintas clases y apartados cada una responsable de una funcionalidad.
	\begin{itemize}
	    \item Descargar servidor web.
	    \item Instalar servidor web.
	    \item Configurar servidor web a través de una plantilla Ruby basada en los ficheros de configuración base del servidor web. Directorios, usuarios, urls, dominios, web de bienvenida, etc.
	    \item Poner en marcha el servidor.
    \end{itemize}
\end{itemize}

\par Puppet da la opción de publicar los módulos desarrollados en un repositorio central de módulos para que estén accesibles para los usuarios de Puppet con un sistema de búsqueda para ayudar a los usuarios.

\par Con este módulo Puppet podemos replicar la instalación en cualquier entorno con tan solo ejecutar el módulo a través de Puppet para la ejecución. Descargar el módulo y ejecutar el commando Puppet.

\par Código Puppet de ejemplo para la instalación del servidor web Apache a través del módulo Puppet publicado en PuppetForge\footnote{Pupept Apache Module - \url{https://forge.puppetlabs.com/puppetlabs/apache}}:

\par Instalar Apache importando la clase del módulo:

\lstset{style=rubybasico}
\begin{lstlisting}[frame=trbl]
class {'apache':  }
\end{lstlisting}

\par Módulo PHP de Apahce:

\lstset{style=rubybasico}
\begin{lstlisting}[frame=trbl]
class {'apache::mod::php': }
\end{lstlisting}

\par Configurar Host Virtual mediante los distintos parámetros de configuración, partiendo del mínimo un ejemplo sería este:

\lstset{style=rubybasico}
\begin{lstlisting}[frame=trbl]
apache::vhost { 'www.example.com':
    priority        => '10',
    vhost_name      => '192.0.2.1',
    port            => '80',
}
\end{lstlisting}

\par Se pueden definir más parámetros de configuración:

\lstset{style=rubybasico}
\begin{lstlisting}[frame=trbl]
apache::vhost { 'www.example.com':
    priority        => '10',
    vhost_name      => '192.0.2.1',
    port            => '80',
    docroot         => '/home/www.example.com/docroot/',
    logroot         => '/srv/www.example.com/logroot/',
    serveradmin     => 'webmaster@example.com',
    serveraliases   => ['example.com',],
}
\end{lstlisting}

% subsection puppet-caso-de-uso (end)

% section puppet (end)

% Instalación SidelabCode Stack
\chapter{Instalación SidelabCode Stack}
\label{app:instalacion-sidelab}

\par En la versión 0.4 de SidelabCode Stack se utiliza Puppet en el proceso de instalación. Ahora es extremadamente sencillo instalar SidelabCode Stack usando Puppet o Vagrant con el nuevo instalador.

\par Puppet nos permite automatizar la instalación en un servidor \emph{Ubuntu} mediante una sencilla configuración.

\par El uso de \emph{Vagrant} permite probar SidelabCode Stack en una máquina virtual en 5 minutos.

\section{Requisitos}
\label{sec:requisitos}

\par Para la instalación de SidelabCode Stack es necesario obtener los siguientes recursos.

\par Dependiendo de la instalación que se utilice \textbar{Vagrant} o \textbar{Puppet} se han de seguir instrucciones diferentes.

\subsection{Descarga SidelabCode Stack}
\label{sub:descarga}

\par Descargar el instalador de SidelabCode Stack (módulo scstack) y sus dependencias de la url:

\begin{itemize}
	\item \url{http://code.sidelab.es/public/sidelabcodestack/artifacts/0.3/puppet-installer-0.3-bin.tar.gz}
\end{itemize}

\lstset{style=rubybasico}
\begin{lstlisting}[frame=trbl]
    cd $HOME
    mkdir tmp
    cd $HOME/tmp
    wget http://code.sidelab.es/public/sidelabcodestack/artifacts/0.3/puppet-installer-0.3-bin.tar.gz
\end{lstlisting}

\par Descomprimir y copiar a la carpeta modules:

\lstset{style=rubybasico}
\begin{lstlisting}[frame=trbl]
    cd $HOME/tmp
    tar xvzf puppet-installer-0.3-bin.tar.gz
\end{lstlisting}

\subsection{Configuración de módulos puppet}
\label{sub:conf-modulos-puppet}

\par Creamos un fichero \texttt{default.pp} en la carpeta \texttt{tmp} con el siguiente contenido:

\lstset{style=rubybasico}
\begin{lstlisting}[frame=trbl]
    exec { "apt-update":
        command => "/usr/bin/apt-get update",
    }
    class { "scstack":
        # Superadmin password. Will be used to access the SidelabCode Stack Console re@lity45
        sadminpass => "re@lity45",
        # Or whatever IP specified in Vagrantfile
        ip => "192.168.33.10", 
        domain => "sidelabcode03.scstack.org",
        baseDN => "dc=sidelabcode03,dc=scstack,dc=org",
        # Your company/organization name
        compname => "SidelabCode Stack version 0.3",
        # A name to be displayed within Redmine
        codename => "SCStack ALM Tools",
    }
\end{lstlisting}

\subsection{Apt Cacher}
\label{sub:apt-cacher}

\par Dependiendo de la conexión de red, el proceso de instalación puede tardar más o menos. En general es buena idea configurar un proxy para los paquetes debian. Esta opción es recomendable en el proceso de instalación a través de Vagrant. Para ello, simplemente hay que instalar \textbf{apt-cacher} en el host:

\lstset{style=rubybasico}
\begin{lstlisting}[frame=trbl]
    sudo apt-get install apt-cacher
\end{lstlisting}

\par Modificar las siguientes líneas del fichero \emph{/etc/apt-cacher/apt-cacher.conf}:

\lstset{style=rubybasico}
\begin{lstlisting}[frame=trbl]
    daemon_addr = 192.168.33.1 # No podemos usar localhost si queremos que los clientes se puedan conectar
    allowed_hosts = * # Permitir a todos los clientes conectarse a este proxy.
    generate_reports = 1 # Generar un informe cada 24h
\end{lstlisting}

\par Reiniciar el servicio:

\lstset{style=rubybasico}
\begin{lstlisting}[frame=trbl]
    sudo /etc/init.d/apt-cacher restart
\end{lstlisting}

\par Modificar el fichero \texttt{/etc/hosts} con la ip del host y el dominio asociado que se define en el fichero de configuración \texttt{default.pp} y la ip y el dominio del cliente para la comunicación con \texttt{apt-cacher} definido en el parámetro \texttt{daemon\_addr} del fichero \texttt{/etc/apt-cacher/apt-cacher.conf}:

\lstset{style=rubybasico}
\begin{lstlisting}[frame=trbl]
    192.168.33.10   sidelabcode03.scstack.org
    192.168.33.1    host.scstack.es
\end{lstlisting}

\par Añadir el siguiente trozo de código en la primera línea del fichero \texttt{default.pp} para que utilice el apt-cacher creado:

\lstset{style=rubybasico}
\begin{lstlisting}[frame=trbl]
    file { "/etc/apt/apt.conf.d/01proxy":
        content => 'Acquire::http::Proxy "http://192.168.33.1:3142/apt-cacher";',
    }
\end{lstlisting}

\section{Vagrant}
\label{sec:vagrant}

\par La instalación a través de Vagrant permite probar SidelabCode Stack en una máquina virtual en 5 minutos.

\subsection{Descripción del entorno de instalación}
\label{sub:entorno-instalacion}

\par Básicamente lo que vamos a hacer es indicar a Vagrant que monte una red privada entre la máquina host y la máquina virtual donde instalaremos SidelabCode Stack. Esto nos permite tener acceso a la forja desde el host. Normalmente, Vagrant asigna, dentro de esa red privada, la IP 192.168.33.1 al host y la IP 192.168.33.10 a la máquina virtual. Estos valores se pueden cambiar como veremos posteriormente.

\subsection{Prerequisitos}
\label{sub:prerequisitos}

\begin{itemize}
    \item Instalar Vagrant y Virtualbox (descargar las últimas versiones de las respectivas páginas web)
    \item Añadir Vagrant al path
\end{itemize}

\lstset{style=rubybasico}
\begin{lstlisting}[frame=trbl]
    $ gedit $HOME/.bashrc
    $ PATH=$PATH:/opt/vagrant/bin
\end{lstlisting}

\subsection{Preparación de la VM}
\label{preparar-vm}

\par Utilizaremos Vagrant para provisionar una VM con Java donde poder instalar SidelabCode Stack. Preparamos la carpeta para el proyecto Vagrant.

\lstset{style=rubybasico}
\begin{lstlisting}[frame=trbl]
    mkdir vagrant
    mkdir vagrant/manifests
    mkdir vagrant/modules
\end{lstlisting}

\par Descargar una imagen vagrant (precise64 es la que elegimos en esta documentación, por ser LTS).

\lstset{style=rubybasico}
\begin{lstlisting}[frame=trbl]
    vagrant box add precise64 http://files.vagrantup.com/precise64.box
\end{lstlisting}

\par Si quiere utilizar una versión distinta de una distribución, puede seleccionarla en la lista de \textbf{boxes} que proporciona \href{http://www.vagrantbox.es/}{vagrant para virtualbox}.

\par Crear un proyecto Vagrant:

\lstset{style=rubybasico}
\begin{lstlisting}[frame=trbl]
    vagrant init
\end{lstlisting}

\par Modificar el Vagrantfile que describe el proyecto (VM, provisioner, etc) para que arranque la vm ``precise64'' y utilice Puppet:

\lstset{style=rubybasico}
\begin{lstlisting}[frame=trbl]
config.vm.box = "precise64"
config.vm.network :hostonly, "192.168.33.10" # Si se desea se puede utilizar el modo bridge y asignar una ip por dhcp dentro de la red en la que se encuentra el host
config.vm.provision :puppet, :module_path => "modules"

\end{lstlisting}

\par Copiar el instalador de la forja en la carpeta modules del proyecto Vagrant:

\lstset{style=rubybasico}
\begin{lstlisting}[frame=trbl]
    cp -R puppet-installer-0.3/* $HOME/vagrant/modules
\end{lstlisting}

\par Copiar el fichero \texttt{default.pp} en la carpeta \texttt{manifests} del proyecto Vagrant:

\lstset{style=rubybasico}
\begin{lstlisting}[frame=trbl]
    cp /tmp/default.pp $HOME/vagrant/manifests
\end{lstlisting}

\par Arrancar la vm.

\lstset{style=rubybasico}
\begin{lstlisting}[frame=trbl]
    cd $HOME/vagrant
    vagrant up
\end{lstlisting}

\section{Puppet}
\label{puppet}

\textbf{TBC}: Descripción

\subsection{Pre requisitos}
\label{sub:puppet-pre-requisitos}

\par Puppet se ha de instalar mediante el gestor de paquetes de la distribución, en este caso \texttt{apt} para Ubuntu:

\lstset{style=rubybasico}
\begin{lstlisting}[frame=trbl]
    $ [sudo] apt-get install puppet
\end{lstlisting}

\par La instalación en otras distribuciones se puede consultar en el \href{http://docs.puppetlabs.com/guides/installation.html}{manual de puppet}.

\subsection{Configuración}
\label{sub:configuracion-puppet}

\par Copiar los módulos puppet al directorio de módulos definido:

\lstset{style=rubybasico}
\begin{lstlisting}[frame=trbl]
    $ mkdir -p $HOME/puppet/modules
    $ cp -R puppet-installer-0.3/* $HOME/puppet/modules
\end{lstlisting}

\par Ejecutar puppet para el proceso de instalación mediante sudo:

\lstset{style=rubybasico}
\begin{lstlisting}[frame=trbl]
    $ sudo puppet apply --modulepath=$HOME/puppet/modules default.pp
\end{lstlisting}

\section{Post instalación}
\label{sec:post-instalacion}

\par El proceso de instalación comenzará automáticamente. Una vez finalizado, en la dirección
\href{http://test.scstack.org/redmine}{http://test.scstack.org/redmine} se mostrará Redmine. La consola es accesible a través de la dirección \href{https://test.scstack.org:5555}{https://test.scstack.org:5555}. Es posible administrar scstack accediendo con el usuario \emph{sadmin} y la contraseña especificada en el parámetro \emph{sadminpass}.

\par \textbf{Nota:} La consola de administración se ha comprobado el funcionamiento para los siguientes navegadores:

\begin{itemize}
	\item \href{http://www.mozilla.org/es-ES/firefox/new/}{Firefox}.
	\item \href{https://www.google.com/intl/es/chrome/browser/?hl=es}{Chrome/Chromium}

\end{itemize}

\subsection{Primeros pasos}
\label{sub:primeros-pasos}

\par Después de la instalación automatizada del entorno se ha de acceder a las herramientas \textbf{Redmine} y \textbf{Archiva} para completar el proceso.

\par Antes de nada se ha de reiniciar la máquina para comprobar que se ejecutan todos los procesos en el inicio.

\subsubsection{Redmine}
\label{subs:conf-redmine}

\par odificar la fecha de creación de la api key de admin desde mysql:

\lstset{style=rubybasico}
\begin{lstlisting}[frame=trbl]
mysql -u root -p
> use redminedb;
> update tokens set created_on='2013-01-16 22:16:00' where id='1';
\end{lstlisting}

\par Modificar los permisos de las carpetas en /opt/redmine/tmp para que sean del usuario de apache:

\begin{itemize}
\item
  \texttt{cd /opt/redmine/tmp \&\& chown -R www-data:www-data *}
\end{itemize}
Acceder a la URL de Redmine, con el usuario \texttt{admin} y el password
\texttt{admin}:

\begin{itemize}
\item \href{https://test.scstack.org/redmine}{https://test.scstack.org/redmine}
\end{itemize}

\par Cambiar la contraseña para que no utilice la genérica a través de la ruta:

\begin{itemize}
\item Administración \textgreater{} Users \textgreater{} admin \textgreater{} Authentication \textgreater{} actualizar.
\end{itemize}

\par Cambiar la API key para securizar el acceso a la API rest (scstack instala una por defecto, pero no es segura):

\begin{itemize}
    \item Acceder como admin \textgreater{} My account \textgreater{} API  access key \textgreater{} Reset \textgreater{} Show -\textgreater{} Copiar la key
    \item Pegar la key en el fichero /opt/scstack-service/scstack.conf
    \item Reiniciar el servicio scstack: \texttt{sudo service scstack-service restart}
\end{itemize}

\subsubsection{Gerrit}
\label{subs:conf-gerrit}

\par El primer usuario que accede a Gerrit obtiene privilegios de administrador. Al instalar la forja, se recomienda crear un usuario ``gerritadmin'' y password ``t0rc0zu310'' y acceder con este usuario a Gerrit. Este usuario se convertirá en administrador automáticamente al hacer login. A partir de este momento, este será el usuario con el que crear los grupos y proyectos (repositorios) en Gerrit.

\par Obtener la clave pública del servidor para asignarla al usuario \textbf{gerritadmin}:

\begin{enumerate}
\item \emph{Settings \textgreater{} SSH Public Keys}\textgreater{} Add.
\item Copiar la clave del fichero \texttt{/opt/ssh-keys/gerritadmin\_rsa.pub}.
\end{enumerate}

\par Configurar permisos para creación de proyectos:

\begin{enumerate}
    \item Acceder a \emph{Projects \textgreater{} List}\textgreater{} All-Projects.
    \item Seleccionar \emph{Access}:
    \item \emph{Editar}:
    \begin{itemize}
	    \item \texttt{Bloque} \textbf{refs/} Add Permission \textgreater{} añadir \emph{Push} el grupo \emph{Administrators}.
	    \item \texttt{Bloque} \textbf{refs/meta/config} Add Group \textgreater{} añadir a \emph{Read} el grupo \emph{Administrators}.
	    \item Save Changes.
    \end{itemize}
\end{enumerate}


\subsubsection{Archiva}
\label{subs:conf-archiva}

\par Acceder a la URL de Archiva:

\begin{itemize}
    \item \url{https://test.scstack.org/archiva}
\end{itemize}

\par La primera vez que se configura archiva pide los datos del administrador. Apuntarlos convenientemente para posteriores necesidades de administración. Se recomienda utilizar la contraseña del administrador de la forja definida en el fichero de configuración \texttt{default.pp}.

\paragraph{Repositorios}

\par Por defecto, Archiva trae configurado un \textbf{repositorio internal} que hace de proxy de \emph{Maven Central} y \emph{java.net}. Si hace falta añadir repositorios remotos adicionales, en Repositories, al final de la página se pueden añadir repositorios remotos.

\par Archiva trae configurado un \textbf{repositorio de snapshots}. Se recomienda crear uno de \textbf{releases}.

\par Para ello accedemos a la administración de Archiva \texttt{Administration -\textgreater{} Repositories} y añadimos uno nuevo con los siguientes parámetros:

\lstset{style=rubybasico}
\begin{lstlisting}[frame=trbl]
Identifier*: *releases*
Name*:       *Archiva Managed Releases Repository*
Directory*:  */opt/tomcat/data/repositories/releases*
...
Repository Purge By Days Older Than: *30*
\end{lstlisting}

\par Se crea el repositorio. Si Tomcat nos muestra un error por pantalla al acceder a la URL \texttt{https://test.scstack.org/archiva/admin/addRepository!commit.action} relacionado con \textbf{NullPointerException} no nos debemos preocupar ya que el repositorio está creado correctamente siendo accesible y funcional. Se puede comprobar volviendo a visualizar la lista de repositorios de \emph{Archiva}.

\paragraph{Usuarios}

\par Archiva no lee los usuarios de \emph{OpenLDAP}, por tanto es necesario añadirlos a mano. En principio, debería ser suficiente con un \textbf{usuario de deploy} para \emph{toda la organización}, o como mucho, un usuario por proyecto o grupo de proyectos.

\par El usuario debe ser \textbf{Observer} de los tres repositorios y \emph{\textbf{manager} de} snapshots y releases.

\section{FAQ}
\label{sec:faq}

\par Posibles problemas que se pueden encontrar tras la instalación:

\begin{itemize}
    \item Error al crear un repositorio Git: Reininciar el servicio scstack-service:
    \lstset{style=rubybasico}
    \begin{lstlisting}[frame=trbl]
        $sudo service scstack-service stop
        $sudo service scstack-service start
    \end{lstlisting}

    \item Error al acceder a \textbf{Archiva} o \textbf{Jenkins} \texttt{404 No encontrado}.
    \begin{itemize}
        \item Configurar el dominio de nombres en el ordenador para que acceda a través de la ip correspondiente.

        \lstset{style=rubybasico}
        \begin{lstlisting}[frame=trbl]
            $ sudo vi /etc/hosts
        \end{lstlisting}

        \item Añadir la línea de conversión entre IP y nombre (elegido en la configuración del fichero default.pp).

        \lstset{style=rubybasico}
        \begin{lstlisting}[frame=trbl]
            138.100.156.246 sidelabcode03.scstack.org
        \end{lstlisting}
     \end{itemize}
\end{itemize}


%%%%%%%%%%%%%%%%
% BIBLIOGRAFIA %
%%%%%%%%%%%%%%%%

\begin{comment}
    Referenciar bibliografía: reference 1 ~\cite{New-commercial-OSS-standford-2010}.
\end{comment}

\begin{thebibliography}{25}
\bibliographystyle{alpha}

\bibitem{larman2003iterative} Larman, Craig and Basili, Victor R. \textit{Iterative and incremental developments. a brief history}. IEEE, 2003.

\bibitem{featurebranch} Fowler, Martin. \textit{Feature Branch}. http://martinfowler.com/bliki/FeatureBranch.html, September 2009.

% Forjas

\bibitem{cenatic-forjas} Jose Angel Diaz Diaz Manuel Velardo Pacheco. \textit{INFORME TÉCNICO Forjas: entornos de desarrollo colaborativo. Su integración en el ámbito empresarial.}. CENATIC (Centro Nacional de Referencia de Aplicación de las TIC basadas en fuentes abiertas), 2009.

\bibitem{open-collaboration-forges} Dirk Riehle, John Ellenberger, Tamir Menahem, Boris Mikhailovski, Yuri Natchetoi, Barak Naveh, Thomas Odenwald. \textit{Open Collaboration within Corporations Using Software Forges}. IEEE Software, March/April 2009.

\bibitem{yodemayorforja} Micael gallego. \textit{Mamá, yo de mayor quiero una forja (de desarrollo software)}. \url{http://sidelab.wordpress.com/2011/07/05/mama-yo-de-mayor-quiero-una-forja-de-desarrollo-software/}, 2011.

\bibitem{google-code-open} Jonathan Rosenberg. \textit{The meaning of open}. Google Official Blog \url{http://googleblog.blogspot.com.es/2009/12/meaning-of-open.html}, December 2009.

\bibitem{comparativa-forjas-wiki} Wikipedia. \textit{Comparison of open-source software hosting facilities}. \url{http://en.wikipedia.org/wiki/Comparison_of_open_source_software_hosting_facilities}, 2013.

\bibitem{bitergia-fusionforge-analysis} Jesus M. Gonzalez-Barahona, \textit{The history of FusionForge and GForge}. \url{http://blog.bitergia.com/2012/11/16/the-history-of-fusionforge-and-gforge/}, Bitergia's blog, November 16 2012

\end{thebibliography}

\end{document}
