\documentclass[11pt]{scrartcl}
\usepackage[a4paper, left=2.5cm, right=2.5cm, top=3cm, bottom=3cm]{geometry}\usepackage[parfill]{parskip}
\usepackage{graphicx}
\usepackage{booktabs}
\usepackage{tabulary}
\usepackage{float}
\usepackage{hyperref}
\usepackage[nottoc, notlot, notlof, notindex]{tocbibind} %% Opciones de índice
\usepackage{latexsym}  %% Logo LaTeX


\graphicspath{{img/}}

\renewcommand{\baselinestretch}{1.5}  %% Interlineado\underline{}

\title{\textbf{SidelabCode Stack ALM Tools}}
\author{Autor: Ricardo Garc\'ia Fern\'andez
\\Tutor: }

\date{\today}

\begin{document}

\maketitle

\vspace{2cm}

\begin{figure}[h]
    \begin{center}
        \includegraphics{urjc}
        \label{fig:urjc}
    \end{center}
\end{figure}

\begin{center}
\large
M\'aster Universitario en Software Libre
\\ Proyecto Fin de M\'aster
\\Curso Acad\'emico 2012/2013
\end{center}

\vfill

\begin{flushright}
    \copyright  2013 Ricardo Garc\'ia Fern\'andez - ricardogarfe [at] gmail [dot] com.

    This work is licensed under a Creative Commons 3.0 Unported License.
    To view a copy of this license visit:
 
    \url{http://creativecommons.org/licenses/by/3.0/legalcode}.
\end{flushright}

\begin{figure}[h]
    \begin{flushright}	
        \includegraphics{by}
        \label{fig:by}
    \end{flushright}
\end{figure}

\newpage

%%%%%%%%%%%%%%%%%%%%%%%%%%%%%%%%%%%%%%

%\tableofcontents  %% Creando índice

%\newpage

%\listoffigures  %% índice de figuras

%\newpage

%\listoftables %% índide de tablas

%\newpage

%%%%%%%%%%%%%%%%%%%%%%%%%%%%%%%%%%%%%%

%\chapter{Introducci\'on}
\label{chap:introduccion}

\par \textbf{SidelabCode Stack} es una forja de desarrollo de Software para su uso como herramienta \textbf{ALM}. Es una herramienta FLOSS (Free Libre Open Source Software) con Licencia \textbf{TBC}.

\begin{figure}[h]
    \begin{center}	
        \includegraphics[width=1\textwidth]{sidelab}
        \label{fig:sidelab}
    \end{center}
\end{figure}

\par El desarrollo de este proyecto se basa en el dise\~no e implementaci\'on del proceso de desarrollo acorde con las metodolog\'ias \'agiles a trav\'es de la forja \emph{SidelabCode Stack}. Unificando herramientas a trav\'es de las distintas APIs basadas en la interoperabilidad, facilitando la instalaci\'on, la replicaci\'on y la recuperaci\'on de los datos mediante el uso de Software Libre para construir Software de calidad.

\par El uso de metodolog\'ias \'agiles se encuentra intr\'insecamente relacionado a lo largo del proceso de desarrollo e implementaci\'on de la forja. Por otra parte podemos afirmar que no es una herramienta intrusiva para la aplicaci\'on de las distintas metodolog\'ias ligadas a un proceso de desarrollo.

\par Presentando las distintas metodolog\'ias de desarrollo de software definidas analizando sus caracter\'isticas tanto las buenas como las malas y como \'estas se aplican al dise\~no de la herramienta encaminado un proceso de desarrollo fluido y \'agil para un desarrollador o un grupo de desarrolladores.

\par Un punto a tener en cuenta es la ergonom\'ia con la que cuenta la herramienta para el d\'ia a d\'ia y las soluciones que aporta con respecto a otras herramientas que cohabitan en el mismo campo.

\par Todo con un mismo fin; producir software de calidad evaluable, es decir, no es necesario crear un producto perfecto sino se sabe mejorarlo y adaptarlo a nuevas necesidades. Para eso utilizamos las herramientas ALM en los proyectos ya que nos ayudan a tener una visi\'on diaria y a posteriori con perspectiva para poder medir y evaluar las mejoras entre dos puntos de tiempo. Por lo tanto \emph{la evoluci\'on del proyecto y la adaptabilidad a los cambios}.

\section{Etimolog\'ia}
\label{sec:etimologia}

\par Sidelab es, un laboratorio de software. Partiendo de esta base, encontramos una definici\'on exacta para SidelabCode \footnote{\url{http://code.sidelab.es/projects/sidelab/wiki/Sidelab}}:

\begin{quotation}
        \emph{Sidelab es el "laboratorio de software y entornos de desarrollo integrados" (Software and Integrated Development Environments Laboratory). Es un grupo de entusiastas de la programaci\'on con inter\'es en pr\'acticamente todos los aspectos del desarollo, desde los lenguajes de programaci\'on y los algoritmos avanzados, hasta la ingenier\'ia del software y la seguridad inform\'atica. Nuestros principales intereses se centran en el desarrollo software y la mejora y personalizaci\'on de los entornos de desarrollo integrados (IDEs) y herramientas relacionadas.}
\end{quotation}

\par En el otro extremo del nombre se encuentra \emph{Code} se refiere al c\'odigo en s\'i hacia donde se orienta esta herramienta. Por ultimo pero no menos importante tenemos \emph{Stack}, es una pila de servicios para la gesti\'on y el desarrollo de proyectos software.

\par Como resultado tenemos \textbf{SidelabCode Stack} una Forja de desarrollo de aplicaciones orientada a un proceso de desarrollo.

% subsection etimologia (end)

\section{Trabajo en TSCompany}
\label{sec:trabajo-tscompany}

\par Explicaci\'on del trabajo efectuado en TSCompany durante el desarrollo de la implementaci\'on de la Forja.

\par El pr\'acticum del M\'aster me dio la oportunidad de entrar a trabajar en la empresa TSCompany para cubrir la plaza de becario. Desde el principio he estado interesado en la producción de software de calidad, en las herramientas e control de versiones, el desarrollo orientado a tests, la documentación y de como unificar las herramientas para obtener un rendimiento óptimo a la hora de desarrollar un proyecto, es decir, un desarrollador que desarrollar al 100\%.

\par Profesionalmente siempre he tenido buenos ejemplos cercanos acerca del desarrollo de software de calidad y me ha generado mucho he ido avanzando hacia 

% subsection trabajo-tscompany (end)

%%%%%%%%%%%%%%%%%%%%%%%%%%%%%%%%%%%%%%

%\section{ALM Tools}
\label{sec:almtools}

\par ALM Tools significa: \emph{Application Lifecycle Management Tools}. La gesti\'on del ciclo de vida de una aplicaci\'on, el conjunto de herramientas encargadas de gu\'iar al desarrollador a trav\'es de un camino basado en metodolog\'ias para la creaci\'on de un Software hacia su estado del arte.

\par Integrar el proceso de desarrollo de Software a trav\'es de las ALM Tools como veh\'iculo, es decir no existe en si mismo una herramienta ALM, sino que la herramienta ALM gobierna a las herramientas incluidas en el proceso de desarrollo.

\par Este conjunto de herramientas se puede dividir en varios grupos:

\begin{itemize}
	\item Gesti\'on de Requisitos.
	\item Arquitectura.
	\item Desarrollo.
	\item Test.
	\item Issue tracking system.
	\item Continuous Integration.
	\item Release Management.
\end{itemize}

\par Hoy en d\'ia las forjas ALM abundan, adem\'as de gozar de una gran popularidad entre los proyectos de Software, como podemos ver en los casos de SourceForge, Googlecode y Github (más adelante discutiremos cada proyecto). En este caso ALM Tools as a Service, debido al servicio que ofrecen, pero sólo las que son FLOSS permiten replicar ese mismo entorno en tu propia máquina, un dato muy importante a tener en cuenta, porque siempre se ha de mirar hacia adelante.

%%%%%%%%%%%%%%%%%%%%%%%%%%%%%%%%%%%%%%

%\include{./003-historia}

%%%%%%%%%%%%%%%%%%%%%%%%%%%%%%%%%%%%%%

\section{Motivaci\'on}
\label{sec:motivacion}

\par En el mismo punto, damos la bienvenida a SidelabCode Stack, la herramienta ALM para el proceso de desarrollo. 

\par Empezaremos con la ubicación de la forja de desarrollo SidelabCode Stack. ¿ Cual es la motivación que nos lleva a implementar un nuevo ALM ? después de haber pasado por las distintas fases como en el caso de los Frameworks.

\par Debido a la necesidad de tener un esqueleto para el desarrollo de un proyecto partiendo de herramientas de Software Libre actuales generando una nueva herramienta de Software Libre.

\par Partiendo del principio DRY (\emph{Don't Repeat Yourself} y de la mano de \emph{No reinventar la rueda} el proyecto surgió con la necesidad de crear una forja de desarrollo Libre y usable en distintos entornos.

\par Después de analizar Las posibles soluciones existentes, se optó por la unificación de las herramientas evaluadas para la generación de una nueva forja. 

\begin{itemize}
	\item ClinkerHQ \url{http://clinkerhq.com/} - Privado.
	\item Github - \url{http://github.com/} - Privado.
	\item SourceForge con Allura - \url{http://sourceforge.net/projects/allura/} - Software Libre pero incompleta (integrated Wiki, Tracker, SCM (svn, git and hg), Discussion, and Blog tools).
	\item Cloudbees DEV@Cloud \url{http://www.cloudbees.com/dev.cb} - Privado.
	\item CollabNet con CloudForge \url{http://www.cloudforge.com/} - Privado.
	\item Plan.io - \url{http://plan.io/en/} - Privado.
	\item Bitnami - \url{http://bitnami.com/} - Privado
\end{itemize}

\par Después de analizar estas soluciones ALM existentes se optó por crear una nueva Forja a partir de las herramientas necesarias para el proceso de desarrollo de Software de Calidad, siguiendo las metodologías ágiles existentes.

\par La integración del proceso de desarrollo en la implementación de una Forja ALM es el punto clave de SidelabCode Stack.

%%%%%%%%%%%%%%%%%%%%%%%%%%%%%%%%%%%%%%

\section{Objetivos}
\label{sec:objetivos}

\par Constancia, metodolog\'ia, facilidad de uso, herramientas al alcance de cualquier desarrollador dentro de un proceso de desarrollo para crear software de calidad.

\par Generar datos aunque no se utilicen, es decir, grano gordo/grano fino.

\par 

%%%%%%%%%%%%%%%%%%%%%%%%%%%%%%%%%%%%%%

\section{Proceso de desarrollo}
\label{sec:procesodesarrollo}

\par Descripci\'on del proceso de desarrollo de software para crear el diseño de implementaci\'on de la forja SdelabCode Stack.


%%%%%%%%%%%%%%%%%%%%%%%%%%%%%%%%%%%%%%

\section{Dise\~no e Implementaci\'on}
\label{sec:diseno}

\par Herramientas que se han utilizado y porqu\'e.

% section diseno (end)

\subsection{Interoperabilidad}
\label{sub:interoperabilidad}

\par Interoperabilidad entre las herramientas que componen la forja. API Rest.

% subsection interoperabilidad (end)
%%%%%%%%%%%%%%%%%%%%%%%%%%%%%%%%%%%%%%

\section{Comunidades FLOSS}
\label{sec:comunidades}

\par El punto diferenciador en el proyecto haciendo hincapi\'e en la interacci\'on con las comunidades de software de cada una de las herramientas.

%%%%%%%%%%%%%%%%%%%%%%%%%%%%%%%%%%%%%%

\section{Pruebas y Validaci\'on}
\label{sec:pruebas}

\par Pruebas a través de virtualizaci\'on de sistemas operativos mediante herramientas libres; Vagrant, kvm, puppet.

%%%%%%%%%%%%%%%%%%%%%%%%%%%%%%%%%%%%%%

\section{Caso Real}
\label{sec:casoreal}

\par La herramienta SidelabCode Stack se encuentra en uso para los estudiante de la UPV y dentro de la empresa TSCompany SL.

%%%%%%%%%%%%%%%%%%%%%%%%%%%%%%%%%%%%%%

\section{Conclusiones}
\label{sec:conclusiones}

\par El uso de estas herramientas y su incremento de la calidad en el desarrollo, por encima de todo siendo FLOSS debido a eso la versatilidad que otorga en el momento de unificarlas en una herramienta nueva; SidelabCode Stack.

%%%%%%%%%%%%%%%%%%%%%%%%%%%%%%%%%%%%%%

\section{Lecciones aprendidas}
\label{sec:lecciones}

\par Colaboraci\'on entre distintos proyectos y comunidades, interoperabilidad entre herramientas, Forjas de desarrollo y los elementos m\'as comunes de las mismas.

%%%%%%%%%%%%%%%%%%%%%%%%%%%%%%%%%%%%%%

\section{Trabajo Futuro}
\label{sec:trabajofuturo}

\par Impulso de la comunidad a través de los canales habituales.

\par Integraci\'on y gesti\'on de nuevas herramientas comunes para los desarrolladores.

\par Centralizaci\'on de la instalaci\'on.

%%%%%%%%%%%%%%%%%%%%%%%%%%%%%%%%%%%%%%

\end{document}
