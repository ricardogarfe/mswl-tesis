\documentclass[11pt]{scrartcl}
\usepackage[a4paper, left=2.5cm, right=2.5cm, top=3cm, bottom=3cm]{geometry}\usepackage[parfill]{parskip}
\usepackage{graphicx}
\usepackage{booktabs}
\usepackage{tabulary}
\usepackage{float}
\usepackage{hyperref}
\usepackage[nottoc, notlot, notlof, notindex]{tocbibind} %% Opciones de índice
\usepackage{latexsym}  %% Logo LaTeX


\graphicspath{{img/}}

\renewcommand{\baselinestretch}{1.5}  %% Interlineado\underline{}

\title{\textbf{SidelabCode Stack ALM Tools}}
\author{Autor: Ricardo Garc\'ia Fern\'andez
\\Tutor: }

\date{\today}

\begin{document}

\maketitle

\vspace{2cm}

\begin{figure}[h]
    \begin{center}
        \includegraphics{urjc}
        \label{fig:by}
    \end{center}
\end{figure}

\begin{center}
\large
M\'aster Universitario en Software Libre
\\ Proyecto Fin de M\'aster
\\Curso Acad\'emico 2012/2013
\end{center}

\vfill

\begin{flushright}
    \copyright  2013 Ricardo Garc\'ia Fern\'andez - ricardogarfe [at] gmail [dot] com.

    This work is licensed under a Creative Commons 3.0 Unported License.
    To view a copy of this license visit:
 
    \url{http://creativecommons.org/licenses/by/3.0/legalcode}.
\end{flushright}

\begin{figure}[h]
    \begin{flushright}	
        \includegraphics{by}
        \label{fig:by}
    \end{flushright}
\end{figure}

\newpage

%%%%%%%%%%%%%%%%%%%%%%%%%%%%%%%%%%%%%%

\tableofcontents  %% Creando índice

\newpage

\listoffigures  %% índice de figuras

\newpage

\listoftables %% índide de tablas

\newpage

%%%%%%%%%%%%%%%%%%%%%%%%%%%%%%%%%%%%%%

\section{Introducci\'on}
\label{sec:introduccion}

\par Uso de Frameworks y forjas de desarrollo, ALM Tools.

%%%%%%%%%%%%%%%%%%%%%%%%%%%%%%%%%%%%%%

\section{Motivaci\'on}
\label{sec:motivacion}

\par Donde se ubica el FLOSS en el campo de las forjas y su importancia con el desarrollo de software de calidad.

%%%%%%%%%%%%%%%%%%%%%%%%%%%%%%%%%%%%%%

\section{Objetivos}
\label{sec:objetivos}

\par Constancia, metodolog\'ia, facilidad de uso, herramientas al alcance de cualquier desarrollador dentro de un proceso de desarrollo para crear software de calidad.

%%%%%%%%%%%%%%%%%%%%%%%%%%%%%%%%%%%%%%

\section{Proceso de desarrollo}
\label{sec:procesodesarrollo}

\par Descripci\'on del proceso de desarrollo de software para crear el diseño de implementaci\'on de la forja SdelabCode Stack.

\subsection{Dise\~no e Implementaci\'on}
\label{sub:diseno}

\par Herramientas que se han utilizado y porqu\'e.

% subsection diseno (end)

\subsection{Interoperabilidad}
\label{sub:interoperabilidad}

\par Interoperabilidad entre las herramientas que componen la forja. API Rest.

% subsection interoperabilidad (end)

%%%%%%%%%%%%%%%%%%%%%%%%%%%%%%%%%%%%%%

\section{Comunidades FLOSS}
\label{sec:comunidades}

\par El punto diferenciador en el proyecto haciendo hincapi\'e en la interacci\'on con las comunidades de software de cada una de las herramientas.

%%%%%%%%%%%%%%%%%%%%%%%%%%%%%%%%%%%%%%

\section{Pruebas y Validaci\'on}
\label{sec:pruebas}

\par Pruebas a través de virtualizaci\'on de sistemas operativos mediante herramientas libres; Vagrant, kvm, puppet.

%%%%%%%%%%%%%%%%%%%%%%%%%%%%%%%%%%%%%%

\section{Caso Real}
\label{sec:casoreal}

\par La herramienta SidelabCode Stack se encuentra en uso para los estudiante de la UPV y dentro de la empresa TSCompany SL.

%%%%%%%%%%%%%%%%%%%%%%%%%%%%%%%%%%%%%%

\section{Conclusiones}
\label{sec:conclusiones}

\par El uso de estas herramientas y su incremento de la calidad en el desarrollo, por encima de todo siendo FLOSS debido a eso la versatilidad que otorga en el momento de unificarlas en una herramienta nueva; SidelabCode Stack.

%%%%%%%%%%%%%%%%%%%%%%%%%%%%%%%%%%%%%%

\section{Lecciones aprendidas}
\label{sec:lecciones}

\par Colaboraci\'on entre distintos proyectos y comunidades, interoperabilidad entre herramientas, Forjas de desarrollo y los elementos m\'as comunes de las mismas.

%%%%%%%%%%%%%%%%%%%%%%%%%%%%%%%%%%%%%%

\section{Trabajo Futuro}
\label{sec:trabajofuturo}

\par Impulso de la comunidad a través de los canales habituales.

\par Integraci\'on y gesti\'on de nuevas herramientas comunes para los desarrolladores.

\par Centralizaci\'on de la instalaci\'on.

%%%%%%%%%%%%%%%%%%%%%%%%%%%%%%%%%%%%%%

\end{document}
