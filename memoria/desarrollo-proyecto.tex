%%%%%%%%%%%%%%%%%%%%%%%%%%%%%%%%%%%%%%%%%%%%%%%%%%%%%%%%%%%%%%%%%%%%%%%%%%%%%%%%%%%%%%%%%%%%%%%%%%%%%%%%%%%
%   \copyright 2013 Ricardo García Fernández - ricardogarfe [at] gmail [dot] com.
%
%    This work is licensed under a Creative Commons 3.0 Unported License.
%    To view a copy of this license visit:
% 
%    http://creativecommons.org/licenses/by/3.0/legalcode
%%%%%%%%%%%%%%%%%%%%%%%%%%%%%%%%%%%%%%%%%%%%%%%%%%%%%%%%%%%%%%%%%%%%%%%%%%%%%%%%%%%%%%%%%%%%%%%%%%%%%%%%%%

\chapter{Desarrollo de un proyecto}
\label{chap:desarrollo}

\par Cómo llevar a cabo el desarrollo de un proyecto con SCStack. Desde la creación de los usuarios y el proyecto hasta la definición del camino a seguir aplicando el desarrollo Iterativo e Incremental como los raíles del proceso. Generación de las distintas ramas del desarrollo adecuando la integración continua y marcando los objetivos dentro de una iteración a través de un grupo de tareas.

\par \textbf{Nota:}

\begin{quote}
    \emph{Si se quiere reproducir este proceso primero se ha de instalar la forja SCStack siguiendo los pasos descritos en el Apéndice ~\ref{app:instalacion-sidelab}.}
\end{quote}

\begin{comment}
Desarrollo en paralelo de un proyecto mediante github + travis-ci vs gerrit + jenkins.
\end{comment}

\section{Alta usuarios}
\label{sec:alta-usuarios}

\par El primer paso dar de alta a los usuarios a través de la Consola de Administración de la forja e incluirlos dentro de un grupo.

\par Se ha de aportar el nombre de usuario, correo electrónico y la contraseña. Ya se han creado los usuarios y éstos tienen acceso a todas las herramientas de SCStack.

\par \textbf{Nota:}

\begin{quote}
    \emph{Los nombres de usuario y proyecto han de cumplir con las reglas de usuarios propias de Redmine, como indica la validación de campos en el formulario.}
\end{quote}

% section alta-usuarios (end)

\section{Crear un proyecto}
\label{sec:crear-proyecto}

\par El siguiente paso es crear el proyecto.

\par Elegimos el nombre del proyecto, los usuarios y el grupo al que pertenecen.

\par Por último se elije si el proyecto ha de tener un repositorio, en este caso \textbf{si} y además ha de ser del tipo \textbf{Git}. Si el repositorio es público o privado no es vinculante para el proceso.

\par Ya disponemos de usuarios, grupos y proyecto al que pertenecen además de un repositorio Git asociado al nombre del nuevo proyecto que contiene dos ramas de desarrollo: \emph{master} y \emph{development}.

\subsection{Proyecto en redmine}
\label{sub:proyeto-redmine}

\par El administrador del nuevo proyecto accede a Redmine utilizando sus credenciales.

\par Se añade la lista de tareas a la Lista de Control.

\par Se define la historia de usuario a través de la interfaz del plugin \emph{Backlogs}.

\par Se crea el Sprint con las tareas seleccionadas a través del plugin \emph{Backlogs}.

\par Ya tenemos la lista de tareas que se van a completar en la primera iteración del proceso.

\par En el siguiente paso se prepara el entorno para los desarrolladores.

% subsection proyeto-redmine (end)

\subsection{Repositorio git}
\label{sub:repo-git}

\par El usuario, en este caso el desarrollador recibe sus credenciales para identificarse como tal en su equipo de trabajo.

\par El entorno de trabajo para el desarrollador en este caso consta de:

\begin{itemize}
	\item Equipo de trabajo con una distribución Ubuntu 12.04 LTS.
	\item Java JDK 1.6.
	\item Maven.
	\item Git-scm.
	\item Eclipse STS.
\end{itemize}

\par Configurar usuario Git del desarrollador de la forja:

\lstset{style=bashbasico}
\begin{lstlisting}[frame=trbl]
$ cat .gitconfig 
[user]
    name = ricardogarfe
    email = ricardogarfe@gmail.com
$ git config --global user.name "ricardogarfe"
$ git config --global user.email "ricardogarfe@gmail.com"
$ cd [path-to-gitrepo]
[path-to-gitrepo]$ git config user.name "ricardogarfe"
[path-to-gitrepo]$ git config user.email "ricardogarfe@gmail.com"
\end{lstlisting}

\par Comprobar las credenciales de Git en el ordenador del desarrollador:

\lstset{style=bashbasico}
\begin{lstlisting}[frame=trbl]
$ git config --list
    user.name=ricardogarfe
    user.email=ricardogarfe@gmail.com
\end{lstlisting}

% subsection repo-git (end)

\subsection{Configuración de Jenkins}
\label{sub:jenkins-configuracion}

\par Configuracón de Jenkins para realizar determinadas tareas de forma automática:

\begin{itemize}
	\item Tags
	\item Construir las versiones vivas
\end{itemize}

\par También proveerá tareas para ser ejecutadas manualmente:

\begin{itemize}
	\item Branches de releases
	\item Desplegar una versión específica con un clic
\end{itemize}

\par Las versiones vivas vivirán en su propia máquina virtual

\begin{figure}[H]
    \centering
    \includegraphics[width=0.6\textwidth]{jenkins-git}
    \caption{Jenkisn-Git relación en el proceso de integración}
    \label{fig:jenkins-git}
\end{figure}

\par Se mantendrán diferentes versiones vivas a la vez para cumplir unos objetivos con forme al desarrollo Iterativo e Incremental: \emph{'Release early, release often'}:

\begin{itemize}
	\item Asegurar la calidad
	\item Hacer el despliegue ágil
	\item Minimizar el riesgo
\end{itemize}

\par Para orientar la integración continua al proceso de desarrollo se ha de configurar \emph{Jenkins} para acceder a múltiples repositorios Git para esto se han de tener en cuenta los siguientes requerimientos:

\begin{itemize}
    \item \emph{Jenkins} construye diferentes proyectos en donde en proyecto puede tener su propio repositorio git.
    \item \emph{Jenkins} debe tener permisos de lectura o lectura/escritura a todos los repositorios de aquellos proyectos que vaya a construir.
    \item Por defecto \emph{Jenkins} usa la clave del usuario en \texttt{\ensuremath{\sim.ssh}} para autenticarse
    \item ¿Con qué usuario se ejecuta Jenkins? con el usuario \textbf{tomcat}.
\end{itemize}

\par El acceso de Jenkins a los distintos repositorios se configura a partir de un usuario por Jenkins repositorio, aislando los problemas descritos en la sección~\ref{sub:ci-jenkins} del capítulo \nameref{chap:procesos-desarrollo}.

\par Configurar \emph{Jenkins} para acceso a \emph{múltiples repositorios git} creando el usuario necesario en la Consola de Administración.

\begin{figure}[H]
    \centering
    \includegraphics[width=\textwidth]{gestion-jenkins-usuarios-000}
    \caption{Crear usuarios para Jenkins a través de la consola de Administración}
    \label{fig:gestion-jenkins-usuarios-000}
\end{figure}

\par Generar un par de claves pública/privada con \texttt{ssh-keygen} para cada usuario en diferentes ficheros y acceder a Gerrit con cada usuario creado.

\begin{figure}[H]
    \centering
    \includegraphics[width=\textwidth]{gestion-jenkins-usuarios-001}
    \caption{Configuración de usuarios Jenkisn en Gerrit}
    \label{fig:gestion-jenkins-usuarios-001}
\end{figure}

\par Añadir la clave pública para este usuario para copiar las claves al servidor de Jenkins en el directorio\texttt{'/opt/ssh-keys'}.

\lstset{style=bashbasico}
\begin{lstlisting}[frame=trbl]
$ git config --list
$ cd /opt/ssh-keys
$ ll
total 24
drwxr-xr-x  2 tomcat tomcat 4096 Jan  4 09:46 ./
drwxr-xr-x 14 root   root   4096 Jan  4 09:42 ../
- rw-------  1 tomcat tomcat 1679 Jan  4 09:46 filetransferci_rsa
- rw-r--r--  1 tomcat tomcat  398 Jan  4 09:46 filetransferci_rsa.pub
- rw-------  1 tomcat tomcat 1679 Jan  4 09:44 samplegitci_rsa
- rw-r--r--  1 tomcat tomcat  396 Jan  4 09:44 samplegitci_rsa.pub
</code>
\end{lstlisting}

\par Configurar SSH con la clave correcta creando el fichero \texttt{'/home/tomcat/.ssh/config'}.

\lstset{style=bashbasico}
\begin{lstlisting}[frame=trbl]
$ git config --list
$ cd /home/tomcat/.ssh
$ cat config
Host samplegit.ricardogarfe.sidelab.es
    HostName ricardogarfe.sidelab.es
    User samplegitci
    IdentityFile /opt/ssh-keys/samplegitci_rsa
Host filetransfer.ricardogarfe.sidelab.es
    HostName ricardogarfe.sidelab.es
    User filetransferci
    IdentityFile /opt/ssh-keys/filetransferci_rsa
\end{lstlisting}

\subsection{Configuración de builds}
\label{sub:jenkins-build-jobs}

\par Los builds de \emph{Jenkins} funcionan a través de la configuración de \textbf{jobs}. Por lo que vamos a definir los jobs necesarios para el proceso de integración continua. Se dividen en tres grupos:

\begin{itemize}
    \item Jobs de \textbf{integración} (read only).
        \begin{itemize}
            \item Descargan el código (checkout).
            \item Construyen.
            \item Pasan tests.
            \item Despliegan la versión construida en ``local''.
        \end{itemize}
    \item Jobs de \textbf{release} (read/write).
        \begin{itemize}
            \item Realizan los pasos anteriores y además.
            \item Tag si los tests pasaron.
            \item Push del tag al repositorio remoto.
        \end{itemize}
    \item Jobs de \textbf{despliegue} (read only).
        \begin{itemize}
            \item Descargan el binario del repositorio de binarios
            \item Desplegar
        \end{itemize}
\end{itemize}

\subsubsection{Job de integración}
\label{subs:jenkins-job-integracion}

\par Configurar el Job de integración mediante Maven para asegurar la fiabilidad dentro de cada iteración en el desarrollo:

\begin{itemize}
	\item Crear un \textbf{job} \emph{Maven}.
	\item Configurar el repositorio git:
        \begin{itemize}
	        \item \emph{ssh://filetransferci@filetransferci.code.tscompany.es/filetransfer}
	        \item Ssh leerá el fichero config y utilizará el fichero de claves correspondiente el host \texttt{filetransferci.code.tscompany.es}.
        \end{itemize}
	\item Añadir las ramas a construir (añadir nuevas ramas con el botón ``Add'')
        \begin{itemize}
	        \item development
	        \item release-0.1.1
        \end{itemize}
	\item Añadir el \texttt{user.email} y \texttt{user.name} que usará \emph{Jenkins}.
        \begin{figure}[H]
            \centering
            \includegraphics[width=0.7\textwidth]{jenkins-job-integracion}
            \caption{Crear Job integración en Jenkins}
            \label{fg:jenkins-job-integracion}
        \end{figure}
	\item Los resultados del build se pueden comprobar en: 
        \begin{itemize}
	        \item \texttt{/opt/jenkins/jobs/filetransfer/workspace}
	        \item Si es un proyecto \emph{Maven}, dentro del proyecto en la carpeta target estará el artefacto generado.
	        \item También se puede acceder vía web y descargar el workspace como un zip.
	        \item Los \textbf{tests} están en la carpeta \texttt{surfire-reports} del proyecto \emph{Maven}.
	        \item También pueden consultarse vía web accediendo al build y seleccionando \emph{'Resultado de los tests'}.
        \begin{figure}[H]
            \centering
            \includegraphics[width=\textwidth]{jenkins-job-resultados}
            \caption{Resultados de los test a través de la interfaz de Jenkins}
            \label{fig:jenkins-job-resultados}
        \end{figure}
        \end{itemize}
\end{itemize}

\subsection{Maven}
\label{sub:jenkins-maven}

\par \emph{Jenkins} permite construir proyectos \emph{Maven}.

\par En determinadas ocasiones los proyectos requieren configuraciones específicas. La información sensible suele ir en el fichero \texttt{settings.xml} en el \texttt{home} del usuario en su máquina de desarrollo.

\begin{itemize}
    \item Info de \textbf{autenticación para Archiva}.
    \item \textbf{Profiles}
\end{itemize}

\par En Jenkins esto se puede gestionar con el plugin \emph{``Config File Provider Plugin''}.

\begin{figure}[H]
    \centering
    \includegraphics[width=\textwidth]{jenkins-config-file-management}
    \caption{Configuración de Maven a través de Jenkins}
    \label{fig:jenkins-config-file-management}
\end{figure}

\par Podemos añadir cualquiera de los ficheros creados con \emph{Config File Management} en un \textbf{job}.

\begin{figure}[H]
    \centering
    \includegraphics[width=\textwidth]{jenkins-config-file-management-settings}
    \caption{Seleccionar archivo de configuración de Maven}
    \label{fig:jenkins-config-file-management-settings}
\end{figure}

\par Para los despliegues, si el certificado es \emph{autofirmado} es \textbf{necesario} generar un \textbf{truststore} a partir del certificado generado por el servidor\footnote{\url{http://www.liferay.com/web/neil.griffin/blog/-/blogs/fixing-suncertpathbuilderexception-caused-by-maven-downloading-from-self-signed-repository}}.

\par Este \textbf{truststore} debe incluirse en todos los \texttt{jdk} que utilice \emph{Jenkins} en la ruta indicada en el enlace anterior.

% subsection jobs-jenkins (end)

% section crear-proyecto (end)

\section{Proceso de desarrollo basado en ramas}
\label{sec:desarrollo-en-ramas}

\par Después completar la configuración del entorno de desarrollo y la forja SCStack, es el turno aplicar el proceso Iterativo e Incremental al código fuente a través del repositorio Git.

\par El flujo de trabajo a través de las ramas se representa en este \emph{Diagrama de Desarrollo} en el que intervienen e interactúan las distintas ramas durante el proceso de Integración dentro de cada Iteración:

\begin{figure}[H]
    \centering
    \includegraphics[width=0.6\textwidth]{flujo-desarrollo-git-000}
    \caption{Flujo de desarrollo Git basado en ramas}
    \label{fig:flujo-desarrollo-git-000}
\end{figure}

\par Se ha definido que el proceso de desarrollo basado en ramas parte de 2 ramas de forma continua:

\begin{itemize}
    \item \textbf{master:} Desarrollo limpio. Sólo versiones estables.
    \item \textbf{develop:} El desarrollo inicial de la versión actual tiene lugar aquí.
\end{itemize}

\par En cada iteración se han de gestionar las ramas para estabilización de las iteraciones, en este caso se utiliza el nombre de \emph{release} acompañado de la etiqueta numérica asociada:

\begin{itemize}
    \item \textbf{release-0.1}, \textbf{release-0.2}; una rama de estabilización cada vez   
\end{itemize}

\par El siguiente proceso se conoce como \emph{proceso de estabilización} y se gestiona a través de las \emph{ramas de estabilización}. Es un paso intermedio de integración del as ramas de estabilización (release-0.1, 0.2) y las ramas continuas (master y development):

\begin{itemize}
    \item Estabilización del código (\emph{RC release candidates})
    \item Arreglar bugs (hotfixes)
    \item Cuando la versión se considera estable se procede al siguiente paso:
        \begin{itemize}
            \item Tag
            \item Mezclar (merge) con development
            \item Mezclar (merge) con master
        \end{itemize}
    \item Si surgen nuevos bugs se vuelve a repetir el \textbf{proceso de estabilización}:
        \begin{itemize}
            \item Se arreglan en la misma rama (release-0.1)
            \item Nuevo tag y mezcla
        \end{itemize}
\end{itemize}

\subsection{Releasing}

\par Para crear una \emph{Release} se define un proceso de gestión a través las ramas:

\begin{itemize}
    \item Checkout del tag
    \item Build (Jenkins)
    \item Deploy (Jenkins)
\end{itemize}

% section desarrollo-en-ramas (end)

%%%%%%%%%%%%%%%%%%%%%%%%%%%%%%%%%%%%%%