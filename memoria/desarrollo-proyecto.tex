\chapter{Desarrollo de un proyecto}
\label{chap:desarrollo}

\par Cómo llevar a cabo el desarrollo de un proyecto con SCStack. Desarrollo en paralelo de un proyecto mediante github + travis-ci vs gerrit + jenkins.

\section{Estructura de un proyecto}
\label{sec:estructura}

% section estructura (end)

\subsection{Código fuente}
\label{sub:codigo-fuente}

% subsection codigo-fuente (end)

\subsection{Tareas}
\label{sub:tareas}

% subsection tareas (end)

\subsection{Binarios}
\label{sub:binarios}

Binarios y/o fuentes empaquetados en lenguajes que no generan binarios.

% subsection binarios (end)

\section{Crear un proyecto}
\label{sec:crear-pryecto}

\subsection{Proyecto en redmine}
\label{sub:proyeto-redmine}

% subsection proyeto-redmine (end)

\subsection{Repositorio git}
\label{sub:repo-git}

% subsection repo-git (end)

\subsection{Jobs en Jenkins}
\label{sub:jobs-jenkins}

% subsection jobs-jenkins (end)


\subsection{Repositorios Archiva}
\label{sub:archiva}

\par repositorios de archiva para binarios.

% subsection archiva (end)
% section crear-pryecto (end)

\section{Alta usuarios}
\label{sec:alta-usuarios}

\par Dar de alta desarrolladores/Project Owners (usuarios de la forja).

% section alta-usuarios (end)

\section{Proceso de desarrollo basado en ramas}
\label{sec:desarrollo-en-ramas}

\par Proceso de desarrollo basado en ramas.

% section desarrollo-en-ramas (end)

%%%%%%%%%%%%%%%%%%%%%%%%%%%%%%%%%%%%%%