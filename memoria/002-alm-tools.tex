\section{ALM Tools}
\label{sec:almtools}

\par ALM Tools significa: \emph{Application Lifecycle Management Tools}. La gesti\'on del ciclo de vida de una aplicaci\'on, el conjunto de herramientas encargadas de gu\'iar al desarrollador a trav\'es de un camino basado en metodolog\'ias para la creaci\'on de un Software hacia su estado del arte.

\par Integrar el proceso de desarrollo de Software a trav\'es de las ALM Tools como veh\'iculo, es decir no existe en si mismo una herramienta ALM, sino que la herramienta ALM gobierna a las herramientas incluidas en el proceso de desarrollo.

\par Este conjunto de herramientas se puede dividir en varios grupos:

\begin{itemize}
	\item Gesti\'on de Requisitos.
	\item Arquitectura.
	\item Desarrollo.
	\item Test.
	\item Issue tracking system.
	\item Continuous Integration.
	\item Release Management.
\end{itemize}

\par Hoy en d\'ia las forjas ALM abundan, adem\'as de gozar de una gran popularidad entre los proyectos de Software, como podemos ver en los casos de SourceForge, Googlecode y Github (más adelante discutiremos cada proyecto). En este caso ALM Tools as a Service, debido al servicio que ofrecen, pero sólo las que son FLOSS permiten replicar ese mismo entorno en tu propia máquina, un dato muy importante a tener en cuenta, porque siempre se ha de mirar hacia adelante.