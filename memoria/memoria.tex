%% LaTeX template for the final master thesis
%%
%% Creating PDF: make pdf
%%


\documentclass[a4paper, 12pt]{book}
%\usepackage[T1]{fontenc}
\usepackage[a4paper, left=2.5cm, right=2.5cm, top=3cm, bottom=3cm]{geometry}
\usepackage{times}
\usepackage[latin1]{inputenc}
\usepackage[spanish]{babel}
\usepackage{url}
\usepackage[dvipdfm]{graphicx}
\usepackage{float}  %% H para posicionar figuras
\usepackage[nottoc, notlot, notlof, notindex]{tocbibind} %% Opciones de índice
\usepackage{latexsym}  %% Logo LaTeX

\title{Title of the Thesis}
\author{Name of the author}

\renewcommand{\baselinestretch}{1.5}  %% Interlineado



\begin{document}


\renewcommand{\refname}{Bibliografía}  %% Renombrando
\renewcommand{\appendixname}{Apéndice}


%%%%%%%%%%%%% PORTADA %%%%%%%%%%%%%%%%
\begin{titlepage}
\begin{center}
\begin{tabular}[c]{c c}
%\includegraphics[bb=0 0 194 352, scale=0.25]{logo} &
\includegraphics[scale=0.25]{img/logo_vect.eps} &
\begin{tabular}[b]{l}
\Huge
\textsf{UNIVERSIDAD} \\
\Huge
\textsf{REY JUAN CARLOS} \\
\end{tabular}
\\
\end{tabular}

\vspace{3cm}

\Large
Máster Universitario en Software Libre

\vspace{0.4cm}

\large
Curso Académico 2011/2012

\vspace{0.8cm}

Proyecto Fin de Máster

\vspace{2.5cm}

\LARGE
Título del Proyecto Fin de Máster

\vspace{4cm}

\large
Autor: Nombre del autor \\
Tutor: Dr. Nombre del tutor
\end{center}
\end{titlepage}
%%%%%%%%%%%%%%%%%%%%%%%%%%%%%%%%%%%%%%

\tableofcontents  %% Creando índice

\listoffigures  %% índice de figuras

\listoftables %% índide de tablas

%%%%%%%%%%%%%%%%%%%%%%%%%%%%%%%%%%%%%%

\chapter*{Resumen}
\label{chap:resumen}



%%%%%%%%%%%%%%%%%%%%%%%%%%%%%%%%%%%%%%

\chapter{Introducci\'on}
\label{chap:intro}

Como decimos en el capítulo~\ref{chap:intro}...

Véase la Fig.~\ref{fig:logo}

\begin{figure}[H]
  \centering
  \includegraphics[width=2cm, keepaspectratio]{img/logo_vect.eps}
  \label{fig:logo}
\end{figure}

Así se cita un libro de la bibliografía~\cite{BuddOO}.


\section{Secci\'on}
\label{sec:seccion}

\subsection{Subsecci\'on}
\label{subsec:subseccion}

\section{Estructura del documento}
\label{sec:estructura}



%%%%%%%%%%%%%%%%%%%%%%%%%%%%%%%%%%%%%%

\chapter{Objetivos}
\label{chap:objetivos}


%%%%%%%%%%%%%%%%%%%%%%%%%%%%%%%%%%%%%%

\chapter{Dise\~no e implementaci\'on}
\label{chap:diseno}


%%%%%%%%%%%%%%%%%%%%%%%%%%%%%%%%%%%%%%

\chapter{Pruebas y validaci\'on}
\label{chap:pruebas}


%%%%%%%%%%%%%%%%%%%%%%%%%%%%%%%%%%%%%%

\chapter{Conclusiones}
\label{chap:conclusiones}


\section{Lecciones aprendidas}
\label{sec:lecciones}


\section{Trabajos futuros}
\label{sec:futuro}


%%%%%%%%%%%%%%%%%%%%%%%%%%%%%%%%%%%%%%

\appendix
\chapter{Apéndice 1}
\label{app:primer}


%%%%%%%%%%%%%%%%
% BIBLIOGRAFIA %
%%%%%%%%%%%%%%%%

\begin{thebibliography}{25}
\bibliographystyle{alpha}

\bibitem{BuddOO} Timothy Budd. \textit{Introducci\'on a la programaci\'on orientada a objetos}. Addison-Wesley Iberoamericana, 1994.

\bibitem{LutzPPy} Mark Lutz. \textit{Programming Python}. O'Reilly, 2001.

\bibitem{LundhPSL} Fredrik Lundh. \textit{Python standard library}. O'Reilly,
2001.

\bibitem{ReeseSQL} George Reese. \textit{Managing and using MySQL}. O'Reilly,
2002.

\end{thebibliography}


\end{document}
